\begin{tabularx}{\textwidth}{Xc}
    \arrayrulecolor{OliveGreen}
    \toprule
    {\bfseries Zu erreichende Ziele der aktuellen Woche} & {\bfseries Erreicht} \\
    \midrule[2pt]
    Analyse der Struktur und Inhalt einer Grafana Anfrage bei der API  & Ja \\
    \rowcolor{OliveGreen!15}
    Analyse der Struktur und Inhalt einer Grafana Antwort von der API  & Ja \\
    \rowcolor{White}
    Analyse der verwendeten Zeitzonen im Grafana FrontEnd, Grafana BackEnd, bei
    der API und bei dem Datenhaltungssystem  & Ja \\
    \bottomrule[2pt]
\end{tabularx}
%
\vspace{1cm}
%
\begin{tabularx}{\textwidth}{Xc}
    \arrayrulecolor{OliveGreen}
    \toprule
    {\bfseries Ziele für die nächste Woche}              &                   \\
    \midrule[2pt]
    Erstellung der Grundbasis des ETL-Prozesses in PostgresSQL & \\
    \rowcolor{OliveGreen!15}
    Erstellung von einer Fehler Handhabung in Form einer Log Tabelle & \\
    \rowcolor{White}
    Hinzufügen der Dateninput-Prüfung im ETL-Prozess nach den collectd Typen & \\
    \rowcolor{OliveGreen!15}
    Erstellung eines Repositoriums (Metadatenbank) für den ETL-Prozess & \\
    \rowcolor{White}
    Redefinierung der automatischen Partitionierung von Tagen auf Stunden & \\
    \rowcolor{OliveGreen!15}
    Erstellung eines Zeitstempel Converter für den Umgang der verschiedenen 
    Zeitzonen und angelieferten Zeitwerten von Grafana und Postgres & \\
    \rowcolor{White}
    Einfügen des Repositoriums (Metadatenbank) in den ETL-Input-Porzess & \\
    \rowcolor{OliveGreen!15}
    Implementierung der automatischen Partiitionierung in den ETL-Prozess & \\
    \rowcolor{White}
    Implementierung einer Daten-Aggregation im ETL-Prozess & \\
    \rowcolor{OliveGreen!15}
    Planung und Start des Sprints ``null point exception'' (01.05 - 15.05) & \\
\end{tabularx}
