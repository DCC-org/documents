\newcommand{\titel}{Wochenbericht~}
\newcommand{\untertitel}{Data Warehousing for Cloud Computing Metrics}

\newcommand{\projekttitel}{Data Warehousing for Cloud Computing Metrics}
\newcommand{\projektuntertitel}{blablabla}

\newcommand{\fachgebiet}{Software-Engineering}
\newcommand{\autor}{Tim Meusel}
\newcommand{\studienbereich}{Staatlich geprüfter Informatiker}

\newcommand{\jahr}{2016}
\newcommand{\ort}{Köln}
\newcommand{\kunde}{Puppet}

\newcommand{\leadingzero}[1]{\ifnum #1<10 0\the#1\else\the#1\fi}             %%fügt bei 1 bis 9 eine führende Null an
\newcommand{\todayD}{\leadingzero{\day}.\leadingzero{\month}.\the\year}

\usepackage{polyglossia}
\setdefaultlanguage[spelling=new]{german}

\usepackage{fontspec}
\setmainfont{TeX Gyre Pagella}
\setkomafont{disposition}{\bfseries}

\graphicspath{{../figures/}}

\usepackage[unicode=true]{hyperref}

\usepackage[usenames,dvipsnames]{xcolor}
\definecolor{hellgelb}{rgb}{1,1,0.9}
\definecolor{colKeys}{rgb}{0,0,1}
\definecolor{colIdentifier}{rgb}{0,0,0}
\definecolor{colComments}{rgb}{1,0,0}
\definecolor{colString}{rgb}{0,0.5,0}

\usepackage{colortbl}

%für leichtere Tabellenbreite Definition
\usepackage{tabularx}
\usepackage{booktabs}
\usepackage{threeparttable}

\usepackage{fancyhdr}
\pagestyle{fancy}

\fancyhf{}
\rhead{
  \small{SPS im Web}\\
  \small{Automatisierung, Konfiguration, Verwaltung und Monitoring}
}
\rhead{
  \small{\titel~\KW}\\
  \autor
}

\rfoot{
  \scriptsize{Seite \thepage~von~\pageref{LastPage}}
}
