\newcommand{\titel}{Wochenbericht~}
\newcommand{\untertitel}{Data Warehousing for Cloud Computing Metrics}

\newcommand{\projekttitel}{Data Warehousing for Cloud Computing Metrics}
\newcommand{\projektuntertitel}{blablabla}

\newcommand{\fachgebiet}{Software-Engineering}
\newcommand{\autor}{Tim Meusel}
\newcommand{\studienbereich}{Staatlich geprüfter Informatiker}

\newcommand{\jahr}{2016}
\newcommand{\ort}{Köln}
\newcommand{\kunde}{Puppet}

\newcommand{\leadingzero}[1]{\ifnum #1<10 0\the#1\else\the#1\fi}             %%fügt bei 1 bis 9 eine führende Null an
\newcommand{\todayD}{\leadingzero{\day}.\leadingzero{\month}.\the\year}

\usepackage[
    automark, % Kapitelangaben in Kopfzeile automatisch erstellen
    headsepline, % Trennlinie unter Kopfzeile
    ilines % Trennlinie linksbündig ausrichten
]{scrpage2}

\usepackage{polyglossia}
\setdefaultlanguage[spelling=new]{german}

\usepackage{fontspec}
\setmainfont{TeX Gyre Pagella}
\setkomafont{disposition}{\bfseries}

% Grafiken ---------------------------------------------------------------------
% Einbinden von JPG-Grafiken ermöglichen
\usepackage[dvips,final]{graphicx}
% hier liegen die Bilder des Dokuments
\graphicspath{{../Bilder/}}

% Befehle aus AMSTeX für mathematische Symbole z.B. \boldsymbol \mathbb --------
\usepackage{amsmath,amsfonts}

% für Index-Ausgabe mit \printindex --------------------------------------------
\usepackage{makeidx}

% Einfache Definition der Zeilenabstände und Seitenränder etc. -----------------
\usepackage{setspace}
\usepackage{geometry}

% Symbolverzeichnis ------------------------------------------------------------
%   Symbolverzeichnisse bequem erstellen. Beruht auf MakeIndex:
%     makeindex.exe %Name%.nlo -s nomencl.ist -o %Name%.nls
%   erzeugt dann das Verzeichnis. Dieser Befehl kann z.B. im TeXnicCenter
%   als Postprozessor eingetragen werden, damit er nicht ständig manuell
%   ausgeführt werden muss.
%   Die Definitionen sind ausgegliedert in die Datei "Glossar.tex".
% ------------------------------------------------------------------------------
\usepackage[intoc]{nomencl}
\let\abbrev\nomenclature
\renewcommand{\nomname}{Abkürzungsverzeichnis}
\setlength{\nomlabelwidth}{.25\hsize}
\renewcommand{\nomlabel}[1]{#1 \dotfill}
\setlength{\nomitemsep}{-\parsep}

% zum Umfließen von Bildern ----------------------------------------------------
\usepackage{floatflt}


% zum Einbinden von Programmcode -----------------------------------------------
\usepackage{listings}
\usepackage[usenames,dvipsnames]{xcolor} 
\definecolor{hellgelb}{rgb}{1,1,0.9}
\definecolor{colKeys}{rgb}{0,0,1}
\definecolor{colIdentifier}{rgb}{0,0,0}
\definecolor{colComments}{rgb}{1,0,0}
\definecolor{colString}{rgb}{0,0.5,0}
\lstset{
    float=hbp,
    basicstyle=\ttfamily\color{black}\small\smaller,
    identifierstyle=\color{colIdentifier},
    keywordstyle=\color{colKeys},
    stringstyle=\color{colString},
    commentstyle=\color{colComments},
    columns=flexible,
    tabsize=2,
    frame=single,
    extendedchars=true,
    showspaces=false,
    showstringspaces=false,
    numbers=left,
    numberstyle=\tiny,
    breaklines=true,
    backgroundcolor=\color{hellgelb},
    breakautoindent=true
}

\usepackage{colortbl}

% fortlaufendes Durchnummerieren der Fußnoten ----------------------------------
\usepackage{chngcntr}

% Formatierung von Listen ändern -----------------------------------------------
\usepackage{paralist}

% bei der Definition eigener Befehle benötigt
\usepackage{ifthen}

%für die benutzung von quotes
\usepackage[autostyle=true,german=quotes]{csquotes}

%für leichtere Tabellenbreite Definition
\usepackage{tabularx}
\usepackage{booktabs}
\usepackage{threeparttable}


%Seitennummerierung Lastpage
\usepackage{lastpage}

%\usepackage{fancyhdr}

% PDF-Optionen -----------------------------------------------------------------
\usepackage[
    bookmarks,
    bookmarksopen=true,
    colorlinks=true,
% diese Farbdefinitionen zeichnen Links im PDF farblich aus
    linkcolor=red, % einfache interne Verknüpfungen
    anchorcolor=black,% Ankertext
    citecolor=blue, % Verweise auf Literaturverzeichniseinträge im Text
    filecolor=magenta, % Verknüpfungen, die lokale Dateien öffnen
    menucolor=red, % Acrobat-Menüpunkte
    urlcolor=cyan, 
% diese Farbdefinitionen sollten für den Druck verwendet werden (alles schwarz)
    %linkcolor=black, % einfache interne Verknüpfungen
    %anchorcolor=black, % Ankertext
    %citecolor=black, % Verweise auf Literaturverzeichniseinträge im Text
    %filecolor=black, % Verknüpfungen, die lokale Dateien öffnen
    %menucolor=black, % Acrobat-Menüpunkte
    %urlcolor=black, 
    backref,
    plainpages=false, % zur korrekten Erstellung der Bookmarks
    pdfpagelabels, % zur korrekten Erstellung der Bookmarks
    hypertexnames=false, % zur korrekten Erstellung der Bookmarks
    %linktocpage % Seitenzahlen anstatt Text im Inhaltsverzeichnis verlinken
]{hyperref}
% Befehle, die Umlaute ausgeben, führen zu Fehlern, wenn sie hyperref als Optionen übergeben werden
\hypersetup{
    pdftitle={\titel \untertitel},
    pdfauthor={\autor},
    pdfcreator={\autor},
    pdfsubject={\titel \untertitel},
    pdfkeywords={\titel \untertitel},
    linktoc=all,
    colorlinks=false,
    hidelinks,
}

% Zeilenabstand 1,5 Zeilen -----------------------------------------------------
\onehalfspacing

% Seitenränder -----------------------------------------------------------------
\setlength{\topskip}{\ht\strutbox} % behebt Warnung von geometry
\geometry{	paper=a4paper,
			left=35mm,
			right=35mm,
			top=30mm,
			bottom=38mm,
			head=30mm,
    		headsep=2.5mm,
    		footskip=18mm
		}

% Kopf- und Fußzeilen ----------------------------------------------------------
\pagestyle{scrheadings}

\clearscrheadings
\clearscrplain
\clearscrheadfoot

% Kopf- und Fußzeile auch auf Kapitelanfangsseiten
\renewcommand*{\chapterpagestyle}{scrheadings} 

% Abstand bei Chapter herausnehmen
\renewcommand*{\chapterheadstartvskip}{\vspace*{-\topskip}} 
\renewcommand*{\chapterheadendvskip}{} 

% Schriftform der Kopfzeile
\renewcommand{\headfont}{\normalfont}

% Kopfzeile ----------------------------------------
\ihead{
			\small{SPS im Web}\\
			\small{Automatisierung, Konfiguration, Verwaltung und Monitoring}
}
\ohead{
			\small{\titel~\KW}\\
			\autor		
}

%\setlength{\headheight}{15mm} % Höhe der Kopfzeile

% Kopfzeile über den Text hinaus verbreitern
%\setheadwidth[0pt]{textwithmarginpar} 
%\setheadsepline[text]{1pt} % Trennlinie unter Kopfzeile


% Fußzeile ----------------------------------------
%\setlength{\footheight}{25mm}
%\setlength{\footskip}{5mm}

\setfootsepline[text]{2pt} % Trennlinie unter Kopfzeile


\cfoot{       
}
\ofoot{
\scriptsize{Seite \thepage~von~\pageref{LastPage}}
		%\vspace*{57mm}
}

% sonstige typographische Einstellungen ----------------------------------------

% erzeugt ein wenig mehr Platz hinter einem Punkt
\frenchspacing 

% Schusterjungen und Hurenkinder vermeiden
\clubpenalty = 10000
\widowpenalty = 10000 
\displaywidowpenalty = 10000

% Quellcode-Ausgabe formatieren
\lstset{numbers=left, numberstyle=\tiny, numbersep=5pt, breaklines=true}
\lstset{emph={square}, emphstyle=\color{red}, emph={[2]root,base}, emphstyle={[2]\color{blue}}}

% Fußnoten fortlaufend durchnummerieren
%\counterwithout{footnote}{chapter}

%section ist die höchste Ebene
\counterwithout{section}{chapter}

% auch subsubsection nummerieren
\setcounter{secnumdepth}{3}
\setcounter{tocdepth}{3}

%Counter zurücksetzen
\setcounter{section}{0}

% Eigene Befehle und typographische Auszeichnungen für diese

% einfaches Wechseln der Schrift, z.B.: \changefont{cmss}{sbc}{n}
\newcommand{\changefont}[3]{\fontfamily{#1} \fontseries{#2} \fontshape{#3} \selectfont}

% Abkürzungen mit korrektem Leerraum 
\newcommand{\ua}{\mbox{u.\,a.\ }}
\newcommand{\zB}{\mbox{z.\,B.\ }}
\newcommand{\dahe}{\mbox{d.\,h.\ }}
\newcommand{\Vgl}{Vgl.\ }
\newcommand{\bzw}{bzw.\ }
\newcommand{\evtl}{evtl.\ }
\newcommand{\bzgl}{bzgl.\ }
\newcommand{\ggf}{ggf.\ }
\newcommand{\etc}{etc.\ }

\newcommand{\abbildung}[1]{Abbildung~\ref{fig:#1}}

\newcommand{\bs}{$\backslash$}

% erzeugt ein Listenelement mit fetter Überschrift 
\newcommand{\itemd}[2]{\item{\textbf{#1}}\\{#2}}

% einige Befehle zum Zitieren --------------------------------------------------
\newcommand{\Zitat}[2][\empty]{\ifthenelse{\equal{#1}{\empty}}{\citep{#2}}{\citep[#1]{#2}}}

% zum Ausgeben von Autoren
\newcommand{\AutorName}[1]{\textsc{#1}}
\newcommand{\Autor}[1]{\AutorName{\citeauthor{#1}}}

% verschiedene Befehle um Wörter semantisch auszuzeichnen ----------------------
\newcommand{\NeuerBegriff}[1]{\textbf{#1}}
\newcommand{\Fachbegriff}[1]{\textit{#1}}

\newcommand{\Eingabe}[1]{\texttt{#1}}
\newcommand{\Code}[1]{\texttt{#1}}
\newcommand{\Datei}[1]{\texttt{#1}}

\newcommand{\Datentyp}[1]{\textsf{#1}}
\newcommand{\XMLElement}[1]{\textsf{#1}}
\newcommand{\Webservice}[1]{\textsf{#1}}
