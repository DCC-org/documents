\newcommand{\titel}{Wochenbericht~}
\newcommand{\untertitel}{Data Warehousing for Cloud Computing Metrics}

\newcommand{\projecttitle}{Data Warehousing for Cloud Computing Metrics}
\newcommand{\projektuntertitel}{blablabla}

\newcommand{\studienbereich}{Staatlich geprüfter Informatiker}

\newcommand{\ort}{Köln}
\newcommand{\kunde}{Puppet}

\newcommand{\leadingzero}[1]{\ifnum #1<10 0\the#1\else\the#1\fi}             %%fügt bei 1 bis 9 eine führende Null an
\newcommand{\todayD}{\leadingzero{\day}.\leadingzero{\month}.\the\year}

\usepackage{polyglossia}
\setdefaultlanguage[spelling=new]{german}

\usepackage{fontspec}
\setmainfont{TeX Gyre Pagella}
\setkomafont{disposition}{\bfseries}

%%\graphicspath{{../figures/}}

\usepackage[unicode=true]{hyperref}

\usepackage[usenames,dvipsnames]{xcolor}

\usepackage{amssymb}

\usepackage{colortbl}
\usepackage{tabularx}
\usepackage{booktabs}
\usepackage{threeparttable}

\usepackage{fancyhdr}
\pagestyle{fancy}

\fancyhf{}
\setlength{\headheight}{23pt} 
\renewcommand{\headrulewidth}{0pt}


\rhead{
  \small{SPS im Web}\\
  \small{Automatisierung, Konfiguration, Verwaltung und Monitoring}
}
\rhead{
  \small{\titel~\KW}\\
  \writer
}

\cfoot{\thepage}
