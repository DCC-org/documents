\section{Einführung}

Die Ansprechpartner für dieses Projekt ist das Unternehmen Puppet Inc. in der
Rolle als Auftraggeber, welches von den Mitarbeitern Herrn David Schmitt und
Herrn Steve Quin vertreten wird, sowie die Auftragnehmer Herrn Tim Meusel als
Teamleiter, Herrn Marcel Reuter als Teammitglied und Herrn Nikolai Luis
ebenfalls als Teammitglieder. Da diese Software im Rahmen eines
Abschlussprojektes des Studienganges der Schule Heinrich-Hertz-Europakolleg
durchgeführt wird, begleitet der Dozent Herr Dirk Stegemann das Projekt über
den gesamten Zeitraum und wird eine abschließende Beurteilung durchführen.

Cloud Provider (Unternehmen) nutzen die Werkzeuge der Virtualisierung von
Computersystemen, um aus einem physischen Host (Server) mehrere virtuelle
Instanzen (vServer) zu gewinnen. Diese virtuellen Server greifen somit auf die
gleichen verfügbaren Ressourcen des physischen Hosts zu. Durch eine oftmals
ausgeglichene Ressourcennutzung jedes virtuellen Servers ist diese Methode
wirksam, jedoch möchte jeder Cloud Provider ein einzelnes physisches System so
oft wie möglich aufteilen, ohne dass die virtuellen Server sich gegenseitig
die verfügbaren Ressourcen des Hosts Systems wegnehmen. Aufgrund Dessen wird
eine Software benötigt, welche die Nutzung der Ressourcen für jeden virtuellen
Server erfasst, dokumentiert und auswertet.


\section{Auftrag}

In diesem Projekt soll eine funktionierende Open Source Software entwickelt
werden, die sich in drei Teile gliedert:
\begin{enumerate}
  \item Die verschiedenen Ressourcetypen (CPU Zeit / Daten Durchsatz / RAM
    Auslastung / Speicher Auslastung / Netzwerk Durchsatz) der einzelnen
    virtuellen Server müssen in einem sinnvollen Intervall periodisch
    ermittelt werden.
  \item Die Daten müssen aggregiert und gespeichert werden. Hierbei ist auf
    eine Skalierung auf mindestens 10.000[a] virtuelle Instanzen unter
    Berücksichtigung der Verfügbarkeit und Performance der Datenbank zu achten.
  \item Diese Daten können dann dem Endanwender präsentiert werden (API und
    Web-UI). Hierzu wird eine Userstory Erhebung unter den drei Anwendertypen
    Kunde, Administrator, Manager bei Partnerunternehmen durchgeführt,
    um gewünschte Algorithmen zur Darstellung zu ermitteln.
\end{enumerate}

Die Daten sollen für jede virtuelle Instanz in einer oder mehreren Datenbanken
gespeichert werden, auf welchen anschließend verschiedene Analysen
durchgeführt werden können. Aufgrund der komplexität der Abfragen muss ein
Data Warehouse Ansatz in Betracht gezogen werden. Die Daten sollen in
Regelmäßigen Abständen von einer Sekunde, bis zu 5 Minuten von den Virtuellen
Instanzen abgefragt werden können. Dieser Intervall kann später über die
Web-UI selber definiert werden.

\section{Abgrenzung des zu entwickelnden Systems}

Es gibt fertige Open Source Lösungen für die einzelnen Punkte im Projekt.
Mindestens die folgenden Lösungen müssen evaluiert werden:

\subsection{Datenerfassungssysteme}

\begin{outline}
  \1 collectd mit Virt Plugin
  \1 coreutils
  \1 zabbix-agent
  \1 python-diamond
  \1 sysstat
  \1 atop
  \1 logtash
  \1 riemann
\end{outline}

\subsection{Datenbanksysteme}

\begin{outline}
  \1 elasticsearch
  \1 cassandra
  \1 postgres
  \1 OpenTSDB
  \1 KNIME
  \1 impala
  \1 hadoop
  \1 hive
\end{outline}

\subsection{Frontends}

\begin{outline}
  \1 grafana
  \1 graphite
  \1 zabbix-frontend
\end{outline}

Sofern dies möglich ist, sollen vorhandene Lösungen für Teilprobleme
kombiniert werden, anstatt diese selbst zu implementieren. Infrastruktur zum
Entwickeln und Testen wird von der Firma Host Europe GmbH und Travis CI GmbH
kostenlos zur Verfügung gestellt.

\section{Beschreibung der Schnittstellen}

In diesem Projekt finden mehrere Kommunikationen mit verschiedenen
Schnittstellen statt. Es beginnt mit der Kommunikation zwischen dem
physischen Host und den darauf betriebenen virtuellen Servern. Dazu wird eine
Software verwendet, welche parallel zur Virtualisierungssoftware betrieben
wird, um die verwendeten Ressourcen jedes virtuellen Servers lokal auszulesen
und kurzzeitig festhalten zu können. Anschließend wird die
Netzwerkschnittstelle des physischen Hosts Systems verwendet, um mit dem
Netzwerk des Rechenzentrums zu kommunizieren, dazu wird eine standardisierte
TCP/UDP Verbindung verwendet. Je nach Aufbau der späteren zentralen
Datenhaltung, wird jeder physische Server die Möglichkeit besitzen, mit einem
oder mehreren Datenbankservern zu kommunizieren. Diese Kommunikation kann auf
verschiedene Weisen stattfinden, welche im Laufe des Projektes validiert
werden müssen. Die voraussichtliche Schnittstellen zwischen den physischen
Host Systemen und der zentralen Datenhaltung könnten wie folgt aufgebaut sein:
Das Host System kommuniziert direkt:

\begin{outline}
  \1 mit einer einzelnen Datenbank (Datenhaltung)
  \1 mit einem Lastenverteiler (Server) und anschließend mit der zugeordneten
  Datenbank (Datenhaltung)
  \1 mit einem vor der Datenbank befindenden System, welches den ETL-Prozess
  übernimmt und anschließend die Daten in die Datenhaltung transportiert.
  \1 mit einem Messagebus-System um die Daten in die Datenhaltung zu
  transportieren.
\end{outline}
Im Laufe des Projektes können durch Validierung und Tests von verschiedenen
Datenhaltung-Systemen weitere Schnittstellen entstehen. Bei dem Zugriff auf
die gespeicherten Daten wird eine Software benötigt, welche Facharbeitern, wie
z.B. einem Datenanalyst oder Serveradministrator, es ermöglicht die Daten im
gewünschten Format über eine API abzurufen.

Jede dieser Schnittstellen übernimmt die Aufgabe des Senders oder Empfängers,
die Aufgabenverteilung soll in folgender Auflistung dokumentiert werden:
\begin{outline}
  \1 Die virtuelle Instanz auf einem physischen Host System
    \2 Sender
  \1 Das physische Host System
    \2 Sender sowie Empfänger
  \1 Die Datenhaltung, bzw. die definierte Lösung für die Datenpflege
    \2 Empfänger
  \1 Die in der Datenhaltung integrierte API
    \2 Sender der Daten, bzw. Empfänger einer Datenanfrage
  \1 Die Weboberfläche für Endanwender
    \2 Sender für eine Datenanfrage an der Datenhaltung, sowie die Darstellung
    auf dem Endgerät des Endanwender und Empfänger für die Daten
  \1 Das Endgerät des Endanwenders
    \2 Empfänger
\end{outline}

\section{Analyse des Problems}

Die größte Problematik für das Projekt ist die erzeugte Datenmenge pro System.
Beim senden der Daten vom Host zur Datenhaltung darf das vorhandene Netzwerk
nicht überlastet werden. Außerdem muss die Software, welche Metriken ausliest,
ressourcensparend sein. Sie darf laufende virtuelle Maschinen nicht
beeinflussen.

\section{Externe Einflüsse}

Die Software, sowie alle dazugehörigen Systeme sollen für einen permanenten
Betrieb ausgelegt sein. Eine Überwachung des Prozesses von IT-Mitarbeitern des
Cloud Providers wird benötigt, jedoch soll das Gesamtsystem auch unbeobachtet
zuverlässig die Aufgaben ausführen, sodass das manuelle Eingreifen nicht
erforderlich wird.

\section{Beurteilung der Machbarkeit des Auftrags}

\begin{enumerate}
  \item Organisatorisch / Betriebswirtschaftlich
        Die benötigte Zeit für das Gesamtprojekt wird mit dem Ticketsystem Jira
        und Methoden der agilen Entwicklung geplant. Die personellen Kosten
        fallen aufgrund der Durchführung über den Studiengang weg.

  \item Technisch
        Die benötigte Hardware, sowie Infrastruktur wird von dem Auftraggeber,
        sowie ggf. dazukommenden Partnern bereitgestellt. Dazu gehört ebenfalls
        eine Testumgebung, auf welcher die spätere Abnahme der Software
        durchgeführt werden könnte.
\end{enumerate}

\section{Gültigkeit des Pflichtenhefts}

Dieses Dokument ist ab der Zustimmung von Auftraggeber und Auftragnehmer
gültig. Änderungen am Inhalt können durchgeführt werden, wenn alle Seiten
zustimmen, dazu zählt der Auftraggeber, der Auftraggeber und der zuständige
Dozent des Heinrich-Hertz-Europakollegs. Ansprechpartner für Fragen ist der
Projektleiter Herr Tim Meusel.

\section{Projektorganisation}

Neben den bereits genannten Personen der Rolle Auftraggeber, sowie dem
begleitenden Dozent des Heinrich-Hertz-Europakollegs ist eine
Aufgabenverteilung innerhalb des Projektteams nötig. Es wurde entschieden, dass
jedes Projektmitglied alle Inhalte und Bereiche des Entwicklungsprozesses
kennenlernt und ggf. aushelfen oder komplett übernehmen kann. Jedoch besitzt
dabei jeder die Aufgabe, in seinem zugeordneten Bereich der Entwicklung voran
zu treiben und zu beobachten. Diese Aufteilung bildet sich wie folgt:
\begin{outline}
  \1 Der Projektteam-Leiter, Herr Tim Meusel, übernimmt die Beobachtung der
    \2 Entwicklung von der Datenbeschaffung auf den physischen Host Systemen,
    \2 sowie den Transport der Daten zwischen Datenhaltung und Datenbeschaffung.
  \1 Das Projektteam-Mitglied, Herr Nikolai Luis, übernimmt die Beobachtung der
    \2 Entwicklung der Datenhaltung,
    \2 sowie der Entwicklung der in der Datenhaltung integrierten API.
  \1 Das Projektteam-Mitglied, Herr Marcel Reuter, übernimmt die Beobachtung der
    \2 Entwicklung des Web Frontends, über welches später die Daten grafisch
    Dargestellt werden.
\end{outline}

\subsection{Phasenplan}

Dieses Projekt wird agil entwickelt und besitzt somit keine detaillierte
Zeitplanung von Projektphasen, diese wird zu jeder Arbeitseinheit von dem
Projektmanager geplant und festgelegt. Dennoch existierte eine grobe
Phasen-Agenda, an welcher sich der Projektmanager orientieren wird. Diese
teilt sich in folgende Phasen auf:
\begin{outline}
  \1 Analyse- / Initialisierungsphase
  \1 Konzeptionierung- / Designphase
  \1 Realisierungsphase
  \1 Testphase
  \1 Dokumentationsphase
  \1 Abnahmephase
\end{outline}

Meilensteine werden in Absprache zwischen dem Projektteam und dem Auftraggeber
oder dem Projektteam und dem Dozenten definiert.

\section{Ablaufkontrolle}

Für die Ablaufkontrolle des Projektes werden verschiedene Software-Werkzeuge,
sowie Prozess Vereinbarungen verwendet. Für die Übersicht und Zuteilung von
Aufgaben wird das Ticket-Management System ``Jira'' und für die Dokumentation
der getätigten Aufgaben ``Confluence'' genutzt. Quellcode wird mit dem
Versionsverwaltungssystem ``git'' verwaltet. Des Weiteren werden alle
Tätigkeiten und Beschlüsse wöchentlich dokumentiert.

\section{Risiken}
Wir planen echte Auslastungsdaten von Partnerfirmen zu bekommen und in unseren
Tests zu verwenden. Allerdings können diese Partner absagen. Für diesen Fall
müssen wir Daten selbst generieren. Dies kostet zusätzlich Zeit und unter
Umständen entsprechen die Daten nicht genug der Realität. Ein weiteres Risiko
wäre, dass einer unserer Projektmitglieder erkrankt oder über einen längeren
Zeitraum arbeitsunfähig ist.

%%% Local Variables:
%%% mode: latex
%%% TeX-master: "fsd-en"
%%% End:
