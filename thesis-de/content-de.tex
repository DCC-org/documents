\section{Erklärung}
Hiermit erklären wir, dass wir die Arbeit selbstständig verfasst und und keine
anderen als die angegebenen Quellen und Hilfsmittel benutzt haben. Diese Arbeit
wurde keinem anderen Prüfungsausschuss in gleicher oder vergleichbarer Form
vorgelegt.

\vfill
{\centering
\renewcommand{\arraystretch}{0.9}
\begin{tabular}{p{0.25\textwidth}p{0.05\textwidth}p{0.25\textwidth}p{0.05\textwidth}p{0.25\textwidth}}
  \dotfill                    & & \dotfill                      & & \dotfill \\
  \centering\footnotesize{Tim Meusel}& & \centering\footnotesize{Marcel Reuter}& & \centering\footnotesize{Nikolai Luis}%
\end{tabular}
}

\newpage

\section{Einführung}
Das Heinrich-Hertz-Europakolleg Bonn verlang im fünften und sechsten Semester
der Weiterbildung zum staatlich geprüften Informatiker eine fachbezogene
Projektarbeit. Dieses Projekt wird in Gruppen von zwei bis vier Personen
durchgeführt und soll fachliche Inhalte sowie Inhalte aus dem Projektmanagement
kombinieren. Es handelt sich um eine praktische Arbeit. Jeder Abschnitt enthält
ein Kürzel des Authors, hierbei bedeutet:
\begin{outline}
  \1 \verb+[+NL\verb+]+ erstellt von Nikolai Luis
  \1 \verb+[+MR\verb+]+ erstellt von Marcel Reuter
  \1 \verb+[+TM\verb+]+ erstellt von Tim Meusel
\end{outline}

\section{Projektvorstellung}
Cloud Provider bieten verschiedenste virtuelle Instanzen auf physischen Hosts
an. Dabei nutzt jeder virtuelle Server die vorhandenen Ressourcen des
physischen Systems unterschiedlich. Hier kommt es, aufgrund der
Mischkalkulation für die Ressourcen, zu einer Überbuchung (Overcommitment) des
Hosts. Weil das Monitoring nicht ausreichend ist, oder kein sinnvolles
Placement implementiert ist (Placement beschreibt den Algorithmus der einen
Node ermittelt auf dem eine neue virtuelle Instanz angelegt wird) kommt es
regelmäßig zu Performanceeinbußen. Für Kunden gibt es keine Transparenz über
die ihm zugeteilten und durch ihn genutzten Ressourcen, weshalb auch keine
ressourcenbasierte Abrechnung erfolgen kann. Teamleiter sind häufig mit der
Effizienzsteigerung der Plattform beschäftigt und müssen die Auslastung
steigern. Dies ist ohne detaillierte Auslastungsreports nicht möglich.

In diesem Projekt soll eine funktionierende Open Source Software entwickelt
werden, die sich in drei Teile gliedert:
\begin{outline}
  \1 Die verschiedenen Ressourcetypen (CPU Zeit / Daten Durchsatz / RAM
  Auslastung / Speicher Auslastung / Netzwerk Durchsatz) der einzelnen
  virtuellen Server müssen in einem sinnvollen Intervall periodisch ermittelt
  werden.
  \1 Die Daten müssen aggregiert und gespeichert werden. Hierbei ist auf eine
  Skalierung auf mindestens 10.000 virtuelle Instanzen unter Berücksichtigung
  der Verfügbarkeit und Performance der Datenbank zu achten (Sharding oder
  Replikation, verteilt oder zentral, dokumentenbasiert oder relational).
  \1 Diese Daten können dann dem Endanwender präsentiert werden (API und
  Web-UI). Hierzu wird eine Userstory Erhebung unter den drei Anwendertypen
  Kunde, Administrator, Manager bei Partnerunternehmen durchgeführt, um
  gewünschte Algorithmen zur Visualisierung zu ermitteln (zum Beispiel ermitteln von
  freien oder überbuchten Nodes, grafische Auswertung für Kunden).
\end{outline}

Dieses Projekt eignet sich besonders gut als Projektarbeit, da es in drei Teile
gegliedert ist. Jeder dieser Teile ist eigenständig und wird einem
Projektmitglied zugeordnet. Dies erleichtert die spätere Bewertung.

\subsection{Projektteam}
\subsection{Auftraggeber}
\subsection{Aktuelle Situation}
\subsection{Anforderungen}

\cite{lee2013introduction}
way more blabla

\section{Projektmanagement}
agil, sprints

\printbibliography
%%% Local Variables:
%%% mode: latex
%%% TeX-master: "thesis-de"
%%% End:
