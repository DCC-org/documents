\section{Erklärung}
Hiermit erklären wir, dass wir die Arbeit selbstständig verfasst und und keine
anderen als die angegebenen Quellen und Hilfsmittel benutzt haben. Diese Arbeit
wurde keinem anderen Prüfungsausschuss in gleicher oder vergleichbarer Form
vorgelegt.

\vfill
{\centering
\renewcommand{\arraystretch}{0.9}
\begin{tabular}{p{0.25\textwidth}p{0.05\textwidth}p{0.25\textwidth}p{0.05\textwidth}p{0.25\textwidth}}
  \dotfill                    & & \dotfill                      & & \dotfill \\
  \centering\footnotesize{Tim Meusel}& & \centering\footnotesize{Marcel Reuter}& & \centering\footnotesize{Nikolai Luis}%
\end{tabular}
}

\newpage

\section{Einführung}
Das Heinrich-Hertz-Europakolleg Bonn verlangt im fünften und sechsten Semester
der Weiterbildung zum staatlich geprüften Informatiker eine fachbezogene
Projektarbeit. Dieses Projekt wird in Gruppen von zwei bis vier Personen
durchgeführt und soll fachliche Inhalte sowie Inhalte aus dem Projektmanagement
kombinieren. Es handelt sich um eine praktische Arbeit. Jeder Abschnitt enthält
ein Kürzel des Autors, hierbei bedeutet:
\begin{outline}
  \1 {[MR]} erstellt von Marcel Reuter
  \1 {[NL]} erstellt von Nikolai Luis
  \1 {[TM]} erstellt von Tim Meusel
\end{outline}
\tm%

\section{Projektvorstellung}
\label{subsubsec:projektvorstellung}
Cloud Provider bieten verschiedenste virtuelle Instanzen auf physischen Hosts
an. Dabei nutzt jeder virtuelle Server die vorhandenen Ressourcen des
physischen Systems unterschiedlich. Hier kommt es, aufgrund der
Mischkalkulation für die Ressourcen, zu einer Überbuchung (Overcommitment) des
Hosts. Weil das Monitoring nicht ausreichend ist, oder kein sinnvolles
Placement implementiert ist (Placement beschreibt den Algorithmus der einen
Node ermittelt auf dem eine neue virtuelle Instanz angelegt wird) kommt es
regelmäßig zu Performanceeinbußen. Für Kunden gibt es keine Transparenz über
die ihm zugeteilten und durch ihn genutzten Ressourcen, weshalb auch keine
ressourcenbasierte Abrechnung erfolgen kann. Teamleiter sind häufig mit der
Effizienzsteigerung der Plattform beschäftigt und müssen die Auslastung
steigern. Dies ist ohne detaillierte Auslastungsreports nicht möglich.

In diesem Projekt soll eine funktionierende Open Source Software entwickelt
werden, die sich in drei Teile gliedert:
\begin{outline}
  \1 Die verschiedenen Ressourcetypen (CPU Zeit / Datendurchsatz / RAM
  Auslastung / Speicher Auslastung / Netzwerkdurchsatz) der einzelnen
  virtuellen Server müssen in einem sinnvollen Intervall periodisch ermittelt
  werden.
  \1 Die Daten müssen aggregiert und gespeichert werden. Hierbei ist auf eine
  Skalierung auf mindestens 10.000 virtuelle Instanzen unter Berücksichtigung
  der Verfügbarkeit und Performance der Datenbank zu achten (Sharding oder
  Replikation, verteilt oder zentral, dokumentenbasiert oder relational).
  \1 Diese Daten können dann dem Endanwender präsentiert werden (\gls{API} und
  Web-UI). Hierzu wird eine Userstory Erhebung unter den drei Anwendertypen
  Kunde, Administrator, Manager bei Partnerunternehmen durchgeführt, um
  gewünschte Algorithmen zur Visualisierung zu ermitteln (zum Beispiel
  ermitteln von freien oder überbuchten Nodes, grafische Auswertung für Kunden).
\end{outline}

Dieses Projekt eignet sich besonders gut als Projektarbeit, da es in drei Teile
gegliedert ist. Jeder dieser Teile ist eigenständig und wird einem
Projektmitglied zugeordnet. Dies erleichtert die spätere Bewertung.
\tm%

\subsection{Projektteam}
Das Projektteam besteht aus den drei Mitgliedern Marcel Reuter, Nikolai Luis
und Tim Meusel.
\tm%

\subsubsection{Marcel Reuter}
Herr Reuter beendete 2013 seine Ausbildung zum IT-Systemelektroniker und
arbeitet seit dem bei der Firma EBF-EDV Beratung Föllmer GmbH. Er ist
verantwortlich für die visuelle Schnittstelle des Projekts (Punkt 3).
\mr%

\subsubsection{Nikolai Luis}
Herr Luis begann die Weiterbildung zum Techniker wärend seiner Ausbildung zum
Fachinformatiker Anwendungsentwicklung, welche er im Januar 2016 beendete. Er
arbeitet als BIG Data Analyst bei der Deutschen Telekom. Er ist verantwortlich
für die Speicherung der Daten und die automatisierte Schnittstelle (Punkt 2).
\nl%

\subsubsection{Tim Meusel}
Herr Meusel schloss seine Ausbildung zum Fachinformatiker Systemintegration
2012 ab. Aktuell arbeitet er als Systems Engineer bei der Host Europe Group.
Er verantwortet die Ermittlung sowie Übertragung der Daten.
\tm%

\subsection{Auftraggeber}
Die Ansprechpartner für dieses Projekt ist das Unternehmen Puppet Inc. (im
folgenden Puppet) in der Rolle als Auftraggeber, welches von den Mitarbeitern
Herrn David Schmitt und Herrn Steve Quin vertreten wird. Puppet ist Marktführer
im Bereich Konfigurationsmangement Software. Software dieser Art hilft es
Administratoren sehr einfach Testumgebungen aufzubauen und auch
Produktivumgebungen zu Verwalten. Das Kernprodukt der Firma Puppet, welches
ebenfalls puppet heißt, steht unter einer Open Source Lizenz und darf frei
genutzt werden. Der Einsatz der Software erlaubt es dem Projektteam
verschiedenste Prototypen in kurzer Zeit zu bauen. Puppet entwickelt außerdem
Software zum Testen. Dies vereinfacht das Qualitätsmanagement im Projekt.
Puppet hat in der Vergangenheit bewiesen, mit agiler Entwicklung und diversen
Testverfahren umgehen zu können, beides wird intensiv in deren Teams genutzt.
Herr Quin und Herr Schmitt bieten tiefgreifendes Wissen zu den verschiedenen
Programmen und Techniken, um das Projektteam zu untersützen und zu beraten.
\tm%

\subsection{Aktuelle Situation}
In der~\ref{subsubsec:projektvorstellung} wurden bereits die Intentionen des
Projekts erläutert. In der Vergangenheit wurde schon mal versucht eine passende
Softwarelösung zu entwickeln. OpenStack ist ein Softwareprojekt mit dem große
Mengen an Rechen- und Netzwerkkapazitäten, sowie persistenter Speicher in einem
Rechenzentrum zu einer \gls{Public Cloud} oder \gls{Private Cloud} Umgebung
zusammengefasst und orchestiert werden~\cite{OpenStack_Intro}. Eines der
Teilprojekte ist Telemetry. Das Ziel hiervon ist es, zuverlässig Daten von
physischen und virtuellen Ressourcen zu ermitteln verlässlich zu speichern. Die
Daten sollen zur Analyse genutzt werden und Aktionen auslösen wenn bestimmte
Kriterien erreicht sind~\cite{OpenStack_Telemetry}. Telemetry bringt leider
mehrere Nachteile mit sich:

\begin{outline}
  \1 Die Standarddatenbank für Telemetry war lange Zeit MongoDB. MongoDB ist
  eine dokumentenorientierte Datenbank, hier werden Daten nicht in Tabellen
  strukturiert sondern in eigenständigen Dokumenten. Jedes Dokument hat eine
  eigene Datenstruktur (z.B. im \gls{JSON}
  Format)~\cite{Dokumentenorientierte_Datenbank}. Dies ermöglicht es, die
  Datenbank sehr einfach vertikal zu skalieren. Hierbei werden alle Dokumente
  auf mehrere Server verteilt. Schreib- und Lese-Anfragen können ebenfalls auf
  alle Server verteilt werden. Die Skalierung ist nahezu
  linear~\cite{MongoDB_Architecture},~\cite{What_is_MongoDB}. Die Grundidee ist
  sehr gut und eignet sich für besonders große Datenmengen oder Installationen
  die Hochverfügbar sein müssen. Die Implementierung in MongoDB hat allerdings
  diverse Nachteile. Kyle Kingsbury hat intensiv MongoDB mit \gls{Jepsen}
  getestet. Die Datenbank hat sich mehrfach als sehr unzuverlässig
  herausgestellt~\cite{MongoDB_on_Jepsen}. Die beiden folgenden Punkte
  disqualifizieren MongoDB für einen Einsatz als persistenten  Datenspeicher,
  da die Persistenz nicht sichergestellt ist. Es wurde von Kyle Kingsbury
  bewiesen, das:
    \2 MongoDB bei einem Schreibvorgang bestätigt, das Daten persistent
    gespeichert wurden, dies allerdings nicht immer der Fall ist. Ein
    unbemerkter Datenverlust ist die Folge.
    \2 MongoDB in bestimmten Situationen die falschen Daten bei einer
    Leseanfrage zurückliefert.

  \1 Das Telemetry Projekt hat diese Probleme ebenfalls erkannt und nach
  alternativen Datenbanken gesucht. Nachdem keine vorhandene Lösung ihren
  Anforderungen entsprach, entschieden sie sich für die Entwicklung einer
  eigenen Datenbank: \gls{Gnocchi}. Diese Datenbank ist noch in der Entwicklung
  und noch nicht für den produktiven Einsatz bereit.
  \1 Es wird immer wieder von Skalierungsproblemen berichtet. Das CERN betreibt
  eine OpenStack Installation und konnte die Probleme mit Telemetry nur mit
  immenser Hardware lösen. Die Kosten für diese Infrastruktur als auch die
  Komplexität sind viel zu hoch. Sie übersteigen das Budget der meisten Cloud
  Umgebungen wodurch diese unwirtschaftlich werden~\cite{OpenStack_CERN}.
  \1 Das Telemetry Projekt ist sehr stark in OpenStack eingebunden. Der Einsatz
  in anderen Cloud Umgebungen ist nicht vorgesehen, da die einzelnen Dienste
  aus dem Telemetry Projekt weitere OpenStack Komponenten benötigen. Ein
  eigenständiger betrieb ohne OpenStack ist nicht geplant für die Zukunft.
\end{outline}

Aufgrund der langen Liste an Problemen ist Telemetry aktuell innerhalb einer
kleinen OpenStack Installation teilweise nutzbar, jedoch nicht wirtschaftlich
in großen Umgebungen oder außerhalb von Openstack. Es ist nicht davon
auszugehen, dass diese Probleme kurz- bis mittelfristig gelöst werden. Somit
entschied sich das Projektteam eine Alternative zu entwicklen die unkompliziert
zu installieren ist und unabhängig von OpenStack arbeitet.
\tm%

\subsection{Anforderungen}
Die Detailanforderungen werden getrennt für die drei Bereiche des Projekts im
folgenden Abschnitt beschrieben. Diese Ergebene sich aus dem Lastenheft und
Pflichtenheft sowie aus den User Stories der Partner. Für Alle Lösungen gillt,
das sie eine aktive Community haben und unter einer Open Source Lizenz stehen.
\tm%

\subsubsection{Datenerfassungssysteme}
Unterschiedliche Ausbaustufen für virtuelle Machinen werden über mehrere
verschiedene Ressourcen, die sich in Ihrer Leistungsklasse unterscheiden,
differenziert. Diese Typen müssen für jede virtuelle Machine ausgelesen werden.
Als Referenz werden die Typen der größten deutschen Anbieter nachweis! für
virtuelle Machinen genommen. Diese sind aktuell:

Zugewiesener Arbeitsspeicher, Anzahl/Leistung der Prozessorkerne, Durchsatz des
Datenträgers, Anbindung an das Internet, nachweis via scrots fachwörter
referenz?

Diese Werte müssen sowohl in den virtuellen Machinen als auch auf dem Host
ermittelt werden können. Im Bereich Managed Hosting hat der Hoster Zugriff in
die virtuellen Instanzen und kann dort direkt sehr detailliert Daten ermitteln.
Im klassischen vServer Bereich (dem Bereitstellen des vServers als Produkt,
nicht als Service) hat der Betreiber keinen Zugriff in die VM und muss Daten
vom Host aus ermitteln. Auf dem Host muss zusätzlich Gesamtauslastung der
Ressourcen sowie der Zustand der Hardware ermittelt werden.

Gesammelte Daten müssen lokal zwischengespeichert werden können, um einem
Verlust bei Netzwerkstörungen zu vermeiden. Der Versand muss nach dem Push- und
nicht nach dem Polling-Verfahren arbeiten. Somit kann Eventbasiert gearbeitet
werden, dies verhindert Overhead im Netzwerk. Eine Visualisierung in Echtzeit
ist nicht gefodert weshalb Daten lokal zwischengespeichert werden können um sie
gebündelt zu verschicken.
\tm%

\subsubsection{Datenbanksysteme}

\subsubsection{Frontends}
Die Weboberfläche muss ausgewählten Benutzern eine Übersicht über die
verbrauchten oder freien Ressourcen, wie z.\,B.\ CPU Auslastung oder RAM
Auslastung einzelner virtueller oder physischen Maschinen bereitstellen.
Hierbei ist wichtig, dass die Sicherheit (Authentisierung, Authentifizierung
und Authorisierung) der Daten gewährleistet wird. Dies bedeutet, das sowohl die
Verbindung zwischen Datenbank und Weboberfläche gesichert sein muss, als auch
die Weboberfläche selber, gegen Zugriff von Unbekannten. Die Weboberfläche
greift hier mittels API Abfragen auf die Datenbank zu, dies ermöglicht uns den
Zugriff nur für das Frontend auf die Datenbank gewährleisten zu können.  Die
Weboberfläche soll Administratoren bei Neukonfiguration von virtuellen
Maschinen unterstützen und zeigt hier die Auslastung der einzelnen physischen
Systemen. Der Administrator kann so gezielt die einzelnen physischen Maschinen
besser auslasten und verteilen. Ebenfalls kann das Frontend eine Schnittstelle
für komplexe Abfragen und Analysen bereitstellen. Bei Bedarf können diese auch
visuell ausgegeben werden.
\mr%

\section{Projektmanagement}
Die Liste an verfügbaren Managementmethoden ist lang. Methoden wie PRINCE2 oder
Lean Management kommen aus dem Bereich des Projektmanagements und sind seit
vielen Jahren auch im Bereich der Softwareentwicklung vertreten. Hinzu kommen
Vorgehensmodelle aus der Softwareentwicklung selbst, wie das Wasserfallmodell
oder das Spiralmodell. In den letzten Jahren gab es einen Wandel hin zu agiler
Entwicklung. Es hat sich gezeigt, dass sich in der schnelllebigen
Informationswelt Anforderungen an Software regelmäßig ändern, auch während der
Entwicklungs- und Planungsphase und nicht erst im späteren Betrieb. Außerdem
gibt es in den meisten Projekten unvorhergesehene Zwischenfälle. Dazu gehören
unter anderem:

\begin{outline}
  \1 Sicherheitslücken in verwendeten Bibliotheken werden entdeckt. Diese
  müssen oftmals aufwendig aktualisiert werden.
  \1 Die Zeiteinschätzung für die Implementierung von Funktionen oder die
  Behebung von Fehlern benötigt wesentlich mehr oder weniger Zeit als
  geschätzt.
\end{outline}

Die Techniken und Vorgehensweisen der agilen Softwareentwicklung versuchen dem
entgegenzubeugen. Sie zeichnen sich dadurch aus, dass sie:

\begin{outline}
  \1 Iterativ arbeiten
  \1 Einen sehr geringen Bürokratie Mehraufwand besitzen
  \1 Sich flexibel an das Projekt und an Änderungen anpassen
\end{outline}

Die Firma Puppet entwickelt seit mehreren Jahren erfolgreich. Sie steht dem
Projektteam mit hilfreichen Tips und Schulungen zur Seite. ``Agile
Softwareentwicklung'' gilt als Oberbegriff für alle Techniken und
Vorgehensweisen in dem Bereich. Das Softwareentwicklungsmodell Scrum nutzt
Teile der agilen Softwareentwicklung. Das Projektteam entschied sich für die
Nutzung agiler Projektmanagement Methoden da diese in der Wirtschaft aktuell am
verbreitetsten sind.
\tm%

\subsection{Scrum}
Scrum ist zu unflexibel für ein Technikerprojekt. Die Methode schreibt vor,
dass man alle Techniken von Scrum nutzen muss und diese nicht abgewandelt
werden dürfen.  Unter anderem wird ein ``Daily Standup'' verlangt. Dies ist
eine Besprechung an jedem Arbeitstag, welches genau 15 minuten lang ist. Die
Zeit muss gleichmäßig auf alle Teilnehmer verteilt werden. Aufgrund der
Vollzeitbeschäftigung aller Mitglieder in Kombination mit dem
Abendschulunterricht erfolgen viele Arbeiten am Projekt asynchron. Die
Durchführung von täglichen Besprechungen ist nicht
praktikabel.~\cite{scrum_talk}
\tm%

\subsection{Agile Vorgehensweise im Projekt}
erklären wieso weshalb warum sprints, milestones, userstories epics
\tm%

\section{Verwendete Tools}
ist das relevant? als eigener punkt?

\section{Analyse von Softwarekomponenten}
Zusammen mit dem Auftraggeber folgende Liste and Komponenten erstellt, die
vergleichen und gegebenfalls evaluiert werden sollen:

\subsection{Datenerfassungssysteme}

\begin{outline}
  \1 collectd mit Virt Plugin
  \1 coreutils
  \1 zabbix-agent
  \1 python-diamond
  \1 sysstat
  \1 atop
  \1 logtash
  \1 riemann
\end{outline}

\subsection{Datenbanksysteme}
\label{subsec:datenbanksysteme}
Zu Beginn des Projektes musste das Projektteam mögliche Datenhaltungssysteme in
einer Liste für eine darauffolgende Evaluierung definieren. Bei der Erstellung
dieser Liste wurden Systeme aufgenommen, welche den Projektmitgliedern aus
vergangenen Projekten, beziehungsweise aus dem täglichen Arbeitstag bereits
bekannt waren. Zusätzlich hatten die Projektmitglieder vorab mit Experten aus
dem eigenen Unternehmen und aus dem Unternehmen des Auftraggebers mögliche
weitere Systeme gesprochen.  Dadurch das Herr Luis in dem Bereich der
Massendatenhaltungssysteme arbeitet, sind ebenfalls Systeme aufgenommen worden,
welche im Bereich der sogenannten Big Data Technologie arbeiten.

Das hinzufügen von weiteren/neuen Datenbanksystemen zur späteren
Evaluierung erfolgt nur nach einer Genehmigung vom Auftraggeber.

Die in der Liste aufgenommen Systeme sind folgende:
\begin{outline}
  \1 elasticsearch
  \1 cassandra
  \1 postgres
  \1 OpenTSDB
  \1 KNIME
  \1 impala
  \1 hadoop
  \1 hive
\end{outline}
\nl%

\subsubsection{Vorbereitung der Evaluierung}
\label{subsubsec:DBS_vorbereitung_der_evaluierung}
Als Vorbereitung für die Evalution der Datenhaltungssysteme wurden mehrere
Kriterien definiert, welche einen Leitpfaden für die spätere Arbeit geben.
Während der Auswahl und Definition der von dem Datenbanksystem zu erfüllenden
Kriterien wurde darauf geachtet, dass diese sich aus dem späteren Zielsystem
ableiten lassen und ebenfalls realistisch erfüllbar sind. Anschließend wurde
die Relevanz jedes Kirteriums von den Teammitgliedern beurteilt und festgelegt.

Die daraus resultierten Kriterien sind nun nach Relevanz absteigend
aufgelistet:
\begin{outline}
  \1 Bereits vorhandenes Wissen über die Datenbank. Dazu gehört, dass diese
  eine umfangreiche Dokumentation besitzt und eine aktive und große Community
  für Diskussionen und Fragen vorhanden ist. Ebenfalls fallen die vorhandenen
  Vorkenntnisse über das Datenbanksystem der Projektmitglieder, sowie die
  Mitarbeiter von dem Unternehmen des Auftraggebers unter diesem Kriterium.  Es
  ist wichtig, dass die zu aufwendene Einarbeitungszeit während dem Projekt und
  im späteren Wirkbetrieb bei Mitarbeitern des Unternehmens so gering wie
  möglich gehalten wird.
  \1 Das Datenhaltungssystem muss einfach, schnell und jederzeit erweiterbar
  sein. Dies bedeutet, dass Ressourcen (CPU/RAM/Speicherplatz) des
  Datenbanksystems im optimalsten Fall während dem normalen Betrieb (kein
  Neustart oder Auszeit) aufgestockt werden kann. Es ist jedoch nach
  Anforderung des Auftraggebers ausreichend, wenn eine Aufstockung nach einer
  Auszeit von maximal 45 Minuten erfolgt ist.
  \1 Umweltschonende und effiziente Datenhaltung ist zu beachten. Dies
  definiert eine optimale Relation zwischen der verwendeten Hardware und dem
  daraus resultierendem Ergebnis. Eine umweltschonende Datenhaltung kann
  unabhängig der korrekten Auswahl der genutzten Stromeffiziensklasse jedes
  Hardwarebauteils bereits bei der im Wirkbetrieb zu benutzende Software
  beeinflusst werden.  Dies ist im Bereich der Datenhaltungs-Software die
  entstehende Speicherplatzgröße für einen einzelnen Datensatz in einer
  Datenbank. Um so kleiner dieser ist, desto mehr kann bei Energiekosten für
  zusätzlichen Speicherplatz eingesparrt werden.

  Unter Berücksichtigung der genannten Attribute, soll jedoch die
  Datenhaltungs-Software mit der gleichen Hardware-Ressource das beste und
  schnellste Ergebnis erbringen können. Dies bedeutet, dass auch bei komplexen
  Analysetechniken und zusätzlich hohen Datenmengen, dass System weiterhin
  gegenüber anderen Datenhaltungssystemen ein valides und schnelles
  Analyseergebnis erbringen kann.
  \1 Die Datenhaltungs-Software sollte bereits Methoden zur Datensicherung
  besitzen. Dabei ist zum einem eine Backup Methode, und zum anderen das
  Verhalten bei Problemen am System, zum Beispiel ein Defekt am Netzwerkkabel,
  mit inbegriffen.
\end{outline}
\nl%

\subsubsection{Durchführung der Evaluierung}
\label{subsubsec:durchfuehrung_der_evaluierung}
Bei der Durchführung der Evalution von den ausgewählten Datenhaltungssystemen
aus der Liste in Punkt~\ref{subsec:datenbanksysteme} wurden die
Evalutionskriterien aus Punkt~\ref{subsubsec:DBS_vorbereitung_der_evaluierung}
verwendet. Diese dienten als Leitpfaden für jedes Datenhaltungssystem und
wurden von jedem Projektmitglied beachtet.

Zu Beginn der Durchführung teilte Herr Meusel als Projektleiter die zu
evaluierenden Datenhaltungssysteme jedem Projektmitglied zu, um eine schnellere
Bearbeitung zu gewährleisten. Bei der Einteilung wurden die bereits vorhandenen
Erfahrungen jeder Projektmitglieder zu dem jeweiligen Datenhaltungssystem
berücksichtigt. Im zweiten Schritt sollte sich anschließend jedes
Projektmitglied mit dem zugeteilten Dantenhaltungssystem kurz
auseinandersetzen. In diesem Schritt ist Herrn Luis aufgefallen, dass es sich
bei der in der Liste aufgenommenen Software ``KNIME'' nicht um ein
Datenhaltungssystem handelt, sondern um ein Werkzeug zur Datenanalyse,
Datenaufbereitung und Datendarstellung.

Die daraus resultierenden Ergebnisse wurden Herrn Luis anschließend von den
Projektmitgliedern bereitgestellt, sodass dieser einen Überblick im
Projektbereich ``Datenhaltung'' erschaffen konnte. Die resultierenden
Ergebnisse und Erkenntnisse sind in dem nachfolgenden Punkt für jede
Datenhaltungs-Software dokumentiert.
\nl%

\subsubsubsection{Elasticsearch}
\label{subsubsubsec:elasticsearch}
\subsubsubsubsection{Beschreibung}
\label{subsubsubsubsec:elasticsearch_beschreibung}
Some Text here!
\nl%
\subsubsubsubsection{Evaluierungsbericht}
\label{subsubsubsubsec:elasticsearch_evaluierung}
Some Text Here!
\nl%

\subsubsubsection{Cassandra}
\label{subsubsubsec:cassandra}
\subsubsubsubsection{Beschreibung}
\label{subsubsubsubsec:cassandra_beschreibung}
Some Text here!
\nl%
\subsubsubsubsection{Evaluierungsbericht}
\label{subsubsubsubsec:cassandra_evaluierung}
Some Text Here!
\nl%

\subsubsubsection{Postgres}
\label{subsubsubsec:postgres}
\subsubsubsubsection{Beschreibung}
\label{subsubsubsubsec:postgres_beschreibung}
Some Text here!
\nl%
\subsubsubsubsection{Evaluierungsbericht}
\label{subsubsubsubsec:postgres_evaluierung}
Some Text Here!
\nl%

\subsubsubsection{OpenTSDB}
\label{subsubsubsec:opentsdb}
\subsubsubsubsection{Beschreibung}
\label{subsubsubsubsec:opentsdb_beschreibung}
Some Text here!
\nl%
\subsubsubsubsection{Evaluierungsbericht}
\label{subsubsubsubsec:opentsdb_evaluierung}
Some Text Here!
\nl%

\subsubsubsection{Hadoop / Hive /Impala}
\label{subsubsubsec:hadoop_hive_impala}
\subsubsubsubsection{Beschreibung Hadoop}
\label{subsubsubsubsec:hadoop_beschreibung}
Das Apache Hadoop System wird zum aktuellen Zeitpunkt bei
großen und leistungsfähigen Datenhaltungssystemen eingesetzt, sodass es
auch oftmals in Verbindung mit den Worten ``BIG DATA'' genannt wird.

Es handelt sich dabei um eine Vielzahl von Komponenten, bestehend aus
mehreren Frameworks und Ökosystemen, also im Klartext eine Reihe von frei
wählbaren Bibliotheken zur Definition der Datenhaltungslandschaft. Apache
Hadoop verfolgt dabei den Ansatz, alle Dateninformationen in nur einem
gebündelten System zu verwalten und keine zusätzlichen Datenhaltungssysteme,
welche gegebenfalls nur auf eine einzelne Anforderung spezialisiert sind,
zu gründen.

Um diesen Ansatz zu erfüllen, wurde das Clusterkonzept eingesetzt, Hadoop
nennt es selber ``Hadoop Distributed File System'' (HDFS).
Dabei werden die Daten in der Datenhaltung nicht mehr auf einem einzelnen
physischen System gehalten, sowie vor einem Hardwareausfall gesichert,
sondern nun auf mehrere Systeme verteilt. Hadoop teilt dabei jedem in dem
Cluster hinzugefügten System eine spezielle Aufgabe zu. Für die
Datenspeicherung existieren dabei drei Aufgabenfelder. Nummer Eins
ist der Data Node, dieser besteht oftmals aus mehreren Festplatten und einer
Netzwerkanbindung. Auf diesem wird eine Datensatz mehrmals auf verschiedenen
Festplatten abgespeichert, sodass bei einem Ausfall der Datensatz gesichert
ist. Es können bei Hadoop beliebig viele Data Node's in der
Datenhalungslandschaft eingesetzt werden, so mehr Data Node's, umso mehr
Speicherplatz steht zur verfügung. Nummer Zwei ist der Name Node, bei diesem
handelt es sich um ein einzelnes System, welches den Speicherort (Data Node
ID), von einem Datensatz bereitstellt. Die letzte Aufgabe erfüllt der Backup
Node, auch bei diesem handelt es sich um ein einzelnes System. Das System
kommt zum Einsatz, wenn der Name Node ausfallen sollte. Er synchronisiert sich
dauerhaft mit dem Name Node, um jederzeit zum Einsatz zu kommen.

Für das managen des Clusterkonzepts und zeitbasierenden Aufgaben (im
englischen: Job scheduling) wurde ``Hadoop YARN'' gegründet. Dieses
gibt den Nodes vor, welche Daten mit welchem Node ausgetauscht werden
und wo die einzelnen Backups stattfinden sollen. Es definiert dabei
auch zu welchem Tageszeitpunkt einzelne Aufgaben erfüllt werden sollen.
Diese Definitionen können von dem System automatisch definiert werden,
jedoch auch von einer Person manuell eingestellt und verwaltet werden. 

Für die Anbindung späterer Systeme und Methodiken der Datenanalyse, wird 
``Hadoop Common'' eingesetzt. Es handelt sich dabei um ein reines
Schnitstellen-Modul, welches standartisierte Protokolle implementiert und
betreibt. Es ermöglicht zum Beispiel die Anbindung der
Software ``Hive'' und ``Impala'', welche in den Punkt
~\ref{subsubsubsubsec:hadoop_beschreibung} und 
~\ref{subsubsubsubsec:impala_beschreibung} erläutert werden.

Für die Bearbeitung und Analyse von zeitkritischen Daten kommt die
Komponente ``Hadoop Map Reduce'' zum Einsatz. Diese ist von dem gleichnamigen
Programmiermodell abgeleitet. Es handelt sich dabei um
eine Hadoop-YARN basiertes System, welches ein parallel abarbeiten von 
Aufgaben ermöglicht. Dabei können Analyseanfragen nun nicht mehr nur in
einzelne Server-Threads aufgeteilt werden, sondern auch auf mehrere Data
Nodes. Dies ermöglicht ein schnelles und effizientes Management von
Datenabfragen. Gleichzeitig besitzt diese Komponente einen Data Node, welcher
aus reinem Arbeitsspeicher besteht. Arbeitsspeicher ist bis dato einer der
schnellsten Speichermedien und kommt bei zeitkritischen Datenabfragen zum
Einsatz. 

Diese vier Komonenten bilden den ``Hadoop Core'', welcher zwingend für den
Betrieb eines Hadoop Systems benötigt wird. Hadoop basierende Systeme, zur
Erfüllung von verschiedene und spezialisierten Aufgaben, werden unter dem
``Hadoop Ecosystem'' zusammengefasst. Eine vollständige Liste von Haddop
basierenden Systemen und Projekten kann hier entnommen
werden~\cite{Hadoop_related_projects}.
\nl%

\subsubsubsubsection{Beschreibung Hive}
\label{subsubsubsubsec:hadoop_beschreibung}
Stellt Schemas bereit (DWH) - Some more Text here
\nl%

\subsubsubsubsection{Beschreibung Impala}
\label{subsubsubsubsec:impala_beschreibung}
Realtime SQL working with Hadoop Map Reduce - Some more Text here
\nl%

\subsubsubsubsection{Evaluierungsbericht}
\label{subsubsubsubsec:hadoop_hive_impala_evaluierung}
Das ganze ist kacke weil zu viel - Some more Text here
\nl%


\subsection{Schnittstelle - Datenbereitstellung}
\label{subsec:schnittstelle_datenbereitstellung}
Neben der herkömmlichen Methode zur Datenabfrage von Datenbank-Systemen über
einen Datenbanktreiber, existiert noch die Möglichkeit einer API.

Die Verwendung von einer API Schnittstelle besitzt Nachteile, sowie Vorteile
beziehungsweise verbirgt gewisse Gefahren in Hinsicht auf den späteren Einsatz
im Produktivsystem. Eine dieser Gefahren ist das vernachlässigen der
Dokumentation einer API. Dadurch, das während der Entwicklung einer API bereits
definiert wird, welche Fähigkeiten und Methodiken die API besitzen soll, ist
eine nicht vorhandene oder unvollständige Dokumentation suboptimal. Darum muss
bei der Auswahl des genutzen API Systems darauf geachtet werden, dass die
Dokumentation von Softwarekomponenten für den Entwickler so einfach wie möglich
erfolgt.

Wenn die Dokumentation einer API vollständig und das ausführende System den
Performance-Anforderungen gerecht ist, bringt die Nutzung von einem API System
Vorteile. Darunter ist eine bessere Verwaltung von Zugriffsrechten und
somit auch gleichzeitig eine Verbesserung der Systemsicherheit. Durch den vorab
festgelegten Funktionsumfang der API kann der Systeminhaber geziehlt
kontrollieren, welche Inhalte ein Benutzer sehen oder modifizieren kann. Dabei
können grobe Einschränkungen, wie das Lesen oder Schreiben auf der gesamten
Datenbank, oder auch detaillierte Einschränkung, wie das Lesen oder Schreiben
von einer einzelnen Tabelle in der Datenbank, kontrolliert werden. Anders wie
bei der Verbindung über einen Datenbanktreiber kann bei der API der potentielle
Angreifer keine eigenen Datenbankbefehle definieren und ausführen. Somit wird
das Risiko der Datenmanipulation durch eine zusätzliche Schicht(Layer)
veringert.

Oftmals werden Daten nicht in der Struktur benötigt, in der sie gespeichert
sind. Die Möglichkeiten einer Transformation auf Datenbankebene sind sehr
begrenzt, eine API Schnittstelle in einer Hochsprache bietet hier deutlich mehr
Möglichkeiten.
\nl%

\subsubsection{Vorbereitung der Evaluierung}
\label{subsubsec:api_vorbereitung_der_evaluierung}
Anders wie bei der Evalierung der Datenbanksysteme wurde bei der Evaluierung
der möglichen API Systeme keine Liste im voraus erstellt. Dies lag daran, dass
keiner der Projektmitglieder oder Interessenten an dem Projekt Erfahrungen mit
API Systemen im vorhinein besitzen.

Aufgrund dessen wurde entschieden, eine Liste von Anforderungen an das
API System zu erstellen und im Anschluss während der Evaluierungsphase
Softwarelösungen zu suchen, welche die Anforderungen am besten erfüllen kann.

Die festgehaltenen Anforderungen sind nun nach Relevanz absteigend aufgelistet:
\begin{outline}
  \1 Die API Software muss auf einer selbst administrierten Hardware-Umgebung
  betrieben werden können. Das verwenden von einem fertigen API Softwarepaket
  von einem Cloudanbieter, bei welchem der Administrator alle Konfigurationen
  über ein Web-Interface des Providers vornimmt und der Cloudanbieter die
  vollständige Hardware-Umgebung administriert, ist ein absolutes
  Ausschlusskriterium. Es ist vorgehsehen, dass das API System möglichst nah an
  dem Datenhaltungssystem liegt, um mögliche Latenzen zwischen den beiden
  Systemen so niedrig wie möglich zu halten. Zudem sollen externe Cloudanbieter
  keinen Zugriff auf die teilweise Kundenbezogenen Daten der Datenhaltung
  erhalten. Zudem kann während dem Projekt entschiedenen werden, dass das
  Datenhaltungssystem nicht von externen Systemen erreichbar sein soll, sodass
  die damit verbundene Lösung über einen Cloudprovider nicht möglich wäre.
  Externe Systeme sind in diesem Projekt Geräte, welche außerhalb des
  Unternehmens betrieben werden.
  \1 Die API Software muss vollständig konfigurierbar und administrierbar sein.
  Entsprechend soll die Software nur eine leere Hülle mit Standardfunktionen
  einer API zur Verfügung stellen. Eine Konfiguration von einzelnen Parametern
  ist dabei erwünscht. Zu diesem Parametern gehört als Beispiel der Port, auf
  welchem das API System betrieben und von anderen Systemen erreicht werden
  kann. Zur Hilfe der admnistration soll das API System ein Framework für
  Entwickler bereitstellen. Mit diesem können wir als Projektmitglieder
  anschließend die Logik in das System implementieren und somit definieren
  welche Funktionen die API ausüben darf. Ein Framework in einer bekannten
  Entwicklersprache ist dabei erwünscht. Dies hat den Vorteil, dass der
  zuständige Entwickler im Projekt keine weitere Einarbeitungszeit für die neue
  Entwicklersprache benötigt.
  \1 Das API System sollte innerhalb des Projektes keine Kosten verursachen.
  Dazu gehören Lizenzkosten sowie laufende Betriebskosten bei externen
  Cloudanbietern. Spätere Kosten bei dem Auftraggeber sind in dieser
  Anforderung nicht mit inbegriffen. Grundsätzlich gillt für das ganze Projekt,
  dass nur Software mit einer Open Source Lizenz genutzt werden darf.
  \1 Die Erstellung von späteren Anwendungshandbüchern der API, soll dem API
  Entwickler so einfach wie möglich fallen. Eine sehr beliebte
  Dokumentationsmöglichkeit ist dabei das dokumentieren von wichtigen
  Informationen direkt in dem Quellcode des API Systems. Entwickler können
  dabei über die Kommentierungs-Funktion der jeweiligen Entwicklersprache eine
  Beschreibung sowie Ein- und Ausgabe von Werten festlegen. Diese werden im
  Anschluss während dem Kompiliervorgang in einem struktierten Dokument
  festgehalten und neben der eigentlichen API Software mit ausgeliefert. Dieses
  seperate Dokument ist oftmals in Form einer HTML oder PDF und kann auf
  gängigen Computersystemen eingesehen werden. Die beschriebene oder ähnliche
  Dokumentationsmöglichkeit sollte die API Software zur Verfügung stellen.
  Alternativ kann auch erst die Dokumentation geschrieben werden und darauf
  basierend Quellcode generiert werden. Beide Verfahren nennt man ``One Source
  of Truth Prinzip''. Hierbei wird sichergestellt, das es nur eine autoritative
  Quelle gibt, und der zugehörige Pendant auf der Quelle beruht. Dies vermeidet
  Dokumentation die nicht mit den eigentlichen Funktionen der Software
  übereinstimmen.
  \1 Bereits vorhandenes Wissen über das API System. Dazu gehört, dass diese
  eine umfangreiche Dokumentation besitzt und eine aktive und große Community
  für Diskussionen und Fragen vorhanden ist. Die dazugehörige
  Programmiersprache der API sollte ebenfalls für die Projektmitglieder
  nachvollziehbar sein, sowie eine umfangreiche Dokumentation und aktive
  Community für Diskussionen und Fragen besitzen.
\end{outline}
\nl%

\subsubsection{Durchführung der Evaluierung}
\label{subsubsec:api_durchfuehrung_der_evaluierung}
Bei der Evaluierung der API Systeme wurden die in Punkt
~\ref{subsubsec:api_vorbereitung_der_evaluierung} festgehaltenen Anforderungen
der API als Leitlinie für die Durchführung verwendet. Als ersten Schritt wurde
zunächst Recherche betrieben, damit das Projektteam einen allgemeinen Eindruck
über die aktuelle Marktsitutation zu API Systemen erhält. Dabei wurden im
Internet mehrere Blogs, Foren sowie Artikel von Fachzeitschriften gelesen.  Die
dort angesprochenen API Systeme wurden anschließend mit den bereits definierten
Anforderungen abgeglichen und aussortiert. Dabei stellten die Projektmitglieder
fest, dass die meisten API Systeme von Cloudprovidern nur in bereits
vorkonfigurierten Paketen mit einem monatlichen Festpreis angeboten werden.
Beides ist ein absolutes Ausschluskriterium für den späteren Wirkbetrieb sowie
Nutzung der Software. Ebenfalls wurde bei den vorkonfigurierten Paketen eine
eigene Definition von Funktionen komplett verboten. Lediglich ein einzelner
Cloudprovider hatte ein Angebotsmodel bei welchem die Kunden das betriebene
API System in Hinischt der Funktionen modifizieren durfte.

Aufgrund diesen Erkentnissen hat sich das Projektteam während der Evaluierung
für ein API System mit dem Namen ``Swagger'' entschieden. Es ist aktuell das
einzige API System auf dem Markt, welches den Betrieb auf einer selbst
administrierten Hardware-Umgebung zulässt.
\nl%

\subsection{Frontends}

\begin{outline}
  \1 grafana
  \1 graphite
  \1 zabbix-frontend
\end{outline}

\section{Realisierung}

\section{Userstories}
auflisten der einzelnen stories + implementierung

\section{Fazit}

\printglossaries%

\printbibliography[heading=bibnumbered]

%%% Local Variables:
%%% mode: latex
%%% TeX-master: "thesis-de"
%%% End:
