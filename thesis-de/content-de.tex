\section{Erklärung}
Hiermit erklären wir, dass wir die Arbeit selbstständig verfasst und und keine
anderen als die angegebenen Quellen und Hilfsmittel benutzt haben. Diese Arbeit
wurde keinem anderen Prüfungsausschuss in gleicher oder vergleichbarer Form
vorgelegt.

\vfill
{\centering
\renewcommand{\arraystretch}{0.9}
\begin{tabular}{p{0.25\textwidth}p{0.05\textwidth}p{0.25\textwidth}p{0.05\textwidth}p{0.25\textwidth}}
  \dotfill                    & & \dotfill                      & & \dotfill \\
  \centering\footnotesize{Tim Meusel}& & \centering\footnotesize{Marcel Reuter}& & \centering\footnotesize{Nikolai Luis}%
\end{tabular}
}

\newpage

\section{Einführung}
Das Heinrich-Hertz-Europakolleg Bonn verlang im fünften und sechsten Semester
der Weiterbildung zum staatlich geprüften Informatiker eine fachbezogene
Projektarbeit. Dieses Projekt wird in Gruppen von zwei bis vier Personen
durchgeführt und soll fachliche Inhalte sowie Inhalte aus dem Projektmanagement
kombinieren. Es handelt sich um eine praktische Arbeit. Jeder Abschnitt enthält
ein Kürzel des Authors, hierbei bedeutet:
\begin{outline}
  \1 {[NL]} erstellt von Nikolai Luis
  \1 {[MR]} erstellt von Marcel Reuter
  \1 {[TM]} erstellt von Tim Meusel
\end{outline}
\tm%

\section{Projektvorstellung}
Cloud Provider bieten verschiedenste virtuelle Instanzen auf physischen Hosts
an. Dabei nutzt jeder virtuelle Server die vorhandenen Ressourcen des
physischen Systems unterschiedlich. Hier kommt es, aufgrund der
Mischkalkulation für die Ressourcen, zu einer Überbuchung (Overcommitment) des
Hosts. Weil das Monitoring nicht ausreichend ist, oder kein sinnvolles
Placement implementiert ist (Placement beschreibt den Algorithmus der einen
Node ermittelt auf dem eine neue virtuelle Instanz angelegt wird) kommt es
regelmäßig zu Performanceeinbußen. Für Kunden gibt es keine Transparenz über
die ihm zugeteilten und durch ihn genutzten Ressourcen, weshalb auch keine
ressourcenbasierte Abrechnung erfolgen kann. Teamleiter sind häufig mit der
Effizienzsteigerung der Plattform beschäftigt und müssen die Auslastung
steigern. Dies ist ohne detaillierte Auslastungsreports nicht möglich.

In diesem Projekt soll eine funktionierende Open Source Software entwickelt
werden, die sich in drei Teile gliedert:
\begin{outline}
  \1 Die verschiedenen Ressourcetypen (CPU Zeit / Daten Durchsatz / RAM
  Auslastung / Speicher Auslastung / Netzwerk Durchsatz) der einzelnen
  virtuellen Server müssen in einem sinnvollen Intervall periodisch ermittelt
  werden.
  \1 Die Daten müssen aggregiert und gespeichert werden. Hierbei ist auf eine
  Skalierung auf mindestens 10.000 virtuelle Instanzen unter Berücksichtigung
  der Verfügbarkeit und Performance der Datenbank zu achten (Sharding oder
  Replikation, verteilt oder zentral, dokumentenbasiert oder relational).
  \1 Diese Daten können dann dem Endanwender präsentiert werden (API und
  Web-UI). Hierzu wird eine Userstory Erhebung unter den drei Anwendertypen
  Kunde, Administrator, Manager bei Partnerunternehmen durchgeführt, um
  gewünschte Algorithmen zur Visualisierung zu ermitteln (zum Beispiel
  ermitteln von freien oder überbuchten Nodes, grafische Auswertung für Kunden).
\end{outline}

Dieses Projekt eignet sich besonders gut als Projektarbeit, da es in drei Teile
gegliedert ist. Jeder dieser Teile ist eigenständig und wird einem
Projektmitglied zugeordnet. Dies erleichtert die spätere Bewertung.
\tm%

\subsection{Projektteam}
Das Projektteam besteht aus den drei Mitgliedern Marcel Reuter, Nikolai Luis
und Tim Meusel.
\tm%

\subsubsection{Marcel Reuter}
Herr Reuter beendete 2013 seine Ausbildung zum IT-Systemelektroniker und
arbeitet seit dem bei der Firma EBF-EDV Beratung Föllmer GmbH. Er ist
verantwortlich für die visuelle Schnittstelle des Projekts (Punkt 3).
\mr%

\subsubsection{Nikolai Luis}
Herr Luis begann die Weiterbildung zum Techniker wärend seiner Ausbildung zum
Fachinformatiker Anwendungsentwicklung, welche er im Januar 2016 beendete. Er
arbeitet als BIG Data Analyst bei der Deutschen Telekom. Er ist verantwortlich
für die Speicherung der Daten und die automatisierte Schnittstelle (Punkt 2).
\nl%

\subsubsection{Tim Meusel}
Herr Meusel schloss seine Ausbildung zum Fachinformatiker Systemintegration
2012 ab. Aktuell arbeitet er als Systems Engineer bei der Host Europe Group.
Er verantwortet die Ermittlung sowie Übertragung der Daten.
\tm%

\subsection{Auftraggeber}
Die Ansprechpartner für dieses Projekt ist das Unternehmen Puppet Inc. (im
folgenden Puppet) in der Rolle als Auftraggeber, welches von den Mitarbeitern
Herrn David Schmitt und Herrn Steve Quin vertreten wird. Puppet ist Marktführer
im Bereicht Konfigurationsmangement Software. Software dieser Art hilft es
Administratoren sehr einfach Testumgebungen aufzubauen und auch
Produktivumgebungen zu Verwalten. Das Kernprodukt der Firma Puppet, welches
ebenfalls puppet heißt, steht unter einer Open Source Lizenz und darf frei
genutzt werden. Der Einsatz der Software erlaubt es dem Projektteam
verschiedenste Prototypen in kurzer Zeit zu bauen. Puppet entwickelt außerdem
Software zum Testen. Dies vereinfacht das Qualitätsmanagement im Projekt.
Puppet hat in der Vergangenheit bewiesen, mit agiler Entwicklung und diversen
Testverfahren umgehen zu können, beides wird intensiv in deren Teams genutzt.
Herr Quin und Herr Schmitt bieten tiefgreifendes Wissen zu den verschiedenen
Programmen und Techniken, um das Projektteam zu untersützen und zu beraten.
\tm%

\subsection{Aktuelle Situation}

\subsection{Anforderungen}
Die Detailanforderungen werden getrennt für die drei Bereiche des Projekts im
folgenden Abschnitt beschrieben. Diese Ergebene sich aus dem Lastenheft und
Pflichtenheft sowie aus den User Stories der Partner. Für Alle Lösungen gillt,
das sie eine aktive Community haben und unter einer Open Source Lizenz stehen.
\tm%

\subsubsection{Datenerfassungssysteme}
Unterschiedliche Ausbaustufen für virtuelle Machinen werden über mehrere
verschiedene Ressourcen, die sich in Ihrer Leistungsklasse unterscheiden,
differenziert. Diese Typen müssen für jede virtuelle Machine ausgelesen werden.
Als Referenz werden die Typen der größten deutschen Anbieter nachweis! für
virtuelle Machinen genommen. Diese sind aktuell:

Zugewiesener Arbeitsspeicher, Anzahl/Leistung der Prozessorkerne, Durchsatz des
Datenträgers, Anbindung an das Internet, nachweis via scrots fachwörter
referenz?

Diese Werte müssen sowohl in den virtuellen Machinen als auch auf dem Host
ermittelt werden können. Im Bereich Managed Hosting hat der Hoster Zugriff in
die virtuellen Instanzen und kann dort direkt sehr detailliert Daten ermitteln.
Im klassischen vServer Bereich (dem Bereitstellen des vServers als Produkt,
nicht als Service) hat der Betreiber keinen Zugriff in die VM und muss Daten
vom Host aus ermitteln. Auf dem Host muss zusätzlich Gesamtauslastung der
Ressourcen sowie der Zustand der Hardware ermittelt werden.

Gesammelte Daten müssen lokal zwischengespeichert werden können, um einem
Verlust bei Netzwerkstörungen zu vermeiden. Der Versand muss nach dem Push- und
nicht nach dem Polling-Verfahren arbeiten. Somit kann Eventbasiert gearbeitet
werden, dies verhindert Overhead im Netzwerk. Eine Visualisierung in Echtzeit
ist nicht gefodert weshalb Daten lokal zwischengespeichert werden können um sie
gebündelt zu verschicken.
\tm%

\subsubsection{Datenbanksysteme}

\subsubsection{Frontends}
Die Weboberfläche muss ausgewählten Benutzern eine Übersicht über die
verbrauchten oder freien Ressourcen, wie z.B. CPU Auslastung oder RAM
Auslastung einzelner virtueller oder physischen Maschinen bereitstellen.
Hierbei ist wichtig, dass die Sicherheit (Authentisierung, Authentifizierung
und Authorisierung) der Daten gewährleistet wird. Dies bedeutet, das sowohl die
Verbindung zwischen Datenbank und Weboberfläche gesichert sein muss, als auch
die Weboberfläche selber, gegen Zugriff von Unbekannten. Die Weboberfläche
greift hier mittels API Abfragen auf die Datenbank zu, dies ermöglicht uns den
Zugriff nur für das Frontend auf die Datenbank gewährleisten zu können.  Die
Weboberfläche soll Administratoren bei Neukonfiguration von virtuellen
Maschinen unterstützen und zeigt hier die Auslastung der einzelnen physischen
Systemen. Der Administrator kann so gezielt die einzelnen physischen Maschinen
besser auslasten und verteilen. Ebenfalls kann das Frontend eine Schnittstelle
für komplexe Abfragen und Analysen bereitstellen. Bei Bedarf können diese auch
visuell ausgegeben werden.
\mr%

\section{Projektmanagement}
agil, sprints, userstories

\section{Verwendete Tools}
ist das relevant? als eigener punkt?

\section{Analyse von Softwarekomponenten}
Zusammen mit dem Auftraggeber folgende Liste and Komponenten erstellt, die
vergleichen und gegebenfalls evaluiert werden sollen:

\subsection{Datenerfassungssysteme}

\begin{outline}
  \1 collectd mit Virt Plugin
  \1 coreutils
  \1 zabbix-agent
  \1 python-diamond
  \1 sysstat
  \1 atop
  \1 logtash
  \1 riemann
\end{outline}

\subsection{Datenbanksysteme}
\label{subsec:datenbanksysteme}
Zu Beginn des Projektes musste das Projektteam mögliche Datenhaltungssysteme in
einer Liste für eine darauffolgende Evaluierung definieren. Bei der Erstellung
dieser Liste wurden Systeme aufgenommen, welche den Projektmitgliedern aus
vergangenen Projekten, beziehungsweise aus dem täglichen Arbeitstag bereits
bekannt waren. Zusätzlich hatten die Projektmitglieder vorab mit Experten aus
dem eigenen Unternehmen und aus dem Unternehmen des Auftraggebers mögliche
weitere Systeme gesprochen.  Dadurch das Herr Luis in dem Bereich der
Massendatenhaltungssysteme arbeitet, sind ebenfalls Systeme aufgenommen worden,
welche im Bereich der sogenannten Big Data Technologie arbeiten.

Das hinzufügen von weiteren/neuen Datenbanksystemen zur späteren
Evaluierung erfolgt nur nach einer Genehmigung vom Auftraggeber.

Die in der Liste aufgenommen Systeme sind folgende:
\begin{outline}
  \1 elasticsearch
  \1 cassandra
  \1 postgres
  \1 OpenTSDB
  \1 KNIME
  \1 impala
  \1 hadoop
  \1 hive
\end{outline}
\nl%

\subsubsection{Vorbereitung der Evaluierung}
\label{subsubsec:vorbereitung_der_evaluierung}
Als Vorbereitung für die Evaluierung der Datenhaltungssysteme wurden mehrere
Kriterien definiert, welche einen Leitpfaden für die spätere Arbeit geben.
Während der Auswahl und Definition der von dem Datenbanksystem zu erfüllenden
Kriterien wurde darauf geachtet, dass diese sich aus dem späteren Zielsystem
ableiten lassen und ebenfalls realistisch erfüllbar sind. Anschließend wurde
die Relevanz jedes Kirteriums von den Teammitgliedern beurteilt und festgelegt.

Die daraus resultierten Kriterien sind nun nach Relevanz absteigend
aufgelistet:
\begin{outline}
  \1 Bereits vorhandenes Wissen über die Datenbank. Dazu gehört, dass diese
  eine umfangreiche Dokumentation besitzt und eine aktive und große Community
  für Diskussionen und Fragen besitzt.  Ebenfalls fallen die vorhandenen
  Vorkenntnisse über das Datenbanksystem der Projektmitglieder, sowie die
  Mitarbeiter von dem Unternehmen des Auftraggebers unter diesem Kriterium.  Es
  ist wichtig, dass die zu aufwendene Einarbeitungszeit während dem Projekt und
  im späteren Wirkbetrieb bei Mitarbeitern des Unternehmens so gering wie
  möglich gehalten wird.
  \1 Das Datenhaltungssystem muss einfach, schnell und jederzeit erweiterbar
  sein. Dies bedeutet, dass Ressourcen (CPU/RAM/Speicherplatz) des
  Datenbanksystems im optimalsten Fall während dem normalen Betrieb (kein
  Neustart oder Auszeit) aufgestockt werden kann. Es ist jedoch nach
  Anforderung des Auftraggebers ausreichend, wenn eine Aufstockung nach einer
  Auszeit von maximal 45 Minuten erfolgt ist.
  \1 Umweltschonende und effiziente Datenhaltung ist zu beachten.  Dies
  definiert eine optimale Relation zwischen der verwendeten Hardware und dem
  daraus resultierendem Ergebnis. Eine umweltschonende Datenhaltung kann
  unabhängig der korrekten Auswahl der genutzten Stromeffiziensklasse jedes
  Hardwarebauteils bereits bei der im Wirkbetrieb zu benutzende Software
  beeinflusst werden.  Dies ist im Bereich der Datenhaltungs-Software die
  entstehende Speicherplatzgröße für einen einzelnen Datensatz in einer
  Datenbank. Um so kleiner dieser ist, desto mehr kann bei Energiekosten für
  zusätzlichen Speicherplatz eingesparrt werden.

  Unter Berücksichtigung der genannten Attribute, soll jedoch die
  Datenhaltungs-Software mit der gleichen Hardware-Ressource das beste und
  schnellste Ergebnis erbringen können. Dies bedeutet, dass auch bei komplexen
  Analysetechniken und zusätzlich hohen Datenmengen, dass System weiterhin
  gegenüber anderen Datenhaltungssystemen ein valides und schnelles
  Analyseergebnis erbringen kann.
  \1 Die Datenhaltungs-Software sollte bereits Methoden zur Datensicherung
  besitzen. Dabei ist zum einem eine Backup Methode, und zum anderen das
  Verhalten bei Problemen am System, zum Beispiel ein Defekt am Netzwerkkabel,
  mit inbegriffen.
\end{outline}
\nl%

\subsubsection{Durchführung der Evaluierung}
\label{subsubsec:durchfuehrung_der_evaluierung}
Bei der Durchführung der Evaluierung von den ausgewählten Datenhaltungssystemen
aus der Liste in Punkt~\ref{subsec:datenbanksysteme} wurden die
Evaluierungskriterien aus Punkt~\ref{subsubsec:vorbereitung_der_evaluierung}
verwendet. Diese dienten als Leitpfaden für jedes Datenhaltungssystem und
wurden von jedem Projektmitglied beachtet.

Zu Beginn der Durchführung teilte Herr Meusel als Projektleiter die zu
evaluierenden Datenhaltungssysteme jedem Projektmitglied zu, um eine schnellere
Bearbeitung zu gewährleisten. Bei der Einteilung wurden die bereits vorhandenen
Erfahrungen jeder Projektmitglieder zu dem jeweiligen Datenhaltungssystem
berücksichtigt. Im zweiten Schritt sollte sich anschließend jedes
Projektmitglied mit dem zugeteilten Dantenhaltungssystem kurz
auseinandersetzen. In diesem Schritt ist Herrn Luis aufgefallen, dass es sich
bei der in der Liste aufgenommenen Software ``KNIME'' nicht um ein
Datenhaltungssystem handelt, sondern um ein Werkzeug zur Datenanalyse,
Datenaufbereitung und Datendarstellung.

Bei der Evaluation von den anderen Datenhaltungssystemen konnten folgende
Ergebnisse ermittelt werden:
\begin{outline}
  \1 elasticsearch

  TODO: paste text here
  \1 cassandra

  TODO: paste text here
  \1 postgres

  TODO: paste text here
  \1 OpenTSDB

  TODO: paste text here
  \1 impala

  TODO: paste text here
  \1 hadoop

  TODO: paste text here
  \1 hive

  TODO: paste text here
\end{outline}
\nl%

\subsection{Frontends}

\begin{outline}
  \1 grafana
  \1 graphite
  \1 zabbix-frontend
\end{outline}

\section{Realisierung}

\section{Userstories}
auflisten der einzelnen stories + implementierung

\section{Fazit}

\section{Quellennachweis}
\printbibliography%
%%% Local Variables:
%%% mode: latex
%%% TeX-master: "thesis-de"
%%% End:
