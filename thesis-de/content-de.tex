\section{Erklärung}
Hiermit erklären wir, dass wir die Arbeit selbstständig verfasst und und keine
anderen als die angegebenen Quellen und Hilfsmittel benutzt haben. Diese Arbeit
wurde keinem anderen Prüfungsausschuss in gleicher oder vergleichbarer Form
vorgelegt.

\vfill
{\centering
\renewcommand{\arraystretch}{0.9}
\begin{tabular}{p{0.25\textwidth}p{0.05\textwidth}p{0.25\textwidth}p{0.05\textwidth}p{0.25\textwidth}}
  \dotfill                    & & \dotfill                      & & \dotfill \\
  \centering\footnotesize{Tim Meusel}& & \centering\footnotesize{Marcel Reuter}& & \centering\footnotesize{Nikolai Luis}%
\end{tabular}
}

\newpage

\section{Einführung}
Das Heinrich-Hertz-Europakolleg Bonn verlang im fünften und sechsten Semester
der Weiterbildung zum staatlich geprüften Informatiker eine fachbezogene
Projektarbeit. Dieses Projekt wird in Gruppen von zwei bis vier Personen
durchgeführt und soll fachliche Inhalte sowie Inhalte aus dem Projektmanagement
kombinieren. Es handelt sich um eine praktische Arbeit. Jeder Abschnitt enthält
ein Kürzel des Authors, hierbei bedeutet:
\begin{outline}
  \1 \verb+[+NL\verb+]+ erstellt von Nikolai Luis
  \1 \verb+[+MR\verb+]+ erstellt von Marcel Reuter
  \1 \verb+[+TM\verb+]+ erstellt von Tim Meusel
\end{outline}

\section{Projektvorstellung}
Cloud Provider bieten verschiedenste virtuelle Instanzen auf physischen Hosts
an. Dabei nutzt jeder virtuelle Server die vorhandenen Ressourcen des
physischen Systems unterschiedlich. Hier kommt es, aufgrund der
Mischkalkulation für die Ressourcen, zu einer Überbuchung (Overcommitment) des
Hosts. Weil das Monitoring nicht ausreichend ist, oder kein sinnvolles
Placement implementiert ist (Placement beschreibt den Algorithmus der einen
Node ermittelt auf dem eine neue virtuelle Instanz angelegt wird) kommt es
regelmäßig zu Performanceeinbußen. Für Kunden gibt es keine Transparenz über
die ihm zugeteilten und durch ihn genutzten Ressourcen, weshalb auch keine
ressourcenbasierte Abrechnung erfolgen kann. Teamleiter sind häufig mit der
Effizienzsteigerung der Plattform beschäftigt und müssen die Auslastung
steigern. Dies ist ohne detaillierte Auslastungsreports nicht möglich.

In diesem Projekt soll eine funktionierende Open Source Software entwickelt
werden, die sich in drei Teile gliedert:
\begin{outline}
  \1 Die verschiedenen Ressourcetypen (CPU Zeit / Daten Durchsatz / RAM
  Auslastung / Speicher Auslastung / Netzwerk Durchsatz) der einzelnen
  virtuellen Server müssen in einem sinnvollen Intervall periodisch ermittelt
  werden.
  \1 Die Daten müssen aggregiert und gespeichert werden. Hierbei ist auf eine
  Skalierung auf mindestens 10.000 virtuelle Instanzen unter Berücksichtigung
  der Verfügbarkeit und Performance der Datenbank zu achten (Sharding oder
  Replikation, verteilt oder zentral, dokumentenbasiert oder relational).
  \1 Diese Daten können dann dem Endanwender präsentiert werden (API und
  Web-UI). Hierzu wird eine Userstory Erhebung unter den drei Anwendertypen
  Kunde, Administrator, Manager bei Partnerunternehmen durchgeführt, um
  gewünschte Algorithmen zur Visualisierung zu ermitteln (zum Beispiel
  ermitteln von freien oder überbuchten Nodes, grafische Auswertung für Kunden).
\end{outline}

Dieses Projekt eignet sich besonders gut als Projektarbeit, da es in drei Teile
gegliedert ist. Jeder dieser Teile ist eigenständig und wird einem
Projektmitglied zugeordnet. Dies erleichtert die spätere Bewertung.

\subsection{Projektteam}
Das Projektteam besteht aus den drei Mitgliedern Marcel Reuter, Nikolai Luis
und Tim Meusel.

\subsubsection{Marcel Reuter}
Herr Reuter beendete 2012 seine Ausbildung zum Fachinformatiker
Systemintegration und arbeitet seit dem bei einer firma. Er ist verantwortlich
für die visuelle Schnittstelle des Projekts (Punkt 3).

\subsubsection{Nikolai Luis}
Herr Luis begann die Weiterbildung zum Techniker wärend seiner Ausbildung zum
Fachinformatiker Anwendungsentwicklung, welche er 2014 beendete. Er arbeitet
als irgendwasmitdatenbanken bei der Deutschen Telekom. Er ist verantwortlich
für die Speicherung der Daten und die automatisierte Schnittstelle (Punkt 2).

\subsubsection{Tim Meusel}
Herr Meusel schloss seine Ausbildung zum Fachinformatiker Systemintegration
ebenfalls 2012 ab. Aktuell arbeitet er als Systems Engineer bei der Host Europe
Group. Er verantwortet die Ermittlung sowie Übertragung der Daten.

\subsection{Auftraggeber}
Die Ansprechpartner für dieses Projekt ist das Unternehmen Puppet Inc. (im
folgenden Puppet) in der Rolle als Auftraggeber, welches von den Mitarbeitern
Herrn David Schmitt und Herrn Steve Quin vertreten wird. Puppet ist Marktführer
im Bereicht Konfigurationsmangement Software. Software dieser Art hilft es
Administratoren sehr einfach Testumgebungen aufzubauen und auch
Produktivumgebungen zu Verwalten. Das Kernprodukt der Firma Puppet, welches
ebenfalls puppet heißt, steht unter einer Open Source Lizenz und darf frei
genutzt werden. Der Einsatz der Software erlaubt es dem Projektteam
verschiedenste Prototypen in kurzer Zeit zu bauen. Puppet entwickelt außerdem
Software zum Testen. Dies vereinfacht das Qualitätsmanagement im Projekt.
Puppet hat in der Vergangenheit bewiesen, mit agiler Entwicklung und diversen
Testverfahren umgehen zu können, beides wird intensiv in deren Teams genutzt.
Herr Quin und Herr Schmitt bieten tiefgreifendes Wissen zu den verschiedenen
Programmen und Techniken, um das Projektteam zu untersützen und zu beraten.

\subsection{Aktuelle Situation}

\subsection{Anforderungen}
Die Detailanforderungen werden getrennt für die drei Bereiche des Projekts wie
folgt definiert:
\subsection{Datenerfassungssysteme}

\subsection{Datenbanksysteme}

\subsection{Frontends}

\section{Projektmanagement}
agil, sprints, userstories

\section{Verwendete Tools}
ist das relevant? als eigener punkt?

\section{Analyse von Softwarekomponenten}
Zusammen mit dem Auftraggeber folgende Liste and Komponenten erstellt, die
vergleichen und gegebenfalls evaluiert werden sollen:

\subsection{Datenerfassungssysteme}

\begin{outline}
  \1 collectd mit Virt Plugin
  \1 coreutils
  \1 zabbix-agent
  \1 python-diamond
  \1 sysstat
  \1 atop
  \1 logtash
  \1 riemann
\end{outline}

\subsection{Datenbanksysteme}

\begin{outline}
  \1 elasticsearch
  \1 cassandra
  \1 postgres
  \1 OpenTSDB
  \1 KNIME
  \1 impala
  \1 hadoop
  \1 hive
\end{outline}

\subsection{Frontends}

\begin{outline}
  \1 grafana
  \1 graphite
  \1 zabbix-frontend
\end{outline}

\section{Implementierung}

\section{Userstories}
auflisten der einzelnen stories + implementierung

\section{Fazit}

\printbibliography
%%% Local Variables:
%%% mode: latex
%%% TeX-master: "thesis-de"
%%% End:
