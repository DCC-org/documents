\chapter*{Danksagung}
Wir bedanken uns bei der Firma Puppet, insbesondere bei David Schmitt und Steve
Quin. Euer Fachwissen und eure Unterstützung haben uns sehr geholfen, das
Projekt ordentlich zu strukturieren und auf einem hohen technischen Niveau
abzuschliessen. Außerdem möchten wir uns bei Hunter Haugen für die
große Hilfestellung im Bereich Unit Testing bedanken.

Ein besonderer Dank gebührt Ulli Kehrle. Ohne seine unermüdlichen Erklärungen
zum Thema \LaTeX{} und seine Hilfestellungen, auch in den späten Abendstunden,
würde diese Dokumentation nicht in dieser Form existieren.

\newpage

\tableofcontents
\listoffigures
\begingroup
\let\clearpage\relax
\lstlistoflistings{}
\listoftables
\endgroup

\chapter{Einführung}
\label{chap:einfuehrung}
Das Heinrich-Hertz-Europakolleg Bonn verlangt im fünften und sechsten Semester
der Weiterbildung zum staatlich geprüften Informatiker eine fachbezogene
Projektarbeit. Dieses Projekt wird in Gruppen von zwei bis vier Personen
durchgeführt und soll fachliche Inhalte sowie Techniken aus dem
Projektmanagement kombinieren. Es handelt sich um eine praktische Arbeit. Jeder
Abschnitt enthält am Ende das Kürzel des Autors, hierbei bedeutet:

\begin{outline}
  \1 {[MR]} erstellt von Marcel Reuter
  \1 {[NL]} erstellt von Nikolai Luis
  \1 {[TM]} erstellt von Tim Meusel
\end{outline}

Die Zusammensetzung dieses Teams entstand während der regulären Unterrichtszeit
im dritten Semester. Herr Meusel und Herr Luis sprachen während eines
Abendessens über die Inhalte ihres täglichen Arbeitstages. Herr Meusel arbeitete
zu diesem Zeitpunkt an einem Projekt im Bereich Cloud-Hosting und Herr Luis war
im Bereich von Datenanalysen mit großen Datenmengen und dessen Methodiken
beschäftigt. Sehr schnell entstand aus diesen beiden Gesprächsinhalten die
Softwareidee, welche in diesem Dokument vorgestellt wird. Da Herrn Meusel und
Herrn Luis der Aufwand dieser Lösung bewusst war und beide keine Expertise im
Bereich Frontend-Entwicklung, bzw. Datenvisualisierung besaßen, stellten Sie
die Idee ihren Mitstudenten vor. Herr Reuter zeigte dabei hohes Interesse und
war vor allem vom englischsprachigen Teil des Projektes überzeugt. Die von den
Dozenten definierten Vorgaben waren somit erfüllt und das Projekt konnte für
das fünfte und sechste Semester angemeldet werden.

Die größte Hürde in einem Projekt von diesem Umfang ist die fehlende Zeit für
die Projektarbeit. Alle drei Schüler führen die Weiterbildung nebenberuflich
durch und müssen die Leistung neben einer regulären Arbeitszeit,
Unterrichtseinheiten und Privatleben erbringen. Eine gut strukturierte
Zeitplanung und Projektorganisation ist dabei unverzichtbar. Gleichzeitig muss
jeder der Schüler auf sein eigenes Wohlbefinden achten. Ein zu schnelles
Arbeitstempo würde einen Einzelnen unter der Last zusammenbrechen lassen und
somit, wie auch bei einem zu langsamen Arbeitstempo, das Projektziel gefährden
können.

Bei der Durchführung des Projektes wurde bei jedem Projektmitglied der
praktische Wissensschatz erweitert. Zum einem wurden neue Wissensinhalte zu
Techniken und Arbeitsmethoden erschlossen, zum anderen Vorteile weiter
fortgebildet und Nachteile erneut auf die Probe gestellt. Entsprechend wurde
zum Beispiel die Belastbarkeit jedes Einzelnen erforscht und die
Selbsteinschätzung somit weiter fortgebildet oder sogar korrigiert.

In einer globalisierten Arbeitswelt stehen die Möglichkeiten von IT-Systemen
und dessen verbundenen Werkzeugen, wie die Optimierungen von Arbeitsabläufen
mit Hilfe von Analysen und Erkenntnissen, sehr stark im Vordergrund.
Gleichzeitig beschäftigen sich Computersystem-Anbieter mit der optimalen
Auslastung ihrer Serverfarmen (Hardware) und den damit verbundenen
Kostenoptimierungen. Durch einen neu entstandenen Trend des Cloud-Hostings ist
die optimale Auslastung der Hardwaremittel wichtiger denn je. Jedoch besitzt
der Markt zum aktuellen Zeitpunkt noch keine Management-Software, welche
eine voll- oder halbautomatische Optimierung von ganzen Rechenzentren oder
Servergruppen übernehmen kann. Entsprechend versuchen Systemadministratoren
weiterhin auf ihren zugewiesenen Systemgruppen eine optimale Leistung zu
erzielen. Dabei fehlt diesen Personen eine Gesamtsicht auf Ihre oder andere
Servergruppen, entsprechend fällt der Nutzen der getätigten Optimierungen im
Vergleich sehr gering aus.

Das Problem der fehlenden Übersicht soll in dieser Projektarbeit gelöst und
dokumentiert werden. Das genaue Ziel ist die Entwicklung von einem
Softwaresystem aus drei Hauptkomponenten für den Einsatz in großen
Rechenzentren für die Ausarbeitung einer Ressourcen-Übersicht. Diese soll
primär von IT-Fachkräften einsehbar sein, sodass die Gesamtperformance des
Rechenzentrums optimiert werden kann. Je nach Datenschutzeinwilligung der
Kunden der einzelnen Computersystem-Anbieter können diese Daten auch
sekundär für ein optimiertes Kundengespräch oder Marketing-Kampagnen eingesetzt
werden. Jedoch handelt es sich dabei nur um ein Nebenprodukt und wird in dieser
Projektarbeit und Dokumentation nicht betrachtet.

Die Methodiken der Projektorganisation, sowie der vereinzelte Zugang zu
Literatur, erarbeiten sich die Projektmitglieder über ein Selbststudium, Inhalte
aus dem Unterricht oder Erfahrungsberichte vom Auftrageber. Dies
beinhaltet auch die Beurteilung von den genutzten Quellen, welche für die
Projektarbeit genutzt und zitiert werden. Es wird vorab bereits durch diese
Wissensdatenbanken ein grober Projektvorgang herausgearbeitet, welcher jedoch
auch in den nachfolgenden Seiten beschrieben ist.
\all%

\chapter{Projektvorstellung}
\label{subsec:projektvorstellung}
Cloud Provider bieten verschiedenste virtuelle Instanzen auf physischen Hosts
an. Dabei nutzt jeder virtuelle Server die vorhandenen Ressourcen des
physischen Systems unterschiedlich. Hier kommt es, aufgrund der
Mischkalkulation für die Ressourcen, zu einer Überbuchung (Overcommitment) des
Hosts. Weil das Monitoring nicht ausreichend ist oder kein sinnvolles Placement
implementiert ist (Placement beschreibt den Algorithmus, der einen Node
ermittelt auf dem eine neue virtuelle Instanz angelegt wird), kommt es
regelmäßig zu Leistungseinbußen. Für Kunden gibt es keine Transparenz über die
ihm zugeteilten und durch ihn genutzten Ressourcen, weshalb auch keine
ressourcenbasierte Abrechnung erfolgen kann. Teamleiter sind häufig mit der
Effizienzsteigerung der Plattform beschäftigt und müssen die Auslastung
steigern. Dies ist ohne detaillierte Berichte über die Auslastung nicht
möglich.

In diesem Projekt soll eine funktionierende Open Source Software entwickelt
werden, die sich in drei Teile gliedert:

\begin{outline}
  \1 Die verschiedenen Ressourcetypen (CPU Zeit / Datendurchsatz / RAM
  Auslastung / Speicher Auslastung / Netzwerkdurchsatz) der einzelnen
  virtuellen Server müssen in einem sinnvollen Intervall periodisch ermittelt
  werden.
  \1 Die Daten müssen aggregiert und gespeichert werden. Hierbei ist auf eine
  Skalierung auf mindestens 10.000 virtuelle Instanzen unter Berücksichtigung
  der Verfügbarkeit und Performance der Datenbank zu achten (Sharding oder
  Replikation, verteilt oder zentral, dokumentenbasiert oder relational).
  \1 Diese Daten können dann dem Endanwender präsentiert werden (\gls{API} und
  Web-UI). Hierzu wird eine Userstory-Erhebung unter den drei Anwendertypen
  Kunde, Administrator und Manager bei Partnerunternehmen durchgeführt, um
  gewünschte Algorithmen zur Visualisierung zu ermitteln (zum Beispiel
  ermitteln von freien oder überbuchten Nodes, grafische Auswertung für Kunden).
\end{outline}

Dieses Projekt eignet sich besonders gut als Projektarbeit, da es in drei Teile
gegliedert ist. Jeder dieser Teile ist eigenständig und wird einem
Projektmitglied zugeordnet. Dies erleichtert die spätere Bewertung.
\all%

\section{Projektteam}
Das Projektteam besteht aus den drei Mitgliedern Marcel Reuter, Nikolai Luis
und Tim Meusel.
\all%

\subsection{Marcel Reuter}
Herr Reuter beendete 2013 seine Ausbildung zum IT-Systemelektroniker und
arbeitet seitdem bei der Firma EBF-EDV Beratung Föllmer GmbH. Er ist
verantwortlich für die visuelle Schnittstelle des Projekts (Punkt 3).
\mr%

\subsection{Nikolai Luis}
Herr Luis begann die Weiterbildung zum Techniker während seiner Ausbildung zum
Fachinformatiker Anwendungsentwicklung, welche er im Januar 2016 beendete. Er
arbeitet als BIG Data Analyst bei der Deutschen Telekom. Er ist verantwortlich
für die Speicherung der Daten und die automatisierte Schnittstelle (Punkt 2).
\nl%

\subsection{Tim Meusel}
Herr Meusel schloss seine Ausbildung zum Fachinformatiker Systemintegration
2012 ab. Aktuell arbeitet er als Systems Engineer bei der Host Europe Group.
Er verantwortet die Ermittlung sowie Übertragung der Daten.
\tm%

\section{Auftraggeber}
Der Ansprechpartner für dieses Projekt ist das Unternehmen Puppet Inc. (im
folgenden Puppet) in der Rolle als Auftraggeber, welches von den Mitarbeitern
Herrn David Schmitt und Herrn Steve Quin vertreten wird. Puppet ist Marktführer
im Bereich der Konfigurationsmanagement-Software. Software dieser Art hilft
Administratoren, sehr einfach Testumgebungen aufzubauen und auch
Produktivumgebungen zu verwalten. Das Kernprodukt der Firma Puppet, welches
ebenfalls puppet heißt, steht unter einer Open Source Lizenz und darf frei
genutzt werden. Der Einsatz der Software erlaubt es dem Projektteam,
verschiedenste Prototypen in kurzer Zeit zu bauen. Puppet entwickelt außerdem
Software zum Testen. Dies vereinfacht das Qualitätsmanagement im Projekt.
Puppet hat in der Vergangenheit bewiesen, mit agiler Entwicklung und diversen
Testverfahren umgehen zu können. Beides wird intensiv in deren Teams genutzt.
Herr Quin und Herr Schmitt bieten tiefgreifendes Wissen zu den verschiedenen
Programmen und Techniken, um das Projektteam zu unterstützen und zu beraten.
\all%

\section{Aktuelle Situation}
Im Abschnitt~\ref{subsec:projektvorstellung} wurden bereits die Intentionen des
Projekts erläutert. In der Vergangenheit wurde schon mal versucht eine passende
Softwarelösung zu entwickeln. OpenStack ist ein Softwareprojekt, welches große
Mengen an Rechen- und Netzwerkkapazitäten, sowie persistenten Speicher in einem
Rechenzentrum zu einer \gls{Public Cloud} oder \gls{Private Cloud} Umgebung
zusammenfasst und orchestriert (vgl.~\cite{OpenStack_Intro}). Eines der
Teilprojekte ist Telemetry. Das Ziel hiervon ist es, zuverlässig Daten von
physischen und virtuellen Ressourcen zu ermitteln und verlässlich zu speichern.
Die Daten sollen zur Analyse genutzt werden und Aktionen auslösen, wenn
bestimmte Kriterien erreicht sind (vgl.~\cite{OpenStack_Telemetry}). Telemetry
bringt leider mehrere Nachteile mit sich:

\begin{outline}
  \1 Die Standarddatenbank für Telemetry war lange Zeit MongoDB\@. MongoDB ist
  eine dokumentenorientierte Datenbank. Darin werden Daten nicht in Tabellen
  strukturiert, sondern in eigenständigen Dokumenten. Jedes Dokument hat eine
  eigene Datenstruktur (zum Beispiel im \gls{JSON}
  Format)(vgl.~\cite{Dokumentenorientierte_Datenbank}). Dies ermöglicht es, die
  Datenbank sehr einfach vertikal zu skalieren. Hierbei werden alle Dokumente
  auf mehrere Server verteilt. Schreib- und Lese-Anfragen können ebenfalls auf
  alle Server verteilt werden. Die Skalierung ist nahezu linear
  (vgl.~\cite{MongoDB_Architecture},vgl.~\cite{What_is_MongoDB}). Die Grundidee
  ist sehr gut und eignet sich für besonders große Datenmengen oder
  Installationen die hochverfügbar sein müssen. Die Implementierung in MongoDB
  hat allerdings diverse Nachteile. Kyle Kingsbury hat intensiv MongoDB mit
  \gls{Jepsen} getestet. Die Datenbank hat sich mehrfach als sehr unzuverlässig
  herausgestellt (vgl.~\cite{MongoDB_on_Jepsen}). Die beiden folgenden Punkte
  disqualifizieren MongoDB für einen Einsatz als persistenten Datenspeicher,
  da die Persistenz nicht sichergestellt ist. Es wurde von Kyle Kingsbury
  bewiesen, dass:
    \2 MongoDB bei einem Schreibvorgang bestätigt, dass Daten persistent
    gespeichert wurden, dies allerdings nicht immer der Fall ist. Ein
    unbemerkter Datenverlust ist die Folge.
    \2 MongoDB in bestimmten Situationen die falschen Daten bei einer
    Leseanfrage zurückliefert.

  \1 Das Telemetry Projekt hat diese Probleme ebenfalls erkannt und nach
  alternativen Datenbanken gesucht. Als keine vorhandene Lösung ihren
  Anforderungen entsprach, entschieden sie sich für die Entwicklung einer
  eigenen Datenbank: \gls{Gnocchi}. Diese Datenbank ist noch in der Entwicklung
  und noch nicht für den produktiven Einsatz bereit.
  \1 Es wird immer wieder von Skalierungsproblemen berichtet. Das CERN betreibt
  eine OpenStack-Installation und konnte die Probleme mit Telemetry nur mit
  immenser Hardware lösen. Die Kosten für diese Infrastruktur, als auch die
  Komplexität, sind viel zu hoch. Sie übersteigen das Budget der meisten
  Cloud-Umgebungen, wodurch diese unwirtschaftlich werden
  (vgl.~\cite{OpenStack_CERN}).
  \1 Das Telemetry Projekt ist sehr stark in OpenStack eingebunden. Der Einsatz
  in anderen Cloud-Umgebungen ist nicht vorgesehen, da die einzelnen Dienste
  aus dem Telemetry Projekt weitere OpenStack Komponenten benötigten. Ein
  eigenständiger Betrieb ohne OpenStack ist für die Zukunft nicht geplant.
\end{outline}

Aufgrund der langen Liste an Problemen ist Telemetry aktuell innerhalb einer
kleinen OpenStack-Installation teilweise nutzbar, jedoch nicht wirtschaftlich
in großen Umgebungen oder außerhalb von Openstack. Es ist nicht davon
auszugehen, dass diese Probleme kurz- bis mittelfristig gelöst würden, da diese
im Design und Architektur des Projekts liegen. Es gibt aktuell keine
Alternativen zu Telemetry mit einem gleichwertigen Funktionsumfang. Somit
entschied sich das Projektteam, eine Alternative zu entwickeln, die
unkompliziert zu installieren ist und unabhängig von OpenStack arbeitet.
\tm%

\section{Anforderungen}
\label{section:anforderungen}
Die Detailanforderungen werden getrennt für die drei Bereiche des Projekts im
folgenden Abschnitt beschrieben. Diese ergeben sich aus dem Lasten- und
Pflichtenheft sowie aus den Userstories der Partner. Für alle Lösungen gilt,
dass sie eine aktive Community haben und unter einer Open-Source-Lizenz stehen.
\tm%

\subsection{Datenerfassungssysteme}
\label{section:datenerfassungssysteme}
Unterschiedliche Ausbaustufen für virtuelle Maschinen werden über mehrere
verschiedene Ressourcen, die sich in ihrer Leistungsklasse unterscheiden,
differenziert. Diese Typen müssen für jede virtuelle Maschine ausgelesen
werden.  Als Referenz werden die Typen der größten deutschen Anbieter für
virtuelle Maschinen genommen. Diese sind aktuell:

\begin{outline}
  \1 Zugewiesener Arbeitsspeicher
  \1 Anzahl/Leistung der Prozessorkerne
  \1 Durchsatz des Datenträgers
  \1 Anbindung an das Internet
\end{outline}

Diese Werte müssen sowohl in den virtuellen Maschinen als auch auf dem Host
ermittelt werden können. Im Bereich Managed Hosting hat der Hoster Zugriff in
die virtuellen Instanzen und kann dort direkt sehr detailliert Daten ermitteln.
Im klassischen vServer-Bereich (dem Bereitstellen des vServers als Produkt,
nicht als Service) hat der Betreiber keinen Zugriff in die VM und muss Daten
vom Host aus ermitteln. Auf dem Host muss zusätzlich die Gesamtauslastung der
Ressourcen sowie der Zustand der Hardware ermittelt werden. Jede
Virtualisierungssoftware bietet Schnittstellen, um Daten über die laufenden
virtuellen Instanzen zu ermitteln. Dies gilt aber nur für Ressourcen, die die
Virtualisierungssoftware direkt verwaltet. Außerdem ist auf dem Host nicht
ersichtlich, wofür Ressourcen in den VMs genutzt werden. So kann zum Beispiel
ermittelt werden, wie viel CPU-Zeit einer Instanz zur Verfügung gestellt wird,
aber nicht wofür die Leistung genutzt wird (zum
Beispiel~\glslink{Soft-IRQ}{Soft-IRQs}, \glslink{Interrupt}{Interrupts},
\gls{iowait}). Um detaillierte Daten für eine bessere Analyse zu bekommen, muss
deshalb auch in den virtuellen Maschinen eine Datenerfassung erfolgen, sofern
Zugriff vorhanden ist.

Gesammelte Daten müssen lokal zwischengespeichert werden können, um einem
Verlust bei Netzwerkstörungen zu vermeiden. Der Versand muss nach dem Push- und
nicht nach dem Polling-Verfahren arbeiten. Somit kann eventbasiert gearbeitet
werden. Dies verhindert Overhead im Netzwerk. Es werden somit nur dann Daten
geschickt, wenn auch tatsächlich welche vorhanden sind. Beim Polling-Verfahren
muss eine zentrale Instanz in konstanten Intervallen abfragen, ob neue Daten
vorhanden sind. Eine Visualisierung in Echtzeit ist nicht gefordert, weshalb
Daten lokal zwischengespeichert werden können, um sie gebündelt zu verschicken.
\tm%

\subsection{Datenhaltungssystem}
Nachfolgende Anforderungen an das Datenhaltungssystem wurden vor der
Projektphase im Team und auch in der ersten Rücksprache mit dem Auftraggeber
definiert. Sie dienen als erster Leitpfaden der Softwarelösung und definierten
sich wie folgt:

\begin{outline}
  \1 Zunächst muss das Datenhaltungssystem für das Abspeichern von mehreren
  tausend Datensätzen in der Sekunde ausgelegt sein. Innerhalb des Projektes
  wird mit 10.000 Datensätzen pro Sekunde gerechnet und getestet.
  \1 Gleichzeitig muss das Datenhaltungssystem über Analysewerkzeuge verfügen,
  sodass auf den gespeicherten Daten eine fachliche Logik aufgebaut werden
  kann.
  \1 Die daraus resultierenden Analyseergebnisse sollten anschließend über im
  Projekt definierte Schnittstellen an verschiedene Systeme weitergeliefert
  werden können. Entsprechend soll das Datenhaltungssystem die dafür
  notwendigen Treiber und Schnittstellen mitliefern.
\end{outline}

Es soll in der Ziellösung eine stabile Schnittstelle im Gesamtsystem bilden
und im Falle von einem Ausfall der beiden anderen Softwarekomponenten die
Informationen weiterhin sicher und vollständig zur Verfügung stellen können.
Eine konkretisierte Anforderungsübersicht an das Datenhaltungssystem kann aus
Punkt~\ref{sec:datenhaltungssystem} entnommen werden.
\nl%

\subsection{Datenvisualisierung}
Die Weboberfläche muss ausgewählten Benutzern eine Übersicht über die
verbrauchten oder freien Ressourcen, wie zum Beispiel CPU-Auslastung oder
RAM-Auslastung einzelner virtueller oder physischer Maschinen bereitstellen.
Hierbei ist wichtig, dass die Sicherheit (Authentisierung, Authentifizierung
und Autorisierung) der Daten gewährleistet wird. Dies bedeutet, dass sowohl die
Verbindung zwischen Datenbank und Weboberfläche gesichert sein muss, als auch
die Weboberfläche selbst gegen Zugriff von Unbekannten. Die Weboberfläche
greift hier mittels API-Abfragen auf die Datenbank zu. Dies ermöglicht uns, den
Zugriff nur für die Datenvisualisierung auf die Datenbank gewährleisten zu
können. Die Weboberfläche soll Administratoren bei Neukonfiguration von
virtuellen Maschinen unterstützen und zeigt hier die Auslastung der einzelnen
physischen Systeme. Der Administrator kann so gezielt die einzelnen physischen
Maschinen besser auslasten und verteilen. Ebenfalls kann die
Datenvisualisierung eine Schnittstelle für komplexe Abfragen und Analysen
bereitstellen. Bei Bedarf können diese auch visuell ausgegeben werden.
\mr%

\chapter{Projektmanagement}
Die Liste an verfügbaren Managementmethoden ist lang. Methoden wie PRINCE2 oder
Lean Management kommen aus dem Bereich des Projektmanagements und sind seit
vielen Jahren auch im Bereich der Softwareentwicklung vertreten. Hinzu kommen
Vorgehensmodelle aus der Softwareentwicklung selbst, wie das Wasserfallmodell
oder das Spiralmodell. In den letzten Jahren gab es einen Wandel hin zu agiler
Entwicklung. Es hat sich gezeigt, dass sich in der schnelllebigen
Informationswelt Anforderungen an Software regelmäßig ändern, auch während der
Entwicklungs- und Planungsphase und nicht erst im späteren Betrieb. Außerdem
gibt es in den meisten Projekten unvorhergesehene Zwischenfälle. Dazu gehören
unter anderem:

\begin{outline}
  \1 Sicherheitslücken in verwendeten Bibliotheken werden entdeckt. Diese
  müssen oftmals aufwendig aktualisiert werden.
  \1 Die Zeiteinschätzung für die Implementierung von Funktionen oder die
  Behebung von Fehlern benötigt wesentlich mehr oder weniger Zeit als
  geschätzt.
  \1 Die Dokumentation einer Bibliothek, anhand dessen eine Zeitschätzung
  gemacht wurde, ist fehlerhaft. Die Bibliothek verhält sich anders als in der
  Dokumentation beschrieben und muss genauer begutachtet werden.
  \1 Die eingesetzte Software erfüllt ihren Zweck, benötigt aber viel zu viel
  Leistung. Hier muss nun die Software analysiert werden oder man migriert auf
  eine Alternative.
  \1 Bei der Verwendung mehrerer Hard- und Softwarekomponenten kommt es zu
  unerwarteten Inkompatibilitäten.
\end{outline}

Die Techniken und Vorgehensweisen der agilen Softwareentwicklung, versuchen dem
entgegen zu wirken. Sie zeichnen sich durch drei Merkmale aus:

\begin{outline}
  \1 Die Arbeit erfolgt iterativ und transparent
  \1 Der bürokratische Mehraufwand ist sehr gering
  \1 Die Methoden passen sich flexibel an das Projekt und an Änderungen an
\end{outline}

\section{Agile Softwareentwicklungsmethoden}
Die Firma Puppet entwickelt seit mehreren Jahren erfolgreich Software. Sie
steht dem Projektteam mit hilfreichen Tipps und Schulungen zur Seite. „Agile
Softwareentwicklung“ gilt als Oberbegriff für alle Techniken und
Vorgehensweisen in dem Bereich. Aus diesem kann man sich die Methoden
heraussuchen, die am besten zum Projekt passen. In den letzten Jahren haben
sich daraus die beiden Softwareentwicklungsmodelle Scrum und Kanban abgeleitet.
Diese nutzen Teile der agilen Softwareentwicklung.
\tm%

\subsection{Scrum}
Das Hauptaugenmerk von Scrum liegt auf den Sprints. Dies sind die Abschnitte in
denen gearbeitet wird. In der Regel beträgt ein Abschnitt zwei Wochen. Am
Anfang von jedem Arbeitstag des Abschnittes muss eine Besprechung erfolgen. Die
Besprechungen, „Daily Standup“ genannt, ist genau 15 Minuten lang und soll im
stehen abgehalten werden. Die Gesprächszeit muss gleichmäßig auf alle
Teilnehmer verteilt werden. Hierbei spielt es keine Rolle, wie groß das Team
ist. Bei jeder Teamstärke beträgt die Gesamtzeit nur 15 Minuten. Dies ist ein
großes Problem für Teams die sehr groß sind oder asynchron arbeiten. Aufgrund
der Vollzeitbeschäftigung aller Mitglieder in diesem Projekt, in Kombination
mit dem Abendschulunterricht, erfolgen viele Arbeiten am Projekt asynchron. Es
ist zeitlich nicht realisierbar, dass an jedem Tag jedes Teammitglied am
Projekt arbeitet. Die Durchführung von täglichen Besprechungen ist somit nicht
möglich. Scrum erlaubt es nicht, einzelne Elemente des Systems nicht zu nutzen
oder zu modifizieren. Somit kann in diesem Projekt nicht nach Scrum gearbeitet
werden (vgl.~\cite{scrum_talk}).
\tm%

\section{Agile Vorgehensweise im Projekt}
\label{sec:agile_vorgehensweise}
Die Projektzeit wird in zweiwöchige Abschnitte, Sprints genannt, eingeteilt. Am
Anfang jedes Sprints erfolgt ein Rückblick auf den vergangenen Sprint
(Retrospective). Es wird kurz besprochen was besonders positiv oder negativ
lief und ob die Projektmitglieder zufrieden sind. Bei Unstimmigkeiten im Team
oder bei negativen Ereignissen muss der Projektleiter sich dieser annehmen.
Seine Aufgabe ist es ein positives Arbeitsklima zu schaffen und nach
Möglichkeit wiederkehrende, negative Punkte zu beseitigen.

Wenn ein Mitglied einen besonderen Fortschritt im Projekt erzielt hat, wird
dieser kurz präsentiert (Review). Hierbei wird keine vollständige Präsentation
mit Folien erwartet, sondern eine kurze Erklärung der erledigten Arbeit mit
einer Demonstration. Jedes Projektmitglied darf selbstständig entscheiden, ob
es seine Arbeit hier präsentiert.

Review und Retrospective werden oft als „R \& R“ abgekürzt.

Am Ende jedes Sprints erfolgt die Planung für den kommenden Sprint (Planning).
Es wird überlegt, welche Arbeit erledigt werden muss, danach erfolgt die
ungefähre Schätzung des Zeitaufwandes. Für jede Aufgabe wird ein Ticket in der
Projektmanagementsoftware erstellt. Das Projektteam hat sich für „JIRA“
entschieden, da dies aktuell am verbreitetsten in der Wirtschaft ist. Neue
Featurewünsche werden als Userstory angelegt. Dies ist ein Ticket in dem aus
Sicht der anfragenden Person ihr Wunsch beschrieben ist. Für wichtige
Ereignisse werden Meilensteine festgelegt. Jeder darf neue Tickets zu jeder
Zeit anlegen. Diese landen dann im \gls{Backlog}. Dies ist der Sammelbegriff
für ausstehende Tickets, welche keinem Sprint zugeordnet sind. Es wird nicht an
Tickets gearbeitet, welche nicht im aktuellen Sprint sind.
\tm%

\chapter{Schnittstellen im Projekt}
Zwischen den Softwarekomponenten im Projekt werden Daten ausgetauscht. Dies
erfolgt über definierte Schnittstellen (auch Interfaces genannt). Im Folgenden
sind die einzelnen Schnittstellen zwischen den Komponenten und von den
Komponenten nach außen erklärt.
\tm%

\section{Datenerfassungssysteme}
Die Datenerfassungssysteme stellen zwei Schnittstellen zur Verfügung. Sie
kommunizieren mit dem lokalen System über eine
\glslink{Bidirektional}{bidirektionale} Schnittstelle. Die Datenerfassung wird
für die meisten Ressourcetypen über Polling realisiert. Hierzu fragt das
Datenerfassungssystem periodisch die Schnittstelle des Hosts nach neuen Daten.
Abfrageschnittstelle des Hosts wird durch den Linux-Kernel bereitgestellt, von
ihr kann ausschließlich gelesen werden. Die Erfassungssysteme arbeiten
zusätzlich eventbasiert. Hierbei schickt der Kernel die neuen Informationen
direkt zur Schnittstelle der Datenerfassungssysteme. Der Kernel ist der
kritischste Teil eines laufenden Computers. Er hat volle Rechte auf alle
Hardwareschnittstellen, weshalb die Interaktion mit ihm besonders geschützt und
minimal sein muss. Programmierfehler in der Vergangenheit erlaubten es
mehrfach, auf Schnittstellen auf den Kernel zu schreiben, obwohl das Interface
nicht beschreibbar sein sollte. Der Kernel selbst bietet nur lokale
Schnittstellen, aber über weitere Software werden diese indirekt im Netzwerk
bereitgestellt.

Die zweite Schnittstelle der Datenerfassungssysteme befindet sich innerhalb des
Projektes und interagiert mit der Datenbank. Diese Schnittstelle ist aus
Sicherheitsgründen \glslink{Unidirektional}{unidirektional}, um den Kernel zu
schützen. Die Erfassungssysteme können also nur Daten über das Netzwerk
verschicken, jedoch akzeptieren sie keine Anfragen, welche über die Netzwerke
eingehen.

Bei beiden Schnittstellen der Datenerfassungssysteme handelt es sich um
Binärschnittstellen. Hierbei werden Daten nicht in einer für Menschen lesbaren
Form übertragen, sondern in einem Binärprotokoll. Diese Protokolle sind
besonders effizient in der Datenübertragung und Datenverarbeitung. Es ist nicht
erforderlich, dass Menschen mit den beiden Schnittstellen kommunizieren.
\tm%

\section{Datenhaltungssystem / API}
Das Datenhaltungssystem, sowie die mit inbegriffene API, ist eine zentrale
Schnittstelle des Gesamtsystems. Im Gegensatz zu dem Datenerfassungssystem,
welches nur über die Informationen eines einzelnen physischen Servers verfügt,
besitzt das Datenhaltungssystem eine Gesamtsicht über alle angebundenen
Informationsquellen. Darunter fallen zunächst alle am System angebundenen
physischen Server, welche aktiv die Nutzungsdaten an das Datenhaltungssystem
liefern. Jedoch kann dies auf Wunsch des jeweiligen Kunden auch um weitere
Informationen erweitert werden, ein Beispiel wären die Kundendaten mit dem
jeweiligen Bestand. In der Ziellösung wird diese Art von Erweiterung jedoch
zunächst nicht beachtet und wurde auch nicht vom Auftraggeber angefordert,
sodass eine Anpassung in einem Folgeauftrag erledigt werden müsste.

Die API ist ebenfalls eine essentielle Schnittstelle des Gesamtsystems. Durch
die direkte Anbindung an das Datenhaltungssystem kann sie zeitkritische Daten
innerhalb kürzester Zeit an Drittsysteme liefern. Dabei arbeitet die API
oftmals auf einer Webapplikation, sodass diese von einem beliebigen Endgerät
erreicht werden kann. Zudem liefert sie die Daten in Form von Text aus, welcher
nach den Datenformaten des \gls{JSON} oder \gls{XML} aufgebaut ist. Die beiden
Datenformate sind die weitverbreitetsten Formate auf dem Markt und können in
vielen Frontend-Systemen ohne zusätzliche Entwicklungskosten angebunden
werden. Sollte ein Drittsystem diese Formate nicht unterstützen, so ist der
Implementierungsaufwand bei einem erfahrenen Entwickler ebenfalls gering, da
diese oftmals mit den Datenformaten bereits in Kontakt gekommen ist und keine
neuen Strukturen erlernen muss.
\nl%

\section{Datenvisualisierung}
Die Weboberfläche besitzt zwei Schnittstellen, welche aufgeteilt sind. Zum
einen die Abfrage der Daten an dem Datenhaltungssystem und zum anderen die der
Ausgabe und Anzeige für den User beziehungsweise den Administrator. Eine
Schnittstelle auf einem Webinterface dient in erster Linie der Anzeige von
Daten, die über ein anderes System (Backend) erhalten wurden.

Die erste Schnittstelle der Weboberfläche ist ein standardisierter Treiber,
welchen man verwenden kann, um auf ein Datenhaltungssystem zuzugreifen. Dieser
Liefert der Weboberfläche alle nötigen Funktionen und Werkzeuge zur
Datenextraktion, Datenmanipulation und der Rechteverwaltung auf einem
Datenhaltungssystem. In der Ziellösung von diesem Projekt ist ein rein
lesender Datenhaltungszugriff für die Weboberfläche vorgesehen. Dies wird durch
den Datenhaltungssystem-Administrator gewährleistet.

Neben der Verwendung eines Treibers in Verbindung mit einem
Berechtigungskonzept, kann auch ein API-System zum Einsatz kommen. Die
Vorteile, sowie die Rolle dieses Systems innerhalb dieses Projektes, können aus
Punkt~\ref{sec:schnittstelle_datenbereitstellung} entnommen werden.

Die zweite Schnittstelle dient zur Anzeige und Ausgabe der durch das
Datenhaltungssystem erhaltenen Daten. Ohne diese Schnittstelle würden Anwender,
welche mit der Weboberfläche arbeiten möchten, nur unstrukturierte Daten
erhalten und könnten diese nicht zur weiteren Analyse verwenden. Der Anwender
erhält somit einen direkten und schnellen Überblick in Form eines
Graphens~\ref{definition_eines_graphen} oder Diagramms über die gesamten
Datenressourcen der Infrastruktur. Diese können bei Bedarf vom Anwender oder
dem Administrator exportiert und als Abbild festgehalten werden. Hiermit können
anschließend Personen, welche keinen direkten Zugriff auf das System besitzen,
Analysen oder Statistiken durchführen. Die Übertragung der Daten zwischen den
Benutzern und dem Webserver erfolgt mit \gls{HTTPS}. Dies dient zum einem der
Verschlüsselung der Verbindung von Client zu Weboberfläche und zurück und zum
anderen damit die Weboberfläche selber authentifiziert werden kann.
\mr%

\chapter{Analyse von Softwarekomponenten}
\section{Datenerfassungssysteme}
Im Abschnitt~\ref{section:datenerfassungssysteme} wurde bereits auf die
einzelnen Ressourcetypen eingegangen, welche erfasst werden müssen. Um die
verschiedenen Datenerfassungssysteme im Hinblick auf die Ressourcetypen zu
evaluieren, muss zunächst geklärt werden was eine Metrik, ein Trend und eine
Zeitreihe (englischer Fachausdruck: Timeseries) ist.
\tm%

\subsection{Begriffsklärung und Anforderungen}
\label{section:Begriffserklärung}
Eine Metrik ist die Kombination aus:

\begin{outline}
  \1 Der Wert eines bestimmten Ressourcetypen zu einer bestimmten Zeit
  (Momentaufnahme).
  \1 Der Zeitstempel der Aufnahme
  \1 der Name des Ressourcetypen
\end{outline}

Eine Metrik wird periodisch als Ganzzahl ermittelt. Die Kombination mehrerer
Werte einer Metrik ergibt eine Timeseries (auch „Datenreihe“ oder „Abfolge“
genannt). Um Speicherplatz zu sparen, können die Werte aggregiert werden.
Hierbei werden für einen bestimmten Zeitraum einer Timeseries die Werte
genommen und folgende Funktionen angewendet:

\begin{outline}
  \1 \lstinline|min()|: Ermittlung des niedrigsten Wertes
  \1 \lstinline|max()|: Ermittlung des höchsten Wertes
  \1 \lstinline|count()|: Addieren aller Werte
  \1 \lstinline|sum()|: Alle Werte werden aufaddiert. Diese Berechnung ist
  nicht bei jedem Ressourcentyp sinnvoll, allerdings wird das Ergebnis für
  weitere Berechnungen benötigt
  \1 \lstinline|avg()|: Berechnung des Durchschnitts mit den Ergebnissen aus
  \lstinline|count()| und \lstinline|sum()|
  \1 \lstinline|95pct()|: Berechnung des Durchschnitts, zuvor werden die Werte
  nach Größe sortiert und die höchsten 5\% ignoriert
\end{outline}

Außerdem gibt es die Möglichkeit, die erhobenen Daten in Relation zur Zeit zu
setzen. Hierbei wird die Differenz zwischen zwei aufeinander folgenden
Messpunkten betrachtet.  Beispiel: Das Betriebssystem misst die geschriebene
Datenmenge auf einem Datenträger seit dem Startvorgang in Bytes. Interessant
für den Anwender ist aber der Datendurchsatz in \si{\mega\byte\per\second} oder
\si{\giga\byte\per\second}. Dies kann wie folgt berechnet werden:

Ab Systemstart wird in regelmäßigen Zeitabständen $\Delta t$ ausgelesen, welche
Datenmenge seit Systemstart geschrieben und gelesen wurde. Bezeichne $m_i$ den
$i$-ten Messwert. Dann kann die durchschnittliche Durchsatzrate $D_i$ zwischen
zwei Messpunkten $m_i$ und $m_{i-1}$ bestimmt werden durch:

\[ D_i = \frac{m_{i} - m_{i-1}}{\Delta t}.\]

Danach werden nur die Ergebnisse einer oder mehrerer Funktionen gespeichert und
die eigentlichen Daten verworfen. Das Aneinanderreihen mehrerer aggregierter
Werte bildet einen Trend (auch History Trend genannt). Trends werden oftmals
visualisiert. Hierbei lässt sich ressourcensparend ein langer Verlauf der
Metrik erkennen. Basierend auf vorhandenen Trends kann auch eine „Trend
Prediction“ erstellt werden. Hierbei werden Daten der Vergangenheit analysiert
und auf eine mögliche Zukunft vorausberechnet.

„Tiered Trends“ bezeichnet eine Datensammlung auf welche mehrfach die oben
genannten Funktionen angewendet wurden. Die benötigte Granularität ändert sich
oftmals mit dem Alter der Daten. Beispiel: In den ersten 24h benötigt man Daten
mit einer Auflösung von 30 Sekunden. In den darauffolgenden 2 Wochen reichen
Trends mit einer Auflösung von 5 Minuten und für die folgenden 3 Monaten reicht
eine Auflösung von 30 Minuten.

Hier nimmt man sich Daten die älter als 24 Stunden sind und teilt diese in
Einheiten von 5 Minuten. Hierauf werden die Funktionen angewendet, die
originalen Daten gelöscht und die Ergebnisse gespeichert. Parallel wird
regelmäßig nach bereits vorhandenen Trends geprüft, welche älter als 2 Wochen
sind. Diese werden wieder in Einheiten von 30 Minuten aufgeteilt und die
Funktionen erneut angewendet. Somit wurden „Tiered Trends“ gebildet.

Das Bilden von Trends kann schon während der Datenerfassung erfolgen, indem die
gesammelten Werte lokal zwischengespeichert werden und ausschließlich die
gebildeten Trends zur Datenhaltung geschickt werden. Dies ist besonders
effizient, da jeder Server nur eine geringe Menge an Trends zu berechnen hat.
Es ist jedoch erforderlich, dass der Client alle Daten zumindest so lange
vorhält, bis alle benötigten Trends gebildet sind. Im obigen Beispiel zu
„Tiered Trends“ sind dies 2 Wochen. Oftmals gibt es die Anforderung Trends ad
hoc zu bilden oder die Intervalle regelmäßig zu verändern. Hierzu muss eine
neue Konfiguration für die Datenerfassungskomponente auf jedem Server erstellt
werden, da diese keine automatisierte Schnittstelle bieten.

Ebenfalls kann die Generierung der Trends auf dem Datenbanksystem erfolgen.
Dies benötigt Rechenkapazitäten, bringt aber diverse Vorteile:

\begin{outline}
  \1 Trends können nicht nur über eine Metrik eines Servers erstellt werden,
  sondern auch über mehrere Server hinweg.
  \1 Trends können erstellt werden, wenn diese angefordert werden
  (Eventbasiert)
  \1 Trends können über verschiedenste Zeitrahmen erstellt werden. Außerdem
  lässt sich dieser einfacher ändern, da er nur zentral konfiguriert ist und
  nicht auf jedem Server,
\end{outline}

Aufgrund der Flexibilität der Trendgenerierung auf der Datenbank entschied sich
das Projektteam dazu, die Generierung von den Servern auf die Datenbank zu
verlagern. Die Fähigkeit Trends generieren zu können ist somit kein
Evaluationskriterium für Datenerfassungssysteme.

Im Folgenden werden die hier gelisteten Softwareprodukte evaluiert. Die Liste
basiert auf dem Lasten- und Pflichtenheft:

\begin{outline}
  \1 Coreutils
  \1 atop
  \1 collectd
  \1 zabbix-agent
  \1 python-diamond
  \1 sysstat
  \1 Logstash
  \1 Riemann
\end{outline}
\tm%

\subsection{Coreutils}
Coreutils ist eine Sammlung von Standardwerkzeugen auf jedem
Linux-Betriebssystem. Sie bieten einfache Möglichkeiten zur Editierung von
Dateien. Coreutils ist mit der GPL3 lizenziert (vgl.~\cite{coreutils}) und
zählt somit als Open-Source-Software. Verwaltet wird die Programmsammlung
aktuell von der Free Software Foundation. Der Linux Kernel stellt Informationen
über schreibgeschützte Dateien bereit. In Kombination mit der Programmsammlung
procps kann auf die Informationen des Kernels zugegriffen werden, um Metriken
zu ermitteln (vgl.~\cite{procps}). Über das Programm \texttt{free} oder die
Datei \texttt{/proc/meminfo} erhält man dann die Menge des vorhandenen sowie
zur Zeit benutzten Arbeitsspeichers. Die Anzahl der Prozessorkerne liefert das
Programm nproc oder die Datei \texttt{/proc/cpuinfo}. \texttt{/proc/diskstats}
liefert Informationen über die Anzahl der Schreib- und Lesevorgänge auf allen
Datenträgern. Außerdem listet die Datei die geschriebene als auch gelesene
Datenmenge. Die Informationen beziehen sich auf die Zeit seit dem letzten
Startvorgang des Servers bis zum Ausgeben der Datei. Ein kontinuierliches
Ausgeben ermöglicht die Berechnung des Durchsatzes pro Sekunde. Diese Aufgabe
wird von der Datenbank übernommen und ist näher im
Abschnitt~\ref{section:Begriffserklärung} erklärt. Das gleiche gilt für die
Datei \texttt{/proc/net/dev}. Sie listet alle Netzwerkadapter des Systems und
die übertragene Datenmenge (Senden und Empfangen) pro Adapter. Mit Coreutils
und Linux Bordmitteln ist es möglich alle benötigten Informationen lokal
auszulesen. Die Aufbereitung der Daten muss allerdings selbst erledigt werden.
Hier ein Beispiel, um alle 30 Sekunden die gesendeten sowie empfangenen Daten
des Netzwerkadapters \texttt{enp1s0} in Bytes auszulesen:

\begin{listing}
  \inputminted{bash}{../listings/coreutils-awk.txt}
  \caption{/proc mit awk parsen}
  \label{lst:aw}
\end{listing}

Dies erzeugt eine \gls{CSV} ähnliche Datenstruktur:

\begin{center}
\begin{minted}{text}
device,receive,transmit
enp1s0,28932776123,1786084537
enp1s0,28935220128,1786163375
enp1s0,28939447268,1786302393
enp1s0,28942260029,1786392663
enp1s0,28946985781,1786547807
enp1s0,28949168552,1786622616
enp1s0,28951976308,1786708990
\end{minted}
\captionof{listing}{traffic stats enp1s0}
\end{center}

Für das Auslesen all dieser Daten werden kleine Skripte wie das im
Figure~\ref{lst:aw} benötigt. Um die Daten über das Netzwerk zu verschicken,
werden weitere Skripte benötigt. Diese Sammlung an Skripten kann auf einem
Server ausgeführt werden, um seine lokalen Statistiken zu erfassen oder direkt
in virtuellen Maschinen.\ libvirt, eine Bibliothek zum Verwalten virtueller
Maschinen, stellt auf einem Hostsystem Informationen über die lokalen Instanzen
bereit.\ libvirt kümmert sich hier um die Interaktion mit diversen Hypervisors
und bietet eine einheitliche Schnittstelle an. Dies ermöglicht es Coreutils,
vom Host aus Daten der VMs zu ermitteln.
\tm%

\subsection{atop}
Das Linux-Kommandozeilenprogramm atop ist zum Visualisieren von Prozessen und
des Systemzustands. Es ist unter der GPL lizenziert und fällt damit unter
Open-Source-Software. Der aktuelle \gls{Maintainer} ist Gerlof Langeveld.  Die
erste stabile Veröffentlichung war im Dezember 2001. Seitdem wird das Programm
kontinuierlich entwickelt. Es ist darauf ausgelegt, im Vollbildmodus eines
Terminals zu arbeiten (siehe~\ref{figure:atop1} und~\ref{figure:atop2} für
Screenshots). Neben der Echtzeitanalyse bietet atop aber auch die Möglichkeit,
im Hintergrund periodisch eine Logdatei zu schreiben. Hier wird ein selbst
entwickeltes Binärformat genutzt. Dies kann von atop aber auch zurück in ASCII
umgewandelt werden. Dann erhält man alle Infos, die auch die interaktive
Version anzeigt. Im Standardverhalten werden alle 10 Minuten Logs geschrieben,
dieses Intervall ist allerdings modifizierbar. Das ASCII Format beruht nicht
auf bekannten Standards und lässt sich deshalb nicht automatisiert
weiterverarbeiten (siehe~\ref{lst:atop} für eine exemplarische Ausgabe). Die
Daten müssen, ähnlich wie bei Coreutils, vor einer Speicherung in einer
zentralen Datenbank erst noch bearbeitet werden. Es ist keine Funktionalität
eingebaut, um gesammelte Daten über das Netzwerk auszutauschen, dies muss
ebenfalls nachgerüstet werden. Es ist keine Funktionalität vorhanden, um Daten
auf einem Host von virtuellen Maschinen zu ermitteln. Außerdem kann die
Software nicht erweitert werden (vgl~\cite{atop}).
\tm%

\subsection{collectd}
Das Projekt collectd wurde von Florian Forster gestartet. Die erste stabile
Version veröffentlichte er im Juli 2005. Seitdem erfolgt eine stetige
Entwicklung mit mehreren neuen Veröffentlichungen pro Jahr.\ collectd bietet
eine Vielzahl von Möglichkeiten, um Daten auf einem System zu ermitteln und
diese in diversen Formen über das Netzwerk auszutauschen oder lokal abzulegen.
Die Software ist in C geschrieben.

\begin{center}
    \inputminted{text}{../listings/collectd-clone.txt}
    \captionof{listing}{collectd git clone}
\end{center}

Wie im obigen Listing zu sehen, haben 415 Leute 9059 Änderungen im
Versionsverwaltungssystem durchgeführt.\ collectd steht unter der MIT Lizenz
und zählt somit als Open-Source-Software. Es kann flexibel mit Plugins
erweitert werden. Hierbei wird zwischen mehreren Kategorien
unterschieden (vgl.~\cite{collectd_plugins}):

\begin{outline}
  \1 Read, Plugins, die Daten aus zusätzlichen Quellen lesen
  \1 Write, Plugins, die erfasste Daten in einem bestimmten Format lokal
  speichern
  \1 Binding, ein Metaplugin, welches den Interpreter bestimmter Skriptsprachen
  einbindet. Dies erlaubt es, eigene Plugins in beliebigen Skriptsprachen zu
  schreiben
  \1 Logging, erlaubt es, collectd interne Fehler und Warnungen in verschiedenen
  Formen zu speichern und auszugeben
\end{outline}

Das Sammeln aller benötigter Daten wird von collectd unterstützt. Es kann
lokale Daten auf einem Hostsystem oder in einer virtuellen Maschine ermitteln.
Außerdem kann es mit Hilfe des \texttt{Virt} Plugins auf einem Host auch alle
notwendigen Informationen über virtuelle Instanzen
abrufen(vgl.~\cite{collectd_virt_plugins}).
\tm%

\subsection{zabbix-agent}
Die Firma Zabbix LLC wurde 2005 gegründet. Sie pflegt die Open-Source-Produkte
zabbix-agent (im folgenden Abschnitt Agent genannt), zabbix-server und
zabbix-frontend (siehe auch Abschnitt~\ref{subsubsec:zabbix-frontend} für die
Analyse des Frontends). Alle Produkte sind unter der GPL2 lizenziert. Der Agent
ist in C geschrieben. Er bietet die Möglichkeit, diverse Daten direkt aus dem
lokalen System auszulesen und an den zabbix-server zu senden. Das Webinterface
ist eine PHP-Applikation, welche die Daten visualisiert und direkt aus der
Datenbank liest. Der Agent kann mit externen Skripten erweitert werden. In
Kombination mit Coreutils kann er alle benötigten Informationen auslesen. Die
Community bietet dafür fertige Skripte an (vgl.~\cite{zabbix_virt_plugins}).

Der Agent benutzt ein eigenes Binärprotokoll zur Kommunikation. Dieses wird
nicht von anderen bekannten Programmen unterstützt. Deshalb muss der Agent
zwangsläufig mit dem hauseigenen zabbix-server genutzt werden. Es ist nicht
möglich den Server, beziehungsweise die unterstützten Datenbanken, redundant
auszulegen. Dies ist eine Anforderung für die Datenhaltungssysteme. Somit ist
der zabbix-agent nicht im Projekt nutzbar (vgl.~\cite{zabbix_architecture}).
\tm%

\subsection{python-diamond}
Python-diamond (auch Diamond genannt) ist ein in Python geschriebener Dienst.
Er kann auf Linux Systemen diverse Daten erheben und diese über das Netzwerk
austauschen. Die erste Veröffentlichung ist vom Juli 2012. Seitdem gibt es
unregelmäßig neue Updates. Das letzte Release ist vom November 2016. Die
Open-Source-Software steht unter der MIT Lizenz (vgl.~\cite{python-diamond}).

\begin{center}
    \inputminted{text}{../listings/diamond-clone.txt}
    \captionof{listing}{Diamond git clone}
\end{center}

Wie im obigen Listing zu sehen gibt es 2922 Beiträge im Versionskontrollsystem
von 292 verschiedenen Personen. Diamond kann alle benötigten Daten lokal und
vom Hypervisor über seine VMs ermitteln. Über das Netzwerk werden diese im
\glslink{Carbon}{Graphite Format} geschickt. Über Plugins sind auch weitere
Formate möglich. Der \gls{Backlog} bei Diamond ist sehr hoch. Die Software hat
aktuell 88 offene Issues im Versionsverwaltungssystem. Dies ist eine Mischung
aus gemeldeten Fehlern und Feature Requests. Außerdem gibt es 139 offene Pull
Requests (siehe Listing~\ref{figure:diamond}. Dies sind Änderungen am
Quellcode, welche die Community erstellt hat, aber noch nicht eingepflegt
worden sind.
\tm%

\subsection{sysstat}
Die Softwaresammlung sysstat ist eine Kollektion an kleinen Programmen zur
Performance-Analyse unter Linux. Die Sammlung steht unter der GPL2 Lizenz. Das
älteste verfügbare Release ist die Version 5.0.5 vom Juni 2004. Seitdem
erscheinen jedes Jahr mehrere Updates. Entwickelt wird das Projekt
von (vgl.~\cite{systat_releases}). Die Liste der Features von sysstat ist sehr
umfangreich.  Neben den geforderten Kriterien zur lokalen Datenerfassung
liefert es auch viele weitere Details. Hierzu zählen zum Beispiel Statistiken
zum Verhalten des Arbeitsspeichers. Die Interrupt-Verarbeitung des Prozessors
und detaillierte Netzwerkinformationen. Ermittelte Daten können als \gls{CSV},
\gls{XML} oder \gls{JSON} gespeichert werden (vgl.~\cite{sysstat_features}).

Die Software sysstat bietet keine Möglichkeit, um Daten von virtuellen
Maschinen über den Host zu ermitteln. Die bereitgestellten Informationen sind
aber in einer sehr guten Struktur dank des XML-Schemas. Das Einbinden in andere
Programme und das Interpretieren ist somit sehr einfach. Es werden nicht alle
Anforderungen von sysstat erfüllt, es kann aber eventuell als Ergänzung zu
anderen Programmen genutzt werden.
\tm%

\subsection{Logstash}
\label{subsec:logstash}
Logstash wird von der Firma Elasticsearch BV entwickelt. Zusammen mit
Elasticsearch (siehe auch Abschnitt~\ref{subsubsec:elasticsearch}) und
\gls{Kibana} bilden sie den ELK-Stack. Die Open-Source-Software steht unter der
Apache License. Logstash arbeitet nach dem modularen \gls{EVA} Prinzip.

\begin{outline}
  \1 Die gewünschten Daten werden in vordefinierten Formaten eingelesen.
  \1 Die Daten werden transformiert und gefiltert.
  \1 Das Ergebnis wird zu einem oder mehreren Endpunkten geschickt.
\end{outline}

Alle drei Schritte können mit über 200 Plugins erweitert
werden (vgl.~\cite{logstash_overview}). Logstash ist für die Verarbeitung von
Ereignissen ausgelegt. Diese können direkt aus diversen Quellen gelesen werden.
Hierzu gehören unter anderem Netzwerksockets, Textdateien, Syslog Streams und
Message queues. Bei der Transformation werden die Daten neu serialisiert und in
ein anderes Format übertragen. Hier können zum Beispiel zeilenweise eingelesene
Ereignisse aus einer Textdatei in das \gls{JSON} Format überführt werden. Dabei
werden die unstrukturierten Daten mit Filtern zerlegt und in Typen eingeteilt,
um strukturierte Daten zu erhalten. Aus einem langen String wird dabei die
eigentliche Nachricht, das Datum, Dringlichkeit und andere Typen erkannt.
Bereits strukturierte Daten können auch einfach in eine andere Struktur
überführt werden. Zum Abschluss werden die Daten ausgegeben. Dies kann
passieren, indem unter anderem in einen Cache Service, eine Message Queue oder
direkt in eine Datenbank geschrieben wird.

Logstash ist nicht auf das Einlesen von Metriken ausgelegt (Ganzzahlen),
sondern für Logs in Textform. Über ein eigenes Plugin ließe sich aber auch das
effiziente Einlesen von Metriken realisieren. Mit der Vielzahl an
Transformationen und Ausgabemöglichkeiten ist Logstash sehr flexibel und passt
sich an jede Situation an. Dadurch dass Daten über mehrere Ausgänge parallel
verarbeitet werden können, ist die Software sehr zukunftssicher. Bei geänderten
Anforderungen lassen sich die Daten zum Beispiel anstatt in eine SQL-basierte
Datenbank in eine No-SQL-Datenbank schreiben.
\tm%

\subsection{Riemann}
Riemann ist ein junges Softwareprojekt. Das erste Release erfolgte im März
2012. Die Open-Source-Software steht unter der Eclipse Public Lizenz Version 1.
Die aktuellste Veröffentlichung trägt die Versionsnummer 0.2.12 und wurde im
Dezember 2016 veröffentlicht. Das Versionsschema von Riemann hält sich an die
Regeln von \gls{Semantic Versioning}, somit sind erst dann Releases zum
produktiven Einsatz empfohlen, wenn sie die Nummer 1.0.0 oder höher enthalten.
Der Autor ist Kyle Kingsbury, welcher auch \gls{Jepsen} geschrieben hat. Der
Fokus der Software liegt auf der Verarbeitung von Ereignissen, auch Events
genannt (vgl.~\cite{riemann_concepts}). Dazu wurde eine eigene Struktur
entworfen mit mehreren Feldern (Siehe Tabelle~\ref{tbl:riemann}). In der
Konfigurationsdatei werden Streams definiert. Diese laufen immer nach dem
gleichen Schema ab:

\begin{outline}
  \lstset{language=Clojure}
  \1 Initialisierung eines neuen Streams mit dem Schlüsselwort
  \lstinline|streams|
  \1 Ein Filter mit dem Schlüsselwort \lstinline|where|, hier kann mit beliebig
  vielen Filterregeln auf alle eingehenden Events geprüft werden. Mehrere
  Regeln können nach den Definitionen der booleschen Algebra verknüpft werden
  (\lstinline|and| im folgenden Beispiel).
  \1 Zum Schluss wird eine Aktion aufgelistet. Das Schlüsselwort ist immer der
  Name der Aktion (\lstinline|email| im folgenden Beispiel). Diese wird auf
  jedes Event ausgeführt, welches durch alle Filter eines Streams gekommen ist.
  Jeder Aktion können Optionen mitgegeben werden.
\end{outline}

\begin{listing}
  \inputminted{clojure}{../listings/riemann-config.txt}
  \caption{Einfache Riemann Konfiguration}
\end{listing}

Im obigen Beispiel wird die Temperatur des ersten Prozessorkerns überprüft.
Sobald diese bei 75°C oder höher liegt, wird eine E-Mail an tim@bastelfreak.de
geschickt.

Riemann besitzt eine lange Liste an vorhandenen Aktionen, die man nutzen kann.
Dazu gehört auch der Export von Events in diverse Datenbanken. Dazu gehören
unter anderem Elasticsearch, graphite, influxdb, kariosdb, opentsdb, prometheus
und generische UDP und TCP Outputs.

Während der Analyse von Riemann hat sich herausgestellt, dass Riemann zwar
besonders gut Daten verarbeiten kann, allerdings keine Möglichkeit mitbringt,
diese selbst zu ermitteln.
\tm%

\subsection{Zusammenfassung}
Unten aufgeführt ist eine Gegenüberstellung der verschiedenen Lösungen und
deren Möglichkeiten. Verglichen wird hier die Fähigkeit der Datenermittlung für
die im Abschnitt~\ref{section:datenerfassungssysteme} definierten Typen.
Betrachtet wird zuerst nur die Ermittlung von lokalen Daten in einer virtuellen
Maschine oder auf einem Hostsystem.

\begin{center}
\begin{tabular}{lcccc}
  \toprule
  Programme      & Zugewiesener RAM & CPU Kerne & Datenträger & Netzwerk \\
  \midrule
  Coreutils      & \cmark{}         & \cmark{}  & \cmark{}    & \cmark{} \\
  atop           & \cmark{}         & \cmark{}  & \cmark{}    & \cmark{} \\
  zabbix-agent   & \cmark{}         & \cmark{}  & \cmark{}    & \cmark{} \\
  collectd       & \cmark{}         & \cmark{}  & \cmark{}    & \cmark{} \\
  python-diamond & \cmark{}         & \cmark{}  & \cmark{}    & \cmark{} \\
  sysstat        & \cmark{}         & \cmark{}  & \cmark{}    & \cmark{} \\
  Logstash       & \~{}             & \~{}      & \~{}        & \~{}     \\
  Riemann        & \xmark{}         & \xmark{}  & \xmark{}    & \xmark{} \\
  \bottomrule
\end{tabular}
\captionof{table}{Ermittlung von lokalen Daten}
\end{center}

In der nachfolgenden Tabelle wird verglichen, welche Datenerfassungssysteme auf
einem Hypervisor Daten über virtuelle Maschinen erfassen können, ohne in eben
diesen selbst zu laufen:

\begin{center}
\begin{tabular}{lcccc}
  \toprule
  Programme      & Zugewiesener RAM & CPU Kerne & Datenträger & Netzwerk \\
  \midrule
  Coreutils      & \cmark{}         & \cmark{}  & \cmark{}    & \cmark{} \\
  atop           & \xmark{}         & \xmark{}  & \xmark{}    & \xmark{} \\
  zabbix-agent   & \cmark{}         & \cmark{}  & \cmark{}    & \cmark{} \\
  collectd       & \cmark{}         & \cmark{}  & \cmark{}    & \cmark{} \\
  python-diamond & \cmark{}         & \cmark{}  & \cmark{}    & \cmark{} \\
  sysstat        & \xmark{}         & \xmark{}  & \xmark{}    & \xmark{} \\
  Logstash       & \xmark{}         & \xmark{}  & \xmark{}    & \xmark{} \\
  Riemann        & \xmark{}         & \xmark{}  & \xmark{}    & \xmark{} \\
  \bottomrule
\end{tabular}
\captionof{table}{Ermittlung von VM Daten}
\end{center}

Atop erfüllt die Anforderungen nicht und fällt somit aus der weiteren
Betrachtung raus. Coreutils ermöglicht zwar das Ermitteln aller benötigten
Daten, dies ist allerdings mit einem sehr großen Aufwand verbunden, da Daten
von Hand aufbereitet werden müssen. Außerdem muss Netzwerkfunktionalität zum
Versenden der Daten nachgerüstet werden. Somit fällt Coreutils auch raus.\
Sysstat ist nicht in der Lage, Daten von virtuellen Maschinen zu ermitteln und
fällt somit auch raus. Logstash ist äußerst flexibel, benötigt aber etwas Arbeit
bei der Entwicklung eines eigenen Eingabe-Plugins.

Übrig bleiben collectd und python-diamond. Beide erfüllen alle Anforderungen.\
Collectd ist in C geschrieben, dies bietet gegenüber Python im Diamond Projekt
einen Geschwindigkeitsvorteil. Das Team rund um collectd arbeitet schon länger
am Projekt und arbeitet kontinuierlich daran. Im weiteren Verlauf des Projekts
wird deshalb mit collectd gearbeitet. Sollten sich hier unerwartete Probleme
ergeben, kann auf Diamond gewechselt werden.
\tm%

\section{Datenhaltungssystem}
\label{sec:datenhaltungssystem}
Zu Beginn des Projektes musste das Projektteam mögliche Datenhaltungssysteme in
einer Liste für eine darauffolgende Evaluierung definieren. Bei der Erstellung
dieser Liste wurden Systeme aufgenommen, welche den Projektmitgliedern aus
vergangenen Projekten, beziehungsweise aus dem täglichen Arbeitstag bereits
bekannt waren. Zusätzlich hatten die Projektmitglieder vorab mit Experten aus
dem eigenen Unternehmen und aus dem Unternehmen des Auftraggebers über mögliche
weitere Systeme gesprochen. Dadurch das Herr Luis in dem Bereich der
Massendatenhaltungssysteme arbeitet, sind ebenfalls Systeme aufgenommen worden,
welche im Bereich der sogenannten Big-Data-Technologie arbeiten.

Das Hinzufügen von weiteren/neuen Datenbanksystemen zur späteren
Evaluierung erfolgt nur nach einer Genehmigung vom Auftraggeber.

Die in der Liste aufgenommen Systeme sind folgende:

\begin{outline}
  \1 elasticsearch
  \1 cassandra
  \1 Postgres
  \1 OpenTSDB
  \1 KNIME
  \1 impala
  \1 hadoop
  \1 hive
\end{outline}
\nl%

\subsection{Vorbereitung der Evaluierung}
\label{subsec:DBS_vorbereitung_der_evaluierung}
Als Vorbereitung für die Evaluation der Datenhaltungssysteme wurden mehrere
Kriterien definiert, welche einen Leitpfaden für die spätere Arbeit geben.
Während der Auswahl und Definition, der von dem Datenbanksystem zu erfüllenden
Kriterien, wurde darauf geachtet, dass diese sich aus dem späteren Zielsystem
ableiten lassen und ebenfalls realistisch erfüllbar sind. Anschließend wurde
die Relevanz jedes Kriteriums von den Teammitgliedern beurteilt und festgelegt.

Die daraus resultierten Kriterien sind nun nach Relevanz absteigend
aufgelistet:
\begin{outline}
  \1 Bereits vorhandenes Wissen über die Datenbank. Dazu gehört, dass diese
  eine umfangreiche Dokumentation besitzt und eine aktive und große Community
  für Diskussionen und Fragen vorhanden ist. Ebenfalls fallen die vorhandenen
  Vorkenntnisse über das Datenbanksystem der Projektmitglieder, sowie die
  Mitarbeiter von dem Unternehmen des Auftraggebers unter dieses Kriterium. Es
  ist wichtig, dass die aufzuwendende Einarbeitungszeit während des Projekts
  und im späteren Wirkbetrieb bei Mitarbeitern des Unternehmens so gering wie
  möglich gehalten wird.
  \1 Das Datenhaltungssystem muss einfach, schnell und jederzeit erweiterbar
  sein. Dies bedeutet, dass Ressourcen (CPU/RAM/Speicherplatz) des
  Datenbanksystems im optimalen Fall während des normalen Betriebs (kein
  Neustart oder Auszeit) aufgestockt werden kann. Es ist jedoch nach
  Anforderung des Auftraggebers ausreichend, wenn eine Aufstockung nach einer
  Auszeit von maximal 45 Minuten erfolgt ist.
  \1 Umweltschonende und effiziente Datenhaltung ist zu beachten. Dies
  definiert eine optimale Relation zwischen der verwendeten Hardware und dem
  daraus resultierenden Ergebnis. Eine umweltschonende Datenhaltung kann
  unabhängig der korrekten Auswahl der genutzten Stromeffizienzklasse jedes
  Hardwarebauteils bereits bei der im Produktivbetrieb zu benutzende Software
  beeinflusst werden. Dies ist im Bereich der Datenhaltungs-Software die
  entstehende Speicherplatzgröße für einen einzelnen Datensatz in einer
  Datenbank. Um so kleiner dieser ist, desto mehr kann bei Energiekosten für
  zusätzlichen Speicherplatz eingespart werden.

  Unter Berücksichtigung der genannten Attribute soll jedoch die
  Datenhaltungs-Software mit der gleichen Hardware-Ressource das beste und
  schnellste Ergebnis erbringen können. Dies bedeutet, dass auch bei komplexen
  Analysetechniken und zusätzlich hohen Datenmengen das System weiterhin
  gegenüber anderen Datenhaltungssystemen ein valides und schnelles
  Analyseergebnis erbringen kann.
  \1 Die Datenhaltungs-Software sollte bereits Methoden zur Datensicherung
  besitzen. Dabei wird unterschieden in:
    \2 Von extern getriggerte Methoden, um Backups zu erstellen.
    \2 Intern genutzte Methoden, um im Fehlerfall (zum Beispiel ein Defekt am
    Netzwerkkabel oder Ausfall eines Servers im Cluster) einem Datenverlust
    vorzubeugen.
\end{outline}
\nl%

\subsection{Durchführung der Evaluierung}
\label{subsec:durchfuehrung_der_evaluierung}
Bei der Durchführung der Evaluation von den ausgewählten Datenhaltungssystemen
aus der Liste in Punkt~\ref{sec:datenhaltungssystem} wurden die
Evaluationskriterien aus Punkt~\ref{subsec:DBS_vorbereitung_der_evaluierung}
verwendet. Diese dienten als Leitpfaden für jedes Datenhaltungssystem und
wurden von jedem Projektmitglied beachtet.

Zu Beginn der Durchführung teilte Herr Meusel als Projektleiter die zu
evaluierenden Datenhaltungssysteme jedem Projektmitglied zu, um eine schnellere
Bearbeitung zu gewährleisten. Bei der Einteilung wurden die bereits vorhandenen
Erfahrungen jeder Projektmitglieder zu dem jeweiligen Datenhaltungssystem
berücksichtigt. Im zweiten Schritt sollte sich anschließend jedes
Projektmitglied mit dem zugeteilten Datenhaltungssystem kurz auseinandersetzen.
In diesem Schritt ist Herrn Luis aufgefallen, dass es sich bei der in der Liste
aufgenommenen Software „KNIME“ nicht um ein Datenhaltungssystem handelt,
sondern um ein Werkzeug zur Datenanalyse, Datenaufbereitung und
Datendarstellung.

Die daraus resultierenden Ergebnisse wurden Herrn Luis anschließend von den
Projektmitgliedern bereitgestellt, sodass dieser einen Überblick im
Projektbereich „Datenhaltung“ erschaffen konnte. Die resultierenden
Ergebnisse und Erkenntnisse sind in dem nachfolgenden Punkt für jede
Datenhaltungs-Software dokumentiert.
\nl%

\subsubsection{Elasticsearch}
\label{subsubsec:elasticsearch}
Elasticsearch ist eine in Java geschriebene, verteilte Suchmaschine mit einer
\gls{RESTful} Schnittstelle. Die erste Veröffentlichung gab es am 8. Februar
2010~\cite{es_release}. Elasticsearch speichert Daten im \gls{JSON} Format ab.
Die komplette Interaktion mit der Suchmaschine erfolgt über
\glslink{RESTful}{REST}, hierzu zählt nicht nur das eigentliche Suchen, sondern
auch die Administration der Software selbst und das Eintragen neuer Daten.
Elasticsearch beschreibt sich selbst als eine dokumentenbasierte No-SQL
Datenbank. Dies bedeutet, das es keine Tabellen mit Relationen gibt. Jeder
Datensatz ist ein Dokument, welches als \gls{JSON} gespeichert wird. Ein
Dokument basiert aus Key->Value Paaren, diese können auch geschachtelt sein.
Elasticsearch erzeugt mit Hilfe der \gls{Lucene} Bibliothek Indexe auf diesen
Daten, anschließend ist dann eine Volltextsuche möglich. Die Firma
Elasticsearch BV aus den Niederlanden finanziert die Entwicklung der
Software. Als Lizenz hat sich die Firma für die Apache License entschieden.

\begin{center}
    \inputminted{text}{../listings/elasticsearch-clone.txt}
    \captionof{listing}{Diamond git clone}
\end{center}

Oben befindet sich eine kurze Analyse des \gls{Git}-
\glslink{Repository}{Repositorys} vom 27.11.2016. Die Suchmaschine besteht
hier aus 25989 Beiträgen von 869 verschiedenen Personen.

Elasticsearch erlaubt es, als verteiltes Cluster betrieben zu werden. Hierfür
werden mindestens zwei Server benötigt. Eingehende Daten werden in Indexe
aufgeteilt, dies ist eine logische Gruppierung basierend auf der Charakteristik
der Daten. Zum Beispiel kann man alle gemessenen Metriken charakterisieren
anhand des Quellservers auf dem sie ermittelt wurden oder auch anhand des Typs
(Festplattenauslastung, CPU-Auslastung). Jeder Index kann in einzelne
Subgruppen aufgeteilt werden (Shards genannt). Jedes Shard kann auf einem
anderen Server gespeichert werden. Das Schreiben in die Shards, sowie Lesen aus
den Shards, kann parallel auf allen Servern erfolgen. Somit erreicht man eine
horizontale \gls{Skalierung}. Es ist auch möglich, Replikas von Shards zu
erzeugen. Hierbei wird eine Kopie eines Shards auf einem anderen Server im
Cluster gespeichert. Bei jeder Änderung des echten Shards wird auch das Replika
geändert. Sobald ein Server im Cluster ausfällt, kann aus dem Replika ein
aktives Shard gemacht werden. Die Anzahl der Replikas pro Shard und deren
Verteilung kann im laufenden Betrieb geändert werden. Somit erreicht man ein
hochverfügbares Setup, welches im laufenden Betrieb um neue Server erweitert
werden kann (vgl.~\cite{es_concepts}).

Mit \gls{Jepsen} wurde Elasticsearch mehrfach untersucht
(vgl.~\cite{es_jepsen_all}). Nach dem ersten Test im Jahr 2014 hat
Elasticsearch BV eine Übersichtsseite erstellt mit allen offenen Problemen und
Szenarien, die zu Datenverlust führen können (vgl.~\cite{es_resiliency}). Die
wenigsten Hersteller und Communitys gehen so offen mit Problemen um.

Der letzte Jepsen-Test zeigt leider, dass Elasticsearch immer noch Probleme bei
Netzwerkausfällen und starker Systemlast hat (vgl.~\cite{jepsen_elastic}).
Aufgrund der Größe solcher Setups ist es üblich, dass jeder Node im Cluster
über zwei Netzwerkverbindungen verfügt. Eine davon ist zur internen
Cluster-Kommunikation und zum Verteilen der Replikas, die andere für die
Kommunikation mit der Außenwelt (Bereitstellung der RESTful-Schnittstelle).

Gegeben ist ein Teilausfall des internen Netzwerks (zum Beispiel ein
Kabelbruch):

\begin{outline}
  \1 Die Nodes merken, dass einer von ihnen nicht erreichbar ist
  \1 Replikas der fehlenden Shards werden aktiv gesetzt
  \1 Das Cluster nimmt weiter Daten an, sofern Replikas zur Verfügung stehen
  \1 Elasticsearch meldet, dass die neuen Daten erfolgreich geschrieben wurden
  \1 Die neuen Daten wurden nicht geschrieben und sind verloren
\end{outline}

Dieses Verhalten ist mit dem Jepsen-Test reproduzierbar
(vgl.~\cite{es_jepsen_iso}). Während Elasticsearch Replikas von Shards eines
ausgefallenen Nodes aktiv setzt, werden neue Daten weiterhin über die
RESTful-Schnittstelle angenommen und dann ohne Rückmeldung verworfen.

Das nächste Problem ist das Verhalten von Elasticsearch unter hoher Systemlast.
Ein Prozess unter Linux kann pausiert und wieder gestartet werden.
Währenddessen kann ein Node deaktiviert werden. Hierfür gibt es verschiedene
Gründe, zum Beispiel eine defekte Netzwerkverbindung oder eine absichtliche
Abschaltung durch den Administrator, weil er Wartungsarbeiten durchführen
möchte.

\begin{outline}
  \1 Elasticsearch Prozess wird pausiert
  \1 Das Cluster detektiert ein Netzwerkproblem zwischen dem Cluster und dem
  pausierten Node
  \1 Der pausierte Node wird im Cluster deaktiviert
  \1 Der Prozess wird wieder gestartet
  \1 Der Prozess prüft nicht, ob er noch aktiv ist für seine Shards
  \1 Der Prozess akzeptiert eingehende Schreibanfragen über die
  RESTful-Schnittstelle und updated seine Shards
  \1 Der Prozess meldet zurück, dass er die Daten erfolgreich geschrieben hat
  \1 Sobald das Netzwerkproblem behoben ist, erhält der Node Anweisungen aus
  dem Cluster, um seine Shards auf den Stand der Kopien im Netzwerk zu bringen
  und verwirft, was er selbst geschrieben hat
\end{outline}

Das Pausieren passiert unter zwei Umständen. Der \gls{Garbage Collector} macht
dies in regelmäßigen Abständen, um den belegten Arbeitsspeicher nach nicht mehr
benötigten Objekten zu scannen und den Speicher dann freizugeben. Der Garbage
Collector pausiert nur für wenige Millisekunden, ein Datenverlust ist hier
aufgrund des Zeitfensters unwahrscheinlich. Auch der Linux Kernel pausiert
Prozesse, nämlich wenn ein System unter sehr hoher Last ist. Wenn die Hardware
am Rand der Leistungsgrenze arbeitet und der Kernel einen Absturz erwartet,
weil die Hardware bald überfordet ist, dann kann er einzelne Prozesse
kurzzeitig pausieren.

Dieser Datenverlust ist ebenfalls mit einem Jepsen-Test reproduzierbar
(vgl.~\cite{es_jepsen_pause}). In großen Setups ist es sehr realistisch, dass
Server in einem Rechnerverbund (auch Cluster genannt) eine starke Last
aufweisen. Sofern es hier zu Lastspitzen kommt, die eine Überlast andeuten, ist
ein Datenverlust sehr wahrscheinlich.

Beide Probleme sind äußerst kritisch für dieses Projekt. In der Theorie ist die
Architektur sehr gut geeignet, denn sie ist skalierbar und hochverfügbar. In
der Praxis zeigt sich leider das Gegenteil. Gerade in großen Setups mit vielen
Nodes ist nicht garantiert, dass Daten wirklich persistent geschrieben werden.
\tm%

\subsubsection{Cassandra}
\label{subsubsec:cassandra}
Cassandra ist ein \gls{SQL} und Java basiertes \gls{DBMS} und wurde ausgelegt
für die Datenhaltung von großen Datenmengen und dessen verbundenem
Management der Arbeitsauslastung auf dem Server. Dazu nutzt die Software ein
Cluster-System über mehrere Server (Nodes), um ein vollständiges Versagen durch
den Ausfall von Hardwarekomponenten oder Softwareteilen auf dem System zu
verhindern. Dazu wird ein \gls{Peer-to-Peer} Verfahren verwendet, welches die
Dateninformationen auf mehrere Server verteilt und verwaltet. Im
Cassandra-Umfeld wird dies über das von dem Hersteller definierte Protokoll
„gossip“ verwaltet. Es handelt sich dabei um ein Kommunikationsprotokoll,
welches den Austausch von Statusinformationen und Metainformationen zwischen den
einzelnen Nodes ermöglicht. Wenn auf einem einzelnen Node Dateninformationen
erstellt, modifiziert oder entfernt werden, erhält ein weiterer Node in der
Infrastruktur diese Information und baut gleichzeitig eine Replikation des
Objektes. Diese Replikationen werden von dem Serveradministrator definiert,
indem dieser den Datenmengen verschiedene \gls{Partitionen} zuteilt. Das
Cassandra-System verteilt jede dieser \gls{Partitionen} an die im Netzwerk
vorhandenen Nodes in einer \gls{Ringtopologie}. Dies bedeutet, dass Partition
eins auf dem ersten Node repliziert wird, Partition zwei auf dem zweiten Node
und immer so weiter, bis das Verfahren den ersten Node wieder erreicht.

\begin{center}
    \inputminted{text}{../listings/cassandra-status.txt}
    \captionof{listing}{cassandra status}
\end{center}

In der obigen Shell-Ausgabe wird ein während der Evaluierungsphase erstelltes
Cassandra-System dargestellt. Dabei besitzt der Prototyp zwei aktive Nodes,
welche für Testzwecke auf dem gleichen physischen Server betrieben werden. Im
Produktivbetrieb sollten diese auf zwei verschiedenen Serversystemen arbeiten,
um die Ausfallsicherheit der Hardware zu gewährleisten. Es ist jedoch zu
erkennen, dass beide Systeme einen Dateneigentum, in der Shellausgabe die
Spalte „Owns“, von 50 Prozentpunkten besitzen. Dies bedeutet, dass jeder Node
jeweils die Hälfte der Dateninformationen bereitstellt, jedoch auch die
zugehörige Hälfte von dem anderen Node als Backup abspeichert. Sollte nun einer
der beiden Nodes ausfallen, könnte der jeweils andere automatisch einen
Dateneigentum von 100 Prozentpunkten erstellen und ohne Verluste
weiterarbeiten. Dies entspricht der \gls{Shared-Nothing-Architektur}.

Je nach Node-Anzahl können die vorhandenen Replikations-Modelle in der
Konfiguration eingestellt und eingesetzt werden. Cassandra bietet zudem drei
Modelle zur Wiederherstellung von Informationen und Datensicherungen an, welche
im Gesamten ein stabiles und sicheres System ermöglichen. Neben diesen
Kerneigenschaften besitzt es jedoch auch noch weitere kleinere Methoden
zur korrekten Datenverarbeitung. Darunter fällt das nicht sofortige Löschen von
Daten in Datenbanktabellen. Stattdessen werden Daten nur als gelöscht markiert
und bei zukünftigen Abfragen zunächst nicht beachtet. Die vollständige Löschung
der Daten wird dann zum Beispiel zu Zeitpunkten erledigt, wenn das System
sicher ist, dass keine Fehler oder Datenleichen entstehen können.
\nl%

\subsubsection{Postgres}
\label{subsubsec:postgres}
Postgres, oder auch PostgreSQL genannt, ist ein \gls{SQL} basiertes \gls{DBMS}.
Die initiale Arbeit an Postgres begann 1986 (vgl.~\cite{old_postgres}). Mit
über 30 Jahren Entwicklung ist es eines der ältesten und am weitesten
verbreitetsten DBMS (vgl.~\cite{db_ranking}). Es steht unter einer eigenen
Open-Source-Lizenz und darf frei genutzt werden (vgl.~\cite{postgres_license}).
Die Community ist sehr aktiv, sie pflegt ein komplexes Wiki
(vgl.~\cite{postgres_wiki}), sowie eine eigene Seite mit Anleitungen für
Einsteiger und Optimierungshilfen für Fortgeschrittene
(vgl.~\cite{postgres_tutorial}).

\begin{center}
    \inputminted{text}{../listings/postgres-clone.txt}
    \captionof{listing}{postgres git clone}
\end{center}

Wie in der obigen Shell-Ausgabe zu sehen, enthält das \gls{Git}-\gls{Repository}
des Postgres Quellcodes am 26.11.2016 41366 einzelne \glslink{Commit}{Commits}
von 41 verschiedenen Autoren. Aufgrund der Patchpolitik von Postgres werden
Patches oftmals an Autoren aus dem Postgres-Team geschickt. Ein Teammitglied
fügt danach den Patch in das Repository ein. Somit lässt sich nicht genau
bestimmen, von wie vielen Leuten Code beigesteuert wurde.

Die Postgres-Architektur sah lange Zeit vor, dass der Dienst auf einem einzigen
Server betrieben wird. Mittlerweile gibt es mehrere Möglichkeiten, die
Architektur anzupassen, um Postgres:

\begin{outline}
  \1 \glslink{Hochverfügbarkeit}{Hochverfügbar} zu betreiben
  \1 \glslink{Skalierung}{Vertikal zu skalieren}
  \1 mit einer Verteilung der Anfragen den einzelnen Server zu entlasten
\end{outline}

Postgres bietet die Möglichkeit der \gls{Streaming Replication}. Hierbei wird
nach dem \gls{Active-Passive-Prinzip} 1-N gearbeitet. Dies bietet zwei
Vorteile:

\begin{outline}
  \1 Leseanfragen können an passive Server gestellt werden.
  \1 im Fehlerfall oder für Wartungsarbeiten kann von einem aktiven auf einen
  beliebigen passiven Server umgeschaltet werden.
  \1 Hochverfügbarer Betrieb der Datenbank.
\end{outline}

Der aktive Server muss somit nur noch Schreibanfragen beantworten. Das
Hinzufügen von passiven Nodes oder das Umschwenken der Schreibanfragen auf
einen von ihnen kann im laufenden Betrieb erfolgen. Die Postgres-Konfiguration
lässt sich zur Laufzeit anpassen, um dem Dienst mehr Ressourcen des physischen
Hosts zuzuweisen.

Mit den Erweiterungen \gls{Pgpool-II} und \gls{Postgres-XC} kann ebenfalls ein
hochverfügbarer Betrieb realisiert werden. Außerdem ermöglichen sie eine
vertikale \gls{Skalierung}. Mehrere Server können hier zu einem Verbund (auch
Cluster genannt) zusammengeschaltet werden. Schreibanfragen können von mehreren
Servern beantwortet werden. Wenn es gewünscht ist, kann nach der
\gls{Shared-Nothing-Architektur} gearbeitet werden. In dem Fall wird nur ein
funktionierender Server im Cluster benötigt. In einem Setup mit mehreren Nodes
führt ein Ausfall einer oder mehrer Nodes nur zu einer verringerten Leistung,
nicht aber zu Datenverlust oder Erreichbarkeitsproblemen des Verbunds.
\tm%

\subsubsection{OpenTSDB}
\label{subsubsec:opentsdb}
OpenTSDB ist ein Zeitreihen-basiertes Programm, auf Englisch „Time Series
Daemon (TSD)“, und arbeitet mit dem Open-Source-Programm \gls{DBMS} „HBase“
zusammen. Es verfügt über die Funktionen von einem Konsolen-basierten Zugang,
sowie verschiedenen Treibern für den Zugriff und Austausch von Informationen
zwischen anderen Netzwerkgeräten. Es übernimmt nicht die Aufgabe der
Datenspeicherung. Dazu wird das zugehörige \gls{DBMS} „HBase“ zwingend
benötigt. OpenTSDB wird genutzt, um einzelne und unabhängige Systeme für den
Datenzugriff bereitzustellen. Dabei können beliebig viele Systeme (Deamons) an
einem HBase-System angebunden sein. Es existiert keinerlei
Informationsaustausch zwischen den einzelnen Deamons, sondern sie nutzen
lediglich alle die gleiche Datengrundlage des HBase-Systems. Dennoch kann man
durch dieses Verfahren die Analyseperformance dynamisch anpassen, da das
HBase-System stark auf die Aggregation und niedrigen Speicherplatzbedarf
optimiert wurde. Die einzelnen Deamons kommunizieren dabei mit dem HBase-System
über einfache Protokolle wie Telnet oder HTTP (API).

Da die Dateninhalte immer in Verbindung mit einem Zeit- und Datumswert stehen,
ist dieses \gls{DBMS} durch die Optimierung des Verarbeitens der Zeit- und
Datums-basierten Datensätze ein Kandidat für die spätere Nutzung in der
Ziellösung.
\nl%

\subsubsection{Hadoop / Hive / Impala}
\label{subsubsec:hadoop_hive_impala}
Das Apache-Hadoop-System wird zum aktuellen Zeitpunkt bei
großen und leistungsfähigen Datenhaltungssystemen eingesetzt, sodass es
auch oftmals in Verbindung mit den Worten „BIG DATA“ genannt wird.

Es handelt sich dabei um eine Vielzahl von Komponenten, bestehend aus mehreren
Frameworks und Ökosystemen, also im Klartext eine Reihe von frei wählbaren
Bibliotheken zur Definition der Datenhaltungslandschaft. Apache Hadoop verfolgt
dabei den Ansatz, alle Dateninformationen in nur einem gebündelten System zu
verwalten und keine zusätzlichen Datenhaltungssysteme, welche gegebenenfalls
nur auf eine einzelne Anforderung spezialisiert sind, zu gründen.

Um diesen Ansatz zu erfüllen, wurde das Cluster-Konzept eingesetzt, Hadoop nennt
es selber „Hadoop Distributed File System“ (HDFS). Dabei werden die Daten in
der Datenhaltung nicht mehr auf einem einzelnen physischen System gehalten,
sowie vor einem Hardwareausfall gesichert, sondern nun auf mehrere Systeme
verteilt. Hadoop teilt dabei jedem in dem Cluster hinzugefügten System eine
spezielle Aufgabe zu. Für die Datenspeicherung existieren dabei drei
Aufgabenfelder. Nummer Eins ist der Data Node, dieser besteht oftmals aus
mehreren Festplatten und einer Netzwerkanbindung. Auf diesem wird ein
Datensatz mehrmals auf verschiedenen Festplatten abgespeichert, sodass bei
einem Ausfall der Datensatz gesichert ist. Es können bei Hadoop beliebig viele
Data Nodes in der Datenhaltungslandschaft eingesetzt werden, je mehr Data
Nodes, umso mehr Speicherplatz und Rechenleistung steht zur Verfügung. Nummer
zwei ist der Name Node. Bei diesem handelt es sich um ein einzelnes System,
welches den Speicherort (Data Node ID) von einem Datensatz bereitstellt. Die
letzte Aufgabe erfüllt der Backup Node. Auch bei diesem handelt es sich um ein
einzelnes System. Das System kommt zum Einsatz, wenn der Name Node ausfallen
sollte. Er synchronisiert sich dauerhaft mit dem Name Node, um im Falle eines
Ausfalls des Name Nodes, sofort die Aufgaben zu übernehmen.

Für das Verwalten des Cluster-Konzepts und zeitbasierenden Aufgaben (im
englischen: Job scheduling) wurde „Hadoop YARN“ gegründet. Dieses System
definiert eigenständig den Austausch von Informationen zwischen den einzelnen
Nodes. Es definiert dabei auch, zu welchem Tageszeitpunkt einzelne Aufgaben
erfüllt werden sollen. Diese Definitionen werden automatisch erstellt und
durchgeführt, jedoch kann dies auch von einer Person manuell eingestellt und
verwaltet werden.

Für die Anbindung späterer Systeme und Methodiken der Datenanalyse, wird
„Hadoop Common“ eingesetzt. Es handelt sich dabei um ein reines
Schnittstellen-Modul, welches standardisierte Protokolle implementiert und
betreibt. Es ermöglicht zum Beispiel die Anbindung der Software „Hive“ und
„Impala“, welche in den Punkt~\ref{subsubsec:hadoop_beschreibung}
und~\ref{subsubsec:impala_beschreibung} erläutert werden.

Für die Bearbeitung und Analyse von zeitkritischen Daten kommt die Komponente
„Hadoop Map Reduce“ zum Einsatz. Diese ist von dem gleichnamigen
Programmiermodell abgeleitet. Es handelt sich dabei um ein
Hadoop-YARN-basiertes System, welches ein paralleles Abarbeiten von Aufgaben
ermöglicht.  Dabei werden Analyseanfragen nun nicht mehr in mehreren
Server-Threads auf einem einzelnen System aufgeteilt, sondern auf mehreren Data
Nodes, welche wiederum die Aufgaben in mehreren Threads abarbeiten können.
Gleichzeitig besitzt diese Hadoop-Komponente einen Data Node, welcher aus
reinem Arbeitsspeicher besteht. Dieser wird genutzt, um Informationen für eine
kurzfristige Zeit auf ein schnelleres Medium zu laden, sodass die Information
im Falle einer erneuten Abfrage innerhalb dieser Zeit erneut genutzt werden
kann, ohne diese wieder von einem langsameren Medium laden zu müssen. Die
Zeitspanne kann dabei von einem Administrator festgelegt werden.
Arbeitsspeicher ist bis dato einer der schnellsten Speichermedien und kommt bei
zeitkritischen Datenabfragen zum Einsatz.

Diese vier Komponenten bilden den „Hadoop Core“, welcher zwingend für den
Betrieb eines Hadoop-Systems benötigt wird. Hadoop basierende Systeme zur
Erfüllung von verschiedenen und spezialisierten Aufgaben werden unter dem
„Hadoop Ecosystem“ zusammengefasst. Eine vollständige Liste von
Hadoop basierenden Systemen und Projekten kann auf der offiziellen Webseite des
Entwicklers entnommen werden (vgl.~\cite{Hadoop_related_projects}).
\nl%

\subsubsection{Beschreibung Hive}
\label{subsubsec:hadoop_beschreibung}
Das Apache-Hive-System ist eine Hadoop basierende Software und nutzt die im
Punkt~\ref{subsubsec:hadoop_hive_impala} erläuterten Schnittstellten für den
Datenzugriff. Es führt dabei eine spezialisierte Aufgabe auf dem Hadoop-System
aus und fällt somit nicht unter ein eigenständiges Datenhaltungssystem. In der
Hadoop-Landschaft wird es unter dem Begriff „Hadoop Ecosystem“ eingeordnet.

Die angesprochene spezialisierte Aufgabe ist dabei die Implementierung von
einem sogenannten „Data Warehouse“ (DWH) in dem Hadoop Netzwerk. Ein Data
Warehouse wird in Unternehmen verwendet, um eine vollständige Übersicht zu
einem spezifischen Inhalt zu erhalten. Dies könnte als Beispiel die im Land
verteilten Einkaufsläden des Unternehmens sein, welche alle ein eigenes
Kassensystem mit einer Datenbank besitzen. Mit Hilfe des Extraktions-,
Transformations- und Ladeprozesses (ETL) werden die unterschiedlich
abgespeicherten Informationen von jedem einzelnen Kassensystem ausgelesen und
anschließend dem Aufbau der Tabellen (Schema/Datenhaltungsstruktur) von dem
Data Warehouse angepasst und somit die vollständige Speicherung im System
gewährleistet.

Dadurch das in dem Hadoop-System keinerlei Struktur der Daten existiert,
übernimmt Apache Hive mit Hilfe der Methodiken von dem Data Warehouse das
Lesen, Schreiben und Managen von Dateninformationen. Es beginnt dabei mit
Werkzeugen für den Zugriff auf die Daten innerhalb des Hadoop-Systems. Dies ist
zum einem die Bereitstellung von einem standardisierten Treiber für den
externen Zugriff auf die Daten, sowie die Vorgabe einer Abfragesprache mit dem
bekannten Namen „SQL“. Jedoch implementiert Apache Hive auf dem Hadoop-System
auch Funktionen wie das Reporten von Inhalten über eine wählbare Schnittstelle
wie zum Beispiel dem Versand einer eMail oder Anzeige auf einer
Internetwebseite.

Zusammengefasst hat Apache Hive die Aufgabe zur Bereitstellung einer strukturierten
Sicht (Schema) auf die Daten innerhalb des Hadoop-Systems. Dabei kann Apache
Hive mehrere Sichten auf einen einzelnen Datensatz definieren und somit auch
das Rechtesystem innerhalb der Datenhaltung unterstützen, indem diese nur
für ausgewählte Benutzeraccounts zur Benutzung freigeschaltet werden. Die in
dem Data Warehouse bekannten „Data Marts“ fallen somit weg und werden durch
den Aufbau der genannten Sichten ersetzt. Ein Data Mart ist eine von dem
Datawarehouse getrennte Datenbank, welche genutzt wird, um für ausgewählte
getrennte Daten Analysen und Reports zu erstellen. Diese können zum Beispeil
nach Abteilungen im Unternehmen getrennt werden, sodass das Marketing nur Daten
zu Produkten und Beständen erhält, jedoch keine technisch orientierten Daten.
Des weiteren werden Data Marts in der dritten Normalform und ein Datawarehouse
in der zweiten Normalform betrieben, entsprechend dienen sie dazu, die Daten
nochmals fachlich aufzubereiten.
\nl%

\subsubsection{Beschreibung Impala}
\label{subsubsec:impala_beschreibung}
Das Apache-Impala-System ist eine Hadoop basierende Software und nutzt die
im Punkt~\ref{subsubsec:hadoop_hive_impala} erläuterteten Schnittstellten
für den Datenzugriff. Es führt dabei eine spezialisierte Aufgabe auf dem
Hadoop-System aus und fällt somit nicht unter ein eigenständiges
Datenhaltungssystem. In der Hadoop-Landschaft wird es unter dem Begriff
„Hadoop Ecosystem“ eingeordnet.

Die angesprochene spezialisierte Aufgabe ist dabei eine ähnliche
Implementierung, wie in Punkt~\ref{subsubsec:hadoop_beschreibung}. Apache
Impala stellt ebenfalls eine Reihe von Werkzeugen bereit, darunter eine
spezialisierte Abfragesprache auf Basis von „SQL“. Apache Impala arbeitet auf
den konsolidierten Metadaten des Hadoop-Systems, benötigt jedoch auch die
Verknüpfung (Interface) eines Strukturmodells. In den meisten Fällen kommt
dabei Apache Hive zum Einsatz. Apache Impala kann die Inhalte von Apache Hive
Eins-zu-eins übernehmen und ausführen. Es kann jedoch auch ohne einem Apache
Hive Systems ausgeführt werden.

Apache Impala wird eingesetzt, um die Schnelligkeit der Datenabfragen auf einem
Hadoop-System zu verbessern. Es wird dabei auf jedem Data Node installiert.
Dabei erhalten die Data Nodes neben der eigentlichen Datenspeicherung noch
zusätzlich die Funktionen zur Ausübung von Datenbankabfragen (SQL). Konkret
wird auf jedem Data Node ein „Query Planer“, „Query Coordinator“ und „Query
Executor“ bereitgestellt. Der Query Planer hält dabei die Kommunikation mit der
SQL-Applikation innerhalb des Netzwerkes. Dieser gibt definierte Datenabfragen
von Benutzern an einen beliebigen Data Node weiter. Nachdem der Query Planer
die Datenabfrage erhalten hat, kommuniziert er dies an den Query Coordinator,
welcher wiederrum alle umliegenden Data Nodes um Unterstützung bei der
Datenabfrage bittet. Dies geschieht über den Query Executor, welcher, wie schon
der Name sagt, die Abfrage auf dem Hadoop-HDFS-System ausführt und die Daten
zurück an den Query Coordinator liefert. Die von dem Query Coordinator
zusammengesetzten Daten liefert dieser anschließend an die SQL-Applikation
zurück und der gesamte Prozess ist abgeschlossen.

Durch das Nutzen von dem Arbeitsspeicher auf den Data Nodes, sowie der
gleichzeitigen Nutzung des Hadoop-Map-Reduce-Verfahrens, werden die
Datenabfragen um ein Vielfaches beschleunigt. Da die Leistung des Clusters
durch mehrere Faktoren beeinflusst wird, kann keine exakte Angabe der
Performanceverbesserung geliefert werden. Diese Faktoren sind zum Beispiel die
Schreibweise der SQL-Syntax, da ein nicht optimierter Code die
Performanceverbesserung wieder mindern kann. Fakt ist jedoch, dass mit einem
korrekt aufgesetzten Cluster System der Einsatz von Apache Impala die
Performance der Datenlandschaft optimiert.
\nl%

\subsubsection{Evaluierungsbericht}
\label{subsubsec:hadoop_hive_impala_evaluierung}
Während der Evaluierungsphase wurde ein vollständiger Hadoop Core als Prototyp
auf zwei Testservern installiert und in den Grundlagen konfiguriert. Während
dieser Arbeit ist dem Projektteam bereits klar geworden, dass nur für die
Konfiguration des ersten Systemstarts, also dem Verbindungsaufbau, sowie den
allgemeinen Funktionen eines Hadoop Core Systems, zu viel Zeit benötigt wird.
Des Weiteren wurde nach dem Testen des Systems die Konfigurationsmöglichkeiten
im Bereich Optimierung der Performance und Geschwindigkeit studiert. Dabei
wurde klar, dass durch die Größe des Systems mehr als die gesamte Projektzeit
benötigt wird, um eine optimale und vollständige Konfiguration und
Systemstruktur zu erarbeiten. Aufgrund dessen wurde die Verwendung eines
Hadoop-Systems in diesem Projekt sehr schnell ausgeschlossen. Hauptgründe dabei
waren zum einem die vielen Werkzeuge und Methoden die das Hadoop-Ecosystem
bereitstellen konnte und zum anderen die Masse an Dokumentationen und dem damit
verbundenen Konfigurationsaufwand. Hinzu kommt noch die darauf folgende
fachliche Definition von Datenstrukturen. Diese hätte durch die breite Auswahl
an Methodiken auf einem Hadoop-System ebenfalls wieder viel Zeit beansprucht.
Zunächst hätte eines der verfügbaren Softwareelemente des Hadoop-Ecosystems
evaluiert und ausgewählt werden gemusst. Zusammengefasst wäre das System den
technischen Anforderungen an dem Projekt gewachsen, jedoch wäre der Einsatz mit
einem zu großen Aufwand verbunden, welchen das Projektteam in dem vorgegebenen
Projektzeitraum nicht erbringen könnte.
\nl%

\subsection{Abschluss der Evaluierung}
\label{subsec:abschluss_db_evaluierung}
Nachdem alle Systeme von dem Projektteam auf die genannten Kriterien untersucht
wurden, wurde gemeinsam abschließend eine normale Entscheidungstabelle
angefertigt. Dabei kann jedes im
Punkt~\ref{subsec:DBS_vorbereitung_der_evaluierung} definierte Kriterium einen
Wert zwischen 0 und 10 erhalten. Zehn ist dabei die beste und Null die
schlechteste Bewertung. Die Systeme mit der höchsten Gesamtpunktzahl würden
anschließend in der engeren Auswahl beurteilt werden. Der Punkt Wissen
bezieht sich auf vorhandenes Wissen innerhalb des Projektteams.

Diese Entscheidungstabelle wurde wie folgt aufgebaut:
\begin{center}
\begin{tabular}{lccccc}
  \toprule
  System & Wissen & Skalierbarkeit & Umwelt & Datensicherung & Gesamt \\
  \midrule
  elasticsearch & 8  & 7  & 6  & 3  & 24 \\
  cassandra     & 7  & 5  & 6  & 8  & 26 \\
  Postgres      & 10 & 10 & 7  & 8  & 35 \\
  OpenTSDB      & 6  & 5  & 7  & 3  & 21 \\
  KNIME         & 0  & 0  & 0  & 0  & 0  \\
  Hadoop Core   & 8  & 10 & 4  & 8  & 30 \\
  \bottomrule
\end{tabular}
\captionof{table}{Gesamtbewertung der Datenbanksysteme}
\end{center}

Wie in der Entscheidungstabelle abzulesen ist, wurde im Anschluss über die
Systeme Postgres und dem Hadoop Core diskutiert. Es wurde bereits während der
Evaluierungsphase erkannt, dass bei der Verwendung eines Hadoop-Core-Setups die
eingeplante Realisierungs-Zeit zu gering ist. Aufgrund dessen und einer höheren
Gesamtpunktzahl wird das System Postgres für dieses Projekt verwendet.
\nl%

\section{Schnittstelle - Datenbereitstellung}
\label{sec:schnittstelle_datenbereitstellung}
Neben der herkömmlichen Methode zur Datenabfrage von Datenbank-Systemen über
einen Datenbanktreiber, existiert noch die Möglichkeit einer API\@.

Die Verwendung von einer API-Schnittstelle besitzt Nachteile sowie Vorteile,
beziehungsweise verbirgt gewisse Gefahren in Hinsicht auf den späteren Einsatz
im Produktivsystem. Eine dieser Gefahren ist das Vernachlässigen der
Dokumentation einer API\@. Dadurch dass während der Entwicklung einer API
bereits definiert wird, welche Fähigkeiten und Methodiken die API besitzen
soll, ist eine nicht vorhandene oder unvollständige Dokumentation suboptimal.
Darum muss bei der Auswahl des genutzten API-Systems darauf geachtet werden,
dass die Dokumentation von Softwarekomponenten für den Entwickler so einfach
wie möglich erfolgt.

Wenn die Dokumentation einer API vollständig und das ausführende System den
Performance-Anforderungen gerecht ist, bringt die Nutzung eines API-Systems
Vorteile. Darunter ist eine bessere Verwaltung von Zugriffsrechten und somit
auch gleichzeitig eine Verbesserung der Systemsicherheit. Durch den vorab
festgelegten Funktionsumfang der API kann der Systeminhaber gezielt
kontrollieren, welche Inhalte ein Benutzer sehen oder modifizieren kann. Dabei
können grobe Einschränkungen, wie das Lesen oder Schreiben auf der gesamten
Datenbank, oder auch detaillierte Einschränkungen, wie das Lesen oder Schreiben
von einer einzelnen Tabelle in der Datenbank, kontrolliert werden. Anders als
bei der Verbindung über einen Datenbanktreiber kann bei der API der potentielle
Angreifer keine eigenen Datenbankbefehle definieren und ausführen. Somit wird
das Risiko der Datenmanipulation durch eine zusätzliche Schicht (Layer)
verringert.

Oftmals benötigen die Systeme, welche auf die Daten des Datenhaltungssystems
zugreifen, die Daten in einer anderen Struktur als diese abgespeichert sind.
Durch die Programmiermöglichkeiten einer API können während der Abfrage der
Informationen auf dem Datenhaltungssystem bereits Datenanpassungen vorgenommen
werden. Diese Anpassungen können zum Beispiel Funktionen, wie das anonymisieren
oder pseudonymisieren sein, welche das Datenhaltungssystem nicht ausführen
kann. Die API kann jedoch auch zum Beispiel die Daten vor der Auslieferung
beliebig anpassen und somit auf ein weiteres System optimal anpassen, ohne
dabei die Ursprungsdaten zu verändern.
\nl%

\subsection{Vorbereitung der Evaluierung}
\label{subsec:api_vorbereitung_der_evaluierung}
Anders als bei der Evalierung der Datenbanksysteme wurde bei der Evaluierung
der möglichen API-Systeme keine Liste im Voraus erstellt. Dies lag daran, dass
keines der Projektmitglieder oder keiner der Interessenten an dem Projekt
Erfahrungen mit API-Systemen im Vorhinein besaß und dies erst während der
Projektphase angeeignet werden musste.

Aufgrund dessen wurde entschieden, eine Liste von Anforderungen an das
API-System zu erstellen und im Anschluss während der Evaluierungsphase
Softwarelösungen zu suchen, welche die Anforderungen am besten erfüllen können.

Die festgehaltenen Anforderungen sind nun nach Relevanz absteigend aufgelistet:

\begin{outline}
  \1 Die API Software muss auf einer selbst administrierten Hardware-Umgebung
  betrieben werden können. Das Verwenden eines fertigen API-Softwarepakets
  eines Cloudanbieters, bei welchem der Administrator alle Konfigurationen
  über ein Web-Interface des Providers vornimmt und der Cloudanbieter die
  vollständige Hardware-Umgebung administriert, ist ein absolutes
  Ausschlusskriterium. Es ist vorgesehen, dass das API-System möglichst nah an
  dem Datenhaltungssystem liegt, um mögliche Latenzen zwischen den beiden
  Systemen so niedrig wie möglich zu halten. Zudem sollen externe Cloudanbieter
  keinen Zugriff auf die teilweise kundenbezogenen Daten der Datenhaltung
  erhalten, auch wenn das System nur Performance-Daten ohne Kundenbezug
  ausliefern soll. Außerdem kann während des Projekts entschieden werden, dass
  das Datenhaltungssystem nicht von externen Systemen erreichbar sein soll,
  sodass die damit verbundene Lösung über einen Cloudprovider nicht möglich
  wäre. Externe Systeme sind in diesem Projekt Geräte, welche außerhalb des
  Unternehmens betrieben werden.
  \1 Die API-Software muss vollständig konfigurierbar und administrierbar sein.
  Entsprechend soll die Software nur eine leere Hülle mit Standardfunktionen
  einer API zur Verfügung stellen. Eine Konfiguration von einzelnen Parametern
  ist dabei erwünscht. Zu diesen Parametern gehört als Beispiel der Port, auf
  welchem das API-System betrieben und von anderen Systemen erreicht werden
  kann. Zur Hilfe der Admnistration soll das API-System ein Framework für
  Entwickler bereitstellen. Mit diesem können wir als Projektmitglieder
  anschließend die Logik in das System implementieren und somit definieren,
  welche Funktionen die API ausüben darf. Ein Framework in einer bekannten
  Entwicklersprache ist dabei erwünscht. Dies hat den Vorteil, dass der
  zuständige Entwickler im Projekt keine weitere Einarbeitungszeit für die neue
  Entwicklersprache benötigt.
  \1 Das API-System sollte innerhalb des Projektes keine Kosten verursachen.
  Dazu gehören Lizenzkosten sowie laufende Betriebskosten bei externen
  Cloudanbietern. Spätere Kosten bei dem Auftraggeber sind in dieser
  Anforderung nicht mit inbegriffen. Grundsätzlich gilt für das ganze Projekt,
  dass nur Software mit einer Open-Source-Lizenz genutzt werden darf.
  \1 Die Erstellung von späteren Anwendungshandbüchern der API soll dem API
  Entwickler so einfach wie möglich fallen. Eine sehr beliebte
  Dokumentationsmöglichkeit ist dabei das Dokumentieren von wichtigen
  Informationen direkt in dem Quellcode des API-Systems. Entwickler können
  dabei über die Kommentierungsfunktion der jeweiligen Entwicklersprache eine
  Beschreibung sowie Ein- und Ausgabe von Werten festlegen. Diese werden im
  Anschluss während des Kompiliervorgangs in einem strukturierten Dokument
  festgehalten und neben der eigentlichen API-Software mit ausgeliefert. Dieses
  separate Dokument ist oftmals in Form einer HTML oder PDF und kann auf
  gängigen Computersystemen eingesehen werden. Die beschriebene oder ähnliche
  Dokumentationsmöglichkeit sollte die API-Software zur Verfügung stellen.
  Alternativ kann auch erst die Dokumentation geschrieben werden und darauf
  basierend Quellcode generiert werden. Beide Verfahren nennt man „One Source
  of Truth Prinzip“. Hierbei wird sichergestellt, dass es nur eine autoritative
  Quelle gibt und der zugehörige Pendant auf der Quelle beruht. Dies vermeidet
  Dokumentation die nicht mit den eigentlichen Funktionen der Software
  übereinstimmen.
  \1 Bereits vorhandenes Wissen über das API-System. Dazu gehört, dass diese
  eine umfangreiche Dokumentation besitzt und eine aktive und große Community
  für Diskussionen und Fragen vorhanden ist. Die dazugehörige
  Programmiersprache der API sollte ebenfalls für die Projektmitglieder
  nachvollziehbar sein sowie eine umfangreiche Dokumentation und aktive
  Community für Diskussionen und Fragen besitzen.
  \1 Das API-System sollte nicht von einem einzelnem Betriebssystem abhängig
  sein. Das bedeutet, dass die API auf jedem beliebigen Betriebssystem
  vollständig und zuverlässig ausgeführt und arbeiten kann.
\end{outline}
\nl%

\subsection{Durchführung der Evaluierung}
\label{subsec:api_durchfuehrung_der_evaluierung}
Bei der Evaluierung der API-Systeme wurden die in
Punkt~\ref{subsec:api_vorbereitung_der_evaluierung} festgehaltenen
Anforderungen der API als Leitlinie für die Durchführung verwendet. Als ersten
Schritt wurde zunächst Recherche betrieben, damit das Projektteam einen
allgemeinen Eindruck über die aktuelle Marktsituation zu API-Systemen erhält.
Dabei wurden im Internet mehrere Blogs, Foren sowie Artikel von
Fachzeitschriften gelesen. Die dort angesprochenen API-Systeme wurden
anschließend mit den bereits definierten Anforderungen abgeglichen und
aussortiert. Dabei stellten die Projektmitglieder fest, dass die meisten
API-Systeme von Cloudprovidern nur in bereits vorkonfigurierten Paketen mit
einem monatlichen Festpreis angeboten werden. Beides ist ein absolutes
Ausschlusskriterium für den späteren Produktivbetrieb sowie Nutzung der
Software. Ebenfalls wurde bei den vorkonfigurierten Paketen eine eigene
Definition von Funktionen komplett verboten. Lediglich ein einzelner
Cloudprovider hatte ein Angebotsmodel, bei dem die Kunden das betriebene
API-System in Hinsischt der Funktionen modifizieren durften.

Aufgrund dieser Erkentnisse hat sich das Projektteam während der Evaluierung
für ein API-System mit dem Namen „Swagger“ entschieden. Es ist aktuell neben den
API-Systemen „RAML“ und „API-Blueprint“ das einzige API-System auf dem Markt,
welches den Betrieb auf einer selbst administrierten Hardware-Umgebung zulässt.
Für den Vergleich der drei Systeme wurde eine Entscheidungstabelle mit
gewichteten Faktoren erstellt und ausgewertet. Die in der nachfolgenden
Entscheidungstabelle genannten Faktoren wurden aus
Punkt~\ref{subsec:api_vorbereitung_der_evaluierung} abgeleitet und können dort
im Detail nachgeschaut werden. Jeder der definierten Kriterien kann einen
Wert zwischen 0 und 10 erhalten. Zehn ist dabei die beste und Null die
schlechteste Bewertung. Die Faktoren bezüglich der selbst administrierbaren
Hardwareumgebung, und die damit verbundenen externen Kosten sind bei allen drei
genannten API-Systemen gegeben und werden nicht berücksichtigt. API-Systeme bei
denen dies nicht gegeben ist, wurden von dem Projektteam aus der Bewertung
ausgeschlossen.

\begin{center}
\begin{tabular}{lccccc}
  \toprule
  System & Modifizierbar & Doku & Wissen & Abhängigkeit & Gesamt \\
  \midrule
  Swagger       & 10 & 9  & 10 & 10 & 39 \\
  RAML          & 10 & 8  & 7  & 10 & 35 \\
  API-Blueprint & 8  & 6  & 5  & 9  & 28 \\
  \bottomrule
\end{tabular}
\captionof{table}{Gesamtbewertung der API-Systeme}
\end{center}
\nl%

\section{Datenvisualisierung}
\label{sec:datenvisualisierung}
Bevor das Projekt begann, musste eine Softwareliste mit möglichen Frontends
erarbeitet werden. Bereits vorab wurden folgende Softwarelösungen in das
Pflichtenheft aufgenommen:

\begin{outline}
  \1 Grafana
  \1 Graphite
  \1 Zabbix-Frontend
\end{outline}

Nach Absprache mit dem Projektteam sind noch folgende Systeme in die
Evaluierungsliste aufgenommen worden:

\begin{outline}
  \1 Datadog
  \1 Gridster-D3
  \1 Netdata
\end{outline}

Das Hinzufügen von weiteren Softwarelösungen während der Evaluierungsphase
bedarf keiner weiteren Zustimmung durch den Auftraggeber, sondern konnten direkt
mit dem Projektteam besprochen und aufgenommen werden. Zusätzlich zu der Liste
mit den einzelnen Softwarelösungen, wurden verschiedene Anforderungen in einem
Kriterienkatalog definiert, welche im späteren Verlauf der Planung für die
Evaluierung der einzelnen Frontends verwendet werden. Die zu evaluierende
Softwareliste besteht zum Teil aus Softwarelösungen, welche im Unternehmen von
einzelnen Projektmitgliedern bereits erfolgreich eingesetzt werden und zum
anderen aus Vorschlägen des Auftraggebers.
\mr%

\subsection{Definition eines Graphen}
\label{definition_eines_graphen}
Ein Graph besteht im Wesentlichen aus mehreren Knoten und Kanten, welche auf
einem Koordinatensystem angelegt sind. Mit Hilfe von zwei Knoten, bestehend
jeweils aus einer Koordinate (x/y), werden diese eindeutig zu einer Kante
definiert. Ein linearer Graph wird genau dann visualisiert, wenn zwei Knoten
miteinander verbunden werden. Zusätzlich gibt es die Unterscheidung zwischen
einem gerichteten und ungerichteten Graphen. Ein gerichteter Graph besteht aus
einem Anfangsknoten A und einem Endknoten B, welche mit einer Kante verbunden
sind. Bei einem ungerichteten Graphen ist sowohl der Knoten A mit dem Knoten B,
als auch der Knoten B mit Knoten A als Kante miteinander
verbunden (vgl.~\cite{kaiser2008c}).

Auf der Weboberfläche werden dem User oder dem Administrator einzelne Graphen
von z.B. CPU-Auslastung, RAM-Auslastung oder SSD-Schreibzugriffe angezeigt.
Diese Graphen werden entweder in Linerar- oder Balkenform dargestellt. Lineare
Graphen haben eine X- und Y-Achse und besitzen zusätzlich einen Anfang und
einen wachsenden Endpunkt. Er wird mittels
Metrik~\ref{section:Begriffserklärung}, welche die Weboberfläche aus den
Datenhaltungssystemen abgreift, erstellt und kann vom Administrator ebenfalls
nur für eine definierte Zeitspanne angezeigt werden. Balkendiagramme haben nur
eine X-Achse, wo der gewünschte Analysewert eingetragen ist und werden z.B. für
ausgefallene SSDs in einem bestimmten Zeitraum verwendet. Diese werden über die
Metrik aus dem Datenhaltungssystem abgegriffen und dem Administrator über eine
Übersicht angezeigt. Ein Graph- bzw. Balkendiagramm visualisiert immer eine
oder mehrere Timeseries~\ref{section:Begriffserklärung}.

\subsection{Vorbereitung der Evaluierung}
\label{subsec:vorbereiten_der_evaluierung_datenvisualisierung}
Damit die Evaluierungen mittels einer Nutzwertanalyse der einzelnen
Softwarelösungen durchgeführt werden können, müssen zuvor im ersten Schritt
Kriterien mit dem Projektteam in einen Kriterienkatalog aufgenommen werden.
Anschließend werden diesen Kriterien Gewichtungen zugewiesen. Bei den Kriterien
musste darauf geachtet werden, dass diese während der Beurteilung der
Anforderungen eine Mach- und Realisierbarkeit besitzen. Nachdem die einzelnen
Kriterien zusammen mit dem Projektteam erstellt worden sind, wurden diese im
Anschluss von jedem einzelnen Projektmitglied auf die Mach- und
Realisierbarkeit geprüft. Erst hiernach wurden diese in einem Kriterienkatalog
aufgenommen.

Die einzelnen Kriterien wurden im späteren Verlauf der Evaluierung
zusammengefasst und Gewichtet. Die folgenden Kriterien wurde in die
Nutzwertanalyse aufgenommen: Skalierbar, Kosten, Sicherheit, Performance und
Dokumentation. Im nachfolgenden, können die einzelnen Kriterien entnommen
werden:

\begin{outline}
  \1 Es muss eine Authentisierung, Authentifizierung und Autorisierung
  gewährleistet werden. Dies bedeutet, dass die Daten und damit auch die
  Weboberfläche selbst vor unautorisiertem Zugriff durch Dritte geschützt sein
  muss. Ebenfalls muss die Verbindung, welche das Frontend nutzt, um die Daten
  abgreifen zu können, vor Dritten so geschützt und gesichert werden. Der
  Zugriff auf die Weboberfläche soll mit einer Transportverschlüsselung durch
  \gls{HTTPS} durchgeführt werden und je nach Kundenanforderung auch eigene
  Client-Zertifikate zulassen. Die Anmeldung am Frontend kann entweder über
  einen lokalen Administrator, als auch über eine LDAP-Anbindung (Lightweight
  Directory Access Protocol) verfügen, um hier z.B. Unternehmensverzeichnisse
  anbinden zu können.
  \1 Eine inhaltlich gute und umfangreiche Community und Dokumentation über die
  Datenvisualisierung. Der Auftraggeber und auch die Projektmitglieder selbst,
  benötigen keine große Schulung mit dem Umgang des Frontends. Dies erleichtert
  zum einen den späteren Betrieb, als auch die Wartung im späteren Verlauf,
  wenn das Produkt bei einem Kunden eingesetzt wird.
  \1 Eine leichte Anpassung, Konfiguration und direkte Ausgabe der
  Visualisierung von CPU, RAM, Netzwerk oder ähnlichen Datenwerten. Hiermit ist
  gemeint, dass gewährleistet werden muss, dass die einzelnen Graphen schnell
  und einfach (endanwenderspezifisch) direkt angepasst werden können. Ebenso
  muss die Anbindung an eventuell schon vorhandene Datenbanksysteme einfach
  durchgeführt werden können.
  \1 Die Datensicherung ist bei dem Frontend ein wesentlicher Aspekt. Dabei
  muss die Weboberfläche bereits Möglichkeiten bieten, eine Datensicherung des
  \glslink{Dashboard}{Dashboards} durchführen zu können. Das Dashboard
  beinhaltet alle Graphen und vorkonfigurierten Werte. Diese sollen über zum
  Beispiel ein externes System (Dateifreigabeserver oder ähnliches) abgelegt
  werden und können somit einfach dupliziert und gesichert werden.
  \1 Plugin-Erweiterung. Das Frontend muss eine Plugin-Erweiterung
  unterstützen. Es dient dazu, dass im späteren Verlauf, wenn das Produkt
  bereits beim Kunden eingesetzt wird, auf bestimmte Kundenwünsche direkt
  eingegangen werden kann. Diese können dann mit schon zur Verfügung gestellten
  Plugins zum Beispiel in der Community oder aber durch selbst programmierte
  Plugins realisiert werden. Die Plugins werden in Javascript geschrieben. Die
  Weboberfläche sollte jedoch auch andere Sprachen unterstützen, um so eine
  Unabhängigkeit von Entwicklern gewährleisten zu können. Unternehmen müssen so
  nicht auf spezielle Entwickler von bestimmten Sprachen zurückgreifen.
  \1 Die genutzten Softwarelösungen müssen eine Open-Source-Lizenz besitzen, um
  hier unabhängig von kostenpflichtigen Lizenzen und Copyright-Rechten zu sein.
  Ebenfalls ist eine Open-Source-Software nötig, da hier der Quelltext der
  Software angepasst und verändert werden darf.
  \1 Bei der Evaluierung der Weboberfläche muss auf die Performance der
  Weboberfläche geachtet werden. Ein wichtiges Kernkriterium ist dabei, dass
  der Administrator eine gewisse Zeitspanne vordefinieren kann, in welcher die
  Graphen neu visualisiert und Daten aus dem Datenhaltungssystem abgerufen
  werden können. Dies dient dazu, dass das Datenhaltsungssystem nicht mit
  Anfragen überlastet wird. Ein kurzes Rechenbeispiel soll dies verdeutlichen.
  Acht unterschiedliche User greifen gleichzeitig auf die Weboberfläche zu und
  visualisieren hier im Fallbeispiel acht Graphen mit unterschiedlichen
  Werten. Nun sind hier insgesamt 64 Graphen, welche sekündlich eine Abfrage an
  das Datenhaltungssystem senden werden. Da auf das Datenhaltungssystem nicht
  nur die Weboberfläche zugreift, sondern auch die Datenerfassung, um Daten
  abzuspeichern, kann es hier zu massiven Performance-Problemen kommen, die
  durch die Weboberfläche verursacht werden können.
\end{outline}
\mr%

\subsection{Durchführung der Evaluierung}
\label{subsec:durchfuehrung_evaluierung_datenvisualisierung}
Für die Durchführung der Evaluierung der ausgewählten Datenvisualisierungs
Software aus der obigen Liste~\ref{sec:datenvisualisierung} wurden die
entsprechenden Kriterien zur Evaluierung aus
Punkt~\ref{subsec:vorbereiten_der_evaluierung_datenvisualisierung} verwendet.


Bevor die Evaluierung der einzelnen Softwarelösungen durchgeführt werden
konnte, wurden vorab bereits eine Vielzahl an Informationen über die einzelnen
Softwarelösungen mittels Internetrecherche und Fachzeitschriften eingeholt.
Zusätzlich wurden die einzelnen Softwarelösungen auf zwei Projektmitglieder
aufgeteilt, um die Evaluierung und damit auch die Projektplanung zeitlich zu
verkürzen.

Nachdem die Analyse der Softwarelösungen von den einzelnen Projektmitgliedern
abgeschlossen wurde, wurden die gesammelten Resultate Herrn Reuter zur
Verfügung gestellt. Mit diesen Resultaten konnte Herr Reuter eine Auswertung
aller Softwarelösungen durchführen. Um die Nutzwertanalyse durchführen zu
können, wurde jede Softwarelösung gegenübergestellt und auf den
Kriterienkatalog geprüft. Hierfür wurde ein Punktesystem von Null bis Fünf
festgelegt, wobei Null die schlechteste Punktzahl und Fünf die beste Punktzahl
ist. Diese vergebenen Punkte, wurden anschließend mit der Gewichtung in eine
Tabelle eingetragen und Multipliziert, welches dann die Gesamtpunktzahl ergibt.
Die Tabellen, werden im Abschnitt~\ref{subsec:abschluss_evaluierung} genauer
erklärt.

Während der Evaluierung der einzelnen Softwarelösungen ist aufgefallen, dass
die Softwarelösung `Gridster-D3' keine direkte Visualisierungssoftware ist. Die
entsprechende Evaluierung, von diesem Produkt, kann unter
Punkt~\ref{subsubsec:gridster-d3} vollständig entnommen werden.

Im Nachfolgenden Punkt sind die einzelnen Softwarelösungen mit den einzelnen
Anforderungen und Evaluierungsergebnissen aufgelistet:
\mr%

\subsubsection{Grafana}
\label{subsubsec:grafana}
Grafana ist eine von Torkel Ödegaard \& Raintank Inc entwickelte
Open-Source-Visualisierungssoftware. Sie bietet neben einfachen
Visualisierungsgraphen noch weitere Ereignisfunktionen. Um die Anmeldung an der
Weboberfläche durchführen zu können, kann hier wahlweise die Anmeldung mit
einem \gls{Active Directory} Konto oder lokalen Benutzer plus
Kennwortkombination durchgeführt werden. Hinzu können verschiedene
Berechtigungen vergeben werden. Somit können zum Beispiel Mitarbeiter aus der
Führungsebene eingeschränkte Zugriffe auf das System erhalten, um hier zum
Beispiel Analysen durchführen zu können. Grafana arbeitet mit fast jedem
Timeseries Datenhaltungssystem, wie z.B. InfluxDB zusammen. Zusätzlich kann es
über Plugins erweitert werden. Damit Grafana API Systeme (siehe auch
Abschnitt~\ref{sec:schnittstelle_datenbereitstellung}) ansprechen kann, wird
hier ein solches Plugin benötigt. Plugins werden zur Erweiterung der Software
in der Programmiersprache Java entwickelt und installiert. Diese werden zum
Beispiel dann benötigt, wenn weitere Graphen, Analysen oder unterstützte
Datenquellen erweitert werden sollen. Dies bedeutet, dass das
Datenhaltungssystem API Statements zur Verfügung stellen kann und hiermit
Graphen visualisieren kann. Grafana kann über einen \gls{Datenbanktreiber} oder
eine Schnittstelle in Form eines Treibers einfache Statements auf das
Datenhaltungssystem durchführen, um die Daten abzugreifen. Es ist zusätzlich
eine Open-Source-Software, wodurch der Quellcode angepasst und verändert werden
darf. Dies bedarf zum Beispiel bei Anpassungen an bestimmte Kundenumgebungen.
Dadurch entstehen auch keine Support-Kosten, da die Software nicht als ein
Dienst angeboten wird. Grafana hat eine stetig wachsende Community mit
sämtlichen Dokumenten zur Einrichtung oder Erweiterung des Systems. Der gesamte
Quellcode ist auf GitHub veröffentlicht, somit haben auch andere Mitglieder von
GitHub die Möglichkeit, das Produkt zu verbessern und neue Funktionen zu
implementieren. Die vom Anwender erstellten Dashboards mit allen Graphen und
Statements können vom System, wie auch vom Anwender selbst, auf einen externen
Dateifreigabeserver oder lokal abgespeichert werden (vgl.~\cite{grafana}).

Zusammenfassung:

Positiv:

\begin{outline}
  \1 Authentisierung, Authentifizierung und Autorisierung
  \1 Gute Community und eine Vielzahl an Dokumentationen
  \1 Leichte Anpassungen an Graphen und Einstellungen möglich
  \1 Dashboards können auf externen Server oder lokal gesichert werden
  \1 API-Unterstüzung
  \1 Plugin-Erweiterung möglich (Java)
  \1 Open-Source-Softwarelösung
\end{outline}

Negativ:

\begin{outline}
  \1 Plugin-Erweiterung nur in Programmiersprache Java möglich
  \1 Webdesign kann nicht angepasst werden
  \1 Nur zwei Webtemplates verfügbar (dark/white)
\end{outline}
\mr%

\subsubsection{Graphite}
\label{subsubsec:graphite}
Die Softwarelösung Graphite wurde von Chris Davis im Jahr 2006 entwickelt und
gehört zu einer der am meisten eingesetzten Softwarelösungen im Bereich
Monitoring und Visualisierung. Graphite ist wie auch Grafana eine
Open-Source-Lösung und ermöglicht es dem Anwender, die Software bearbeiten und
anpassen zu können. Es verfügt über eine inhaltlich gute und umfangreiche
Community mit einem eigenem Wiki-Board, wo alle relevanten Dokumentationen
enthalten sind. Graphen können leicht mittels Statements über eine
API~\ref{sec:schnittstelle_datenbereitstellung} Datenabfrage oder direkt an das
Datenhaltungssystem durchgeführt werden. Eine Anmeldung an der WebOberfläche
mittels Benutzername und Kennwort oder \gls{LDAP} Anbindung ist ebenfalls
möglich. Es können einzelne Graphen, wie auch das gesamte Dashboard, auf
externe Server oder lokal abgespeichert werden, um einem Verlust von einzelnen
Graphen vorbeugen zu können. Graphite ist, wie auch
Grafana~\ref{subsubsec:grafana}, kompatibel zu Timeseries
Datenhaltungssystemen. Es bietet keine einfache Möglichkeit, weitere Plugins
hinzufügen zu können, wie es beispielsweise Grafana~\ref{subsubsec:grafana}
kann. Hierzu muss der Quellcode von Graphite massiv angepasst werden, wodurch
gute Kenntniss der einzelnen Dienste erforderlich sind. Graphite ist ein
System, welches aus mehreren Diensten besteht. Der Graphite-Webdienst ist für
die Anzeige der Weboberfläche zuständig. Der Carbon-Cache-Dienst wird
verwendet, um die Daten, welche vom Datenerfassungsystem ermittelt worden sind,
aufzuarbeiten. Dies wird benötigt, damit Graphite diese für die Visualisierung
verwenden kann. Grundsätzlich installiert sich mit der Installation von
Graphite noch eine Whisper-Datenbank, welche jedoch durch eine beliebige, wie
OpenTSDB~\ref{subsubsec:opentsdb} oder InfluxDB, ausgetauscht werden
kann (vgl.~\cite{graphite}).

Zusammenfassung:

Positiv:

\begin{outline}
  \1 Authentisierung, Authentifizierung und Autorisierung
  \1 Gute Community und eine vielzahl an Dokumenten innerhalb eines Wiki
  \1 API-Unterstützung
  \1 Dashboards und einzelne Graphen können auf externen Servern oder lokal
  abgespeichert werden
  \1 Open-Source-Softwarelösung
  \1 Dienste können hochverfügbar und auf unterschiedliche Server aufgeteilt
  werden
\end{outline}

Negativ:

\begin{outline}
  \1 Keine Möglichkeit, leicht Plugins hinzuzufügen
  \1 Webdesign kann nicht angepasst werden
  \1 Besteht aus mehreren Diensten, wie graphite-web und carbon-cache
\end{outline}
\mr%

\subsubsection{Zabbix-Frontend}
\label{subsubsec:zabbix-frontend}
Zabbix-Frontend ist eine von Zabbix LCC entwickelte Softwarelösung, welche dem
User/Administrator eine leichte und einfache Konfiguration von einzelnen
Graphen ermöglicht. Es bietet zudem eine inhaltlich gute Wissensdatenbank, in
der viele Dokumente zur Konfiguration von einzelnen Funktionen bis hin zur
Erläuterungen von jedem Button enthalten sind. Das Zabbix-Frontend benötigt
keinerlei Plugins, da hier bereits viele Funktionen in der eigentlichen
Software enthalten sind. Plugins können dennoch in der Programmiersprache Java
entwickelt und implementiert werden. Es ist darauf ausgelegt bei verschiedenen
Events, beispielsweise das Erreichen eines kritischen Festplattenstands, eine
E-Mail an den Administrator oder weitere User zu versenden, weshalb Zabbix sehr
häufig gerade in großen Infrastrukturen zum Einsatz kommt. Die Anmeldung an der
Weboberfläche kann wahlweise mit einem lokalen User mit Benutzername und
Kennwort oder mit einer LDAP-Anbindung durchgeführt werden. So benötigt der
Administrator beziehungsweise der User lediglich seine
Active-Directory-Benutzerdaten zur Authentifizierung. Zabbix ist zudem mit
API~\ref{sec:schnittstelle_datenbereitstellung} Abfragen kompatibel. So können
hier Graphen mit API-Abfragen visualisiert werden. Sollte keine Abfrage über
eine API gewünscht sein, so kann dies ebenfalls mittels Statement direkt auf
dem Datenhaltungssystem durchgeführt werden. Um die Datensicherheit und
Ausfallsicherheit gewährleisten zu können, bietet Zabbix die Möglichkeit,
einzelne Graphen mit Statements, wie auch das gesamte Dashboard exportieren und
importieren zu können (vgl.~\cite{zabbix-frontend}).


Zusammenfassung:

Positiv:

\begin{outline}
  \1 Authentisierung, Authentifizierung und Autorisierung
  \1 Gute inhaltliche Community und eine Vielzahl an Dokumenten innerhalb
  eines Wikis
  \1 Leichte Anpassungen an Graphen und Einstellungen möglich
  \1 API-Unterstützung
  \1 Dashboards und einzelne Graphen können auf externe Servern oder lokal
  abgespeichert werden
  \1 Bringt bereits eine Softwarelösung mit, womit Daten von virtuellen
  Maschinen exportiert werden können
  \1 Benötigt keine Plugins
  \1 E-Mail-Benachrichtigung bei bestimmten vordefinierten oder
  selbstdefinierten Events
  \1 Open-Source-Softwarelösung
\end{outline}

Negativ:

\begin{outline}
  \1 Keine Möglichkeit, das Webdesign anzupassen
\end{outline}
\mr%

\subsubsection{Datadog}
\label{subsubsec:datadog}
Datadog ist eine von Oliver Pomel entwickelte Analysesoftware und wird bereits
in großen Firmen, wie zum Beispiel EA oder Citrix erfolgreich eingesetzt. Es
ist nicht wie die anderen Software-Lösungen ein Open-Source-Produkt und wird in
verschiedenen Lizenzen von der Anzahl der entsprechenden virtuellen Instanzen,
welche abgegriffen werden sollen, gestaffelt. Die Basis-Lizenz gibt es hier
bereits kostenlos, jedoch können hier nur maximal 5 virtuelle Instanzen
abgefragt werden und die Daten werden maximal einen Tag zurückgehalten. Die
teuerste Lizenz kostet hier 23 Dollar und beinhaltet alle Hosts, welche der
Kunde anbinden möchte. Zusätzlich bietet diese Lizenz einen E-Mail-, Chat- oder
Telefonsupport. Dieser ist nicht in der Basis-Lizenz vorhanden. Graphen und
auch Dashboards können in Datadog einfach vom Administrator erstellt werden. Um
die Anmeldung an der Weboberfläche durchführen zu können, wird ein Benutzer und
Kennwort benötigt. Es ist hier nicht möglich, ein LDAP an die Weboberfläche
anbinden zu können. Innerhalb eines Graphen können bestimmte Teilabschnitte mit
zum Beispiel Social-Media-Plattformen wie Facebook oder Twitter geteilt werden.
Datadog bietet zudem eventbasierte Mitteilungen, welche vom Administrator
festgelegt werden können. Datadog hat eine vollständige
API~\ref{sec:schnittstelle_datenbereitstellung} Unterstützung und Dashboards
können vom Administrator exportiert werden. Eine Möglichkeit, Graphen zu
exportieren, bietet Datadog hier nicht. Die Community, in der viele inhaltliche
Dokumente enthalten sind, sind hier nur für Kunden bestimmt, welche eine
Pro-Lizenz besitzen. Ebenso gibt es hier einen \gls{IRC-Channel}, um direkt
Kontakt mit dem Support aufnehmen zu können (vgl.~\cite{datadog}).

Zusammenfassung:

Positiv:

\begin{outline}
  \1 Authentisierung, Authentifizierung, Autorisierung
  \1 Leichte Anpassung an Graphen, Dashboard und Einstellungen möglich
  \1 API-Unterstützung
  \1 Gute inhaltliche Community mit IRC-Channel und Telefonsupport (steht
  nur in der Kauflizenz zur Verfügung)
  \1 Benötigt keine zusätzlichen Plugins
  \1 Teilabschnitte von Graphen können in Social-Media-Plattformen geteilt
  werden
\end{outline}

Negativ:

\begin{outline}
  \1 Keine Open-Source-Lizenz. Lizenzen sind hier nach Anwendungsfall
  gestaffelt
  \1 Keine Möglichkeit der Anbindung an einen externen Verzeichnisdienst zur
  Benutzerauthentifizierung.
\end{outline}
\mr%

\subsubsection{Gridster-D3}
\label{subsubsec:gridster-d3}
Gridster ist eine von Anmol Koul entwickelte Java-Bibliothek, welche auf
\gls{GitHub} einsehbar und dokumentiert ist. Während der Evaluierung von dieser
Software, stellte sich heraus, dass es sich hierbei nicht direkt um eine
Visualisierungssoftware handelt, sondern lediglich um eine Funktion, nämlich
das Drag \& Drop von einzelnen Graphen oder Elementen innerhalb einer
Weboberfläche. Mit dieser hat man die Möglichkeit, ein Dashboard, welches
selbst entwickelt worden ist, individuell anpassen zu
können (vgl.~\cite{gridster-d3}).

Somit fällt Gridster-D3, da es sich hierbei nicht um eine
Visualisierungssoftware handelt, aus der weiteren Evaluierung heraus.

\subsubsection{Netdata}
\label{subsubsec:netdata}
Netdata ist eine Open-Source-Monitoring-Softwarelösung. Es wurde von Costa
Tsaousis entwickelt und bietet eine umfangreiche Möglichkeit, Daten in Form von
Graphen oder anderen Visualisierungen darzustellen. Der vollständige Code steht
auf GitHub, worüber auch die Community gepflegt wird. Andere Mitglieder der
GitHub-Community haben hier die Möglichkeit, Quelltext anzupassen und zu
verändern, jedoch müssen diese von anderen Mitgliedern bestätigt werden. Es
bietet nicht die Möglichkeit, eine Authentifizierung auf der Weboberfläche
durchzuführen. Ebenso werden alle Graphen, welche abgefragt werden, auf einer
Seite angezeigt, weshalb die Übersicht stark eingeschränkt ist. Graphen können
bei Netdata nicht einfach angepasst werden, da diese im Quellcode verändert und
angepasst werden müssen. Das Netdata läuft ebenfalls nur auf Linux-Oberflächen.
Ein Support von Microsoft Windows ist hier unterstützt. Ein klarer Nachteil ist
allerdings, dass diese Softwarelösung jeweils nur eine Instanz gleichzeitig
überwachen kann und der Vorteil bei Netdata, dass kein weiteres
Datenhaltungssystem oder ähnliches benötigt wird (vgl.~\cite{netdata}).

Zusammenfassung:

Positiv:

\begin{outline}
  \1 Es werden sämtliche Daten unmittelbar ohne Datenbank bereits visualisiert
  \1 Gute inhaltliche Community
  \1 Vollständige Open-Source-Softwarelösung
\end{outline}

Negativ:

\begin{outline}
  \1 Sehr unübersichtlich vom Aufbau der Weboberfläche
  \1 Authentisierung, Authentifizierung und Autorisierung kann nicht
  gewährleistet werden
  \1 Bietet keine API-Unterstützung
  \1 Kann keine zusätzliche Datenbank ansprechen, somit ist eine Datensicherung
  nicht gewährleistet
\end{outline}
\mr%

\subsection{Abschluss der Evaluierung}
\label{subsec:abschluss_evaluierung}
Nach Abschluss der Evaluierung der einzelnen Softwarelösungen für die
Datenvisualisierung wurden diesen während der Evaluierung entsprechende
Bewertungspunkte für die Kriterien von Null bis Fünf festgelegt. Damit eine
Softwarelösung in die nähere Auswahl aufgenommen werden kann, muss es
mindestens 3,00 Gesamtpunkte in der Nutzwertanalyse erhalten. In
nachfolgender Tabelle, kann hier die Nutzwertanalyse der Softwarelösungen
Datadog, Gridster-D3 und Netdata entnommen werden.

\begin{table}[H]
\resizebox{\textwidth}{!}{%
\begin{tabular}{@{}lccccccc@{}}
\toprule
\multicolumn{1}{c}{Kriterien} & Gewicht & \multicolumn{2}{c}{Datadog} &
\multicolumn{2}{c}{Gridster-D3} & \multicolumn{2}{c}{Netdata} \\
\midrule
 &  & Bewertung & Gesamt & Bewertung & Gesamt & Bewertung & Gesamt \\
Skalierbar & 25\% & 3 & 0,75 & 0 & 0,00 & 2 & 0,50 \\
Kosten & 20\% & 1 & 0,20 & 0 & 0,00 & 5 & 1,00 \\
Sicherheit & 30\% & 3 & 0,90 & 0 & 0,00 & 1 & 0,30 \\
Performance & 15\% & 4 & 0,60 & 0 & 0,00 & 1 & 0,15 \\
Dokumentation & 10\% & 5 & 0,50 & 0 & 0,00 & 4 & 0,40 \\
\midrule
Gesamt & 100\% &  & 2,95 &  & 0,00 &  & 2,35
\end{tabular}%
}
\caption{Nutzwertanalyse Datadog, Gridster-D3 und Netdata}
\label{nwa_dgn}
\end{table}

In der oben stehenden Tabelle, erhält die Softwarelösung
Datadog~\ref{subsubsec:datadog} nur 2,95 Gesamtpunkte, da diese Softwarelösung
eine Kostenpflichtige Lizenz benötigt. Dies ist innerhalb des Projektes ein
Kernkriterium, weshalb es als Softwarelösung nicht weiter verwendet.
Gridster-D3~\ref{subsubsec:gridster-d3} ist bereits während der
Evaluierungsphase ausgeschieden, da es sich hierbei nicht direkt um eine
Softwarelösung handelt sondern lediglich um eine Funktion. Aus diesem Grund,
hat Gridster-D3 0,00 Gesamtpunkte in der Nutzwertanalyse erhalten. Im laufe der
Evaluierung hat Netdata~\ref{subsubsec:netdata} 2,35 Gesamtpunkte erhalten, da
es sehr unübersichtlich aufgebaut ist und keine Sicherheit gewährleisten kann.
Alle drei Softwärelösungen sind damit ausgeschieden, da eine mindest
Gesamtpunktzahl von 3,00 nicht erreicht worden ist.

Nachfolgend bleiben folgende drei Datenvisualisierungslösungen für die nähere
Auswahl:

\begin{outline}
  \1 Grafana
  \1 Graphite
  \1 Zabbix-Frontend
\end{outline}

Ebenfalls wie auch bei den Softwarelösungen Datadog, Gridster-D3 und Netdata,
wurde auch hierfür eine Nutzwertanalyse durchgeführt, die wie folgt aussieht:

\begin{table}[H]
\resizebox{\textwidth}{!}{%
\begin{tabular}{@{}lccccccc@{}}
\toprule
\multicolumn{1}{c}{Kriterien} & Gewicht & \multicolumn{2}{c}{Grafana} &
\multicolumn{2}{c}{Graphite} & \multicolumn{2}{c}{Zabbix-Frontend} \\
\midrule
 &  & Bewertung & Gesamt & Bewertung & Gesamt & Bewertung & Gesamt \\
Skalierbar & 25\% & 4 & 1,00 & 3 & 0,75 & 2 & 0,50 \\
Kosten & 20\% & 5 & 1,00 & 5 & 1,00 & 5 & 1,00 \\
Sicherheit & 30\% & 4 & 1,20 & 4 & 1,20 & 4 & 1,20 \\
Performance & 15\% & 3 & 0,45 & 2 & 0,30 & 2 & 0,30 \\
Dokumentation & 10\% & 5 & 0,50 & 5 & 0,50 & 5 & 0,50 \\
\midrule
Gesamt & 100\% &  & 4,15 &  & 3,75 &  & 3,50
\end{tabular}%
}
\caption{Nutzwertanalyse Grafana, Graphite und Zabbix-Frontend}
\label{nwa_ggz}
\end{table}

Nachdem die Nutzwertanalyse für Grafana, Graphite und Zabbix-Frontend
durchgeführt worden ist, hat die
Zabbix-Frontend~\ref{subsubsec:zabbix-frontend} Softwarelösung eine
Gesamtpunktzahl von 3,50 erhalten. Während der Evaulierung der Zabbix-Frontend
Softwarelösung, fiel auf, dass das Zabbix-Frontend ohne einen Zabbix-Server
nicht verwendet werden kann. Der Zabbix-Server ist ein von Zabbix LLC
entwickelter Netzwerkdienst. Er empfängt Metriken von Agents, bereitet diese
auf und speichert diese Daten anschließend in einem Datenhaltungssystem ab.
Zusätzlich ist aufgefallen, dass das Zabbix-Frontend maximal nur mit einer
Datenbank gleichzeitig kommunizieren und verwendet werden kann. Anders als bei
Grafana, welches mit mehreren Datenbanken gleichzeitig kommunizieren kann.
Innerhalb der einzelnen Graphen werden hier mittels einer Abfrage die
Datenbanken direkt angesprochen. Bei der Evaluierung der
Graphite~\ref{subsubsec:graphite} Softwarelösung, hat Graphite 3,75
Gesamtpunkte erhalten.  Während der Implementierung von Graphite fiel auf, dass
es deutlich komplexer ist als Grafana. Graphite besteht aus mehreren Diensten,
welche bei der Installation konfiguriert und angepasst werden müssen, damit
Graphite funktional ist. Wenn ein Dienst ausfallen sollte, so betrifft dies das
gesamte Graphite-System und die Weboberfläche wäre nicht mehr erreichbar.
Zusätzlich ist hier gerade in der Wartung der Dienste ein Mehraufwand, da der
Administrator alle Dienste in Form von Abhängigkeit zueinander und
Konfiguration kennen muss. Da Graphite aus mehreren Diensten besteht, benötigt
Graphite auch mehr Systemressourcen als ein einzelner Dienst, wie es bei
Grafana der Fall ist. Hierdurch bedingt, wird Graphite für die weitere
Implementierung nicht verwendet. Grafana~\ref{subsubsec:grafana} hat in der
Nutzwertanalyse die meisten Gesamtpunkte mit 4,15 erhalten. Es erfüllt alle von
den Projektmitgliedern zuvor definierte Kriterien.
\mr%

\chapter{Userstories}
Im Punkt~\ref{sec:agile_vorgehensweise} wurde bereits kurz auf Userstories
eingegangen. Im Folgenden wird die generelle Vorgehensweise erklärt, sowie die
Implementierung und Realisierung der Userstories in diesem Projekt.

Zu jeder Userstory wird ein Drahtgittermodell (Englisch: Wireframe) erstellt.
Dies ist eine schematische Zeichnung der angefragten Änderung. Bei Änderungen
an einer Webseite wird eine Nachbildung der Seite gezeichnet, bei
Datenbankänderungen eine Tabelle. Im Anschluss darauf erfolgt die Zerlegung
(Englisch: Decomposition) des Wireframes in die einzelnen
Informationsbestandteile. Als Beispiel kann angenommen werden, dass eine
Userstory einen zusätzlichen Graphen auf einer Webseite anfordert.
Ausschlaggebend für die Zerlegung sind folgende Fragen:

\begin{outline}
  \1 Werden zusätzliche Daten benötigt, um die Userstory zu implementieren oder
  reichen die vorhandenen Daten, die bereits auf der Webseite genutzt werden?
  \1 Wo kommen diese Daten her?
  \1 Wer liefert die Daten bzw.\ wird dafür eine Genehmigung benötigt?
  \1 Werden bereits genutzte Daten nicht mehr benötigt?
  \1 Ist die Änderung atomar oder kann sie weiter zerlegt werden?
\end{outline}

Daraus ergeben sich mehrere Änderungen, die am Projekt gemacht werden müssen,
um die Userstory zu implementieren. Für jede Änderung wird ein Ticket erstellt.
Ziel des Ganzen ist es, eine komplexe Userstory, die einen einzelnen erfahrenen
Entwickler lange mit der Implementierung beschäftigt, möglichst weit
herunterzubrechen. Die entstandenen Teilaufgaben erfordern größtenteils keinen
erfahrenen Entwickler. Dieser ist nur noch für einige, wenige Tickets
notwendig.  Ein weiterer Vorteil ist, dass nun mehrere Leute parallel an den
einzelnen Tickets arbeiten können.
\tm%

\section{SSD Userstory}
\textbf{Name der Story:} Der Administrator möchte eine Anzeige der SSDs, welche
einen Hardwaredefekt haben könnten.

\textbf{Beschreibung:} In der heutigen Zeit werden vermehrt langsame
Hard Disk Drives durch schnellere Solid State Disks ausgetauscht. Gerade im
Serverbetrieb sorgt dies für einen schnelleren Datenzugriff. SSDs haben
jedoch, wie auch HDDs, einen Lebenszyklus, der bei SSDs schneller erreicht
werden kann, als bei handelsüblichen HDDs. Der Administrator soll hier darüber
benachrichtigt werden, sobald von einer SSD ein kritischer Schreib- oder
Lesezyklus erreicht worden ist, um die SSD bereits vor dem Ausfall austauschen
zu können. Ebenfalls soll hier auch eine Statistik ausgegeben werden, welche
SSD von welchem Hersteller in der Wahrscheinlichkeit am häufigsten ausfallen
könne. Im Folgenden finden sich die Akzeptanzkriterien:

\begin{outline}
  \1 Die Ausgabe und Visualisierung in einem Balken- oder Graphendiagramm
  für jede einzelne Instanz
  \1 Auflistung einer Statistik in Form eines Balkendiagramms über die
  Hersteller, welche am wahrscheinlich häufigsten in den Instanzen ausfallen
  \1 Eine Benachrichtigung, ob per E-Mail oder auf anderen Kommunikationswegen,
  wenn ein kritischer Schreib- oder Lesezyklus erreicht worden ist
\end{outline}

Das dazugehörige Wireframe, kann dem Anhang~\ref{figure:ssduserstory} entnommen
werden.
\mr%

\section{CPU Userstory}
\label{section:CPU_Userstory}
\textbf{Name der Story:} Der Administrator möchte eine Auswertung der
CPU-Auslastung jeder Instanz, um Engpässe zu erkennen.

\textbf{Beschreibung:} Die CPUs der heutigen Zeit werden in der Taktrate und
Verarbeitungsgeschwindigkeit der Prozesse immer schneller und besser. Bei
virtuellen Maschinen werden aus einem physischen CPU-Kern des Host-Systems
mehrere virtuelle CPU-Kerne emuliert und einer Instanz zugewiesen. So kann eine
VM 2 virtuelle CPU-Kerne mit z.B. 2,5\si{\giga\hertz} Taktrate zugewiesen
haben. Auf jeder Instanz laufen andere Dienste, weshalb jede VM
unterschiedliche CPUs zugeteilt haben kann. Hier kann es vorkommen, dass eine
VM mehr CPU-Leistung benötigt, als ihr selbst zugewiesen ist. Um dies
frühestmöglich zu erkennen, benötigt der Administrator eine detaillierte
Übersicht in Form eines Graphen, wie viel CPU-Leistung von einer virtuellen
Instanz verwendet wird und auf dem System verfügbar ist. Im Folgenden finden
sich die Akzeptanzkriterien:

\begin{outline}
  \1 Visualisierung der Auslastung und Verfügbarkeit der CPU-Leistung für jede
  einzelne Instanz
  \1 Alle Werte der CPU, \gls{Soft-IRQ},\gls{iowait} oder \gls{Interrupt}
  müssen visualisiert werden
  \1 Eine frühestmögliche Erkennung von einem Engpass einer virtuellen Maschine
\end{outline}

Das dazugehörige Wireframe, kann dem Anhang~\ref{figure:cpuuserstory} entnommen
werden.
\mr%

\section{Memory Userstory}
\textbf{Name der Story:} Der Administrator möchte eine Auswertung der
Arbeitsspeicherauslastung von jeder Instanz, um Engpässe zu erkennen.

\textbf{Beschreibung:} Vergleichbar wie auch bei der CPU-Auslastung, gibt es
auch eine ähnliche Thematik zur Auslastung bei dem verwendeten Arbeitsspeicher.
Meist ist hier jedoch nicht die Taktrate des Arbeitsspeichers entscheidend,
sondern eher wie viel Arbeitsspeicher einer virtuellen Maschine zugewiesen ist.
Hier kann es ebenfalls wie auch bei der~\ref{section:CPU_Userstory}
vorkommen, dass es zu Engpässen auf der virtuellen Maschine bei der Verteilung
des Arbeitsspeichers kommen kann. Im Folgenden finden sich die
Akzeptanzkriterien:

\begin{outline}
  \1 Ausgabe und Visualisierung in Form eines Graphen für den verwendeten,
  freien und zwischengespeicherten Arbeitsspeicher für jede einzelne virtuelle
  Maschine
  \1 Eine frühestmögliche Erkennung von einem Engpass des Arbeitsspeichers
  einer virtuellen Maschine
\end{outline}

Das dazugehörige Wireframe, kann dem Anhang~\ref{figure:memoryuserstory}
entnommen werden.
\mr%

\section{Zeitdefinierte Analyse Userstory}
\textbf{Name der Story:} Der Administrator möchte eine Angabe über die im
vergangenen Jahr verwendeten Ressourcen einer virtuellen Maschine.

\textbf{Beschreibung:} Ein Onlineshop für zum Beispiel saisonale Ware, wie
Adventskalender, benötigt eine Analyse über die verwendeten Ressourcen aus dem
vergangenen Jahr, um hier im kommenden Jahr die Ressourcen bereits bei Beginn
richtig skalieren zu können. Dies dient in erster Linie, um die Erreichbarkeit
des Onlineshop gewährleisten zu können, wie auch die Performance im Allgemeinen
zu verbessern. So können zum Beispiel bestimmten virtuellen Maschinen zu einem
gewissen Zeitraum mehr Ressourcen zur Verfügung gestellt werden, damit diese
die Ressourcen verwenden können. In den Monaten in dem keine große Nachfrage
der Webseite besteht, können hier dann weniger Ressourcen vergeben werden. Dies
hat den Vorteil, dass spezielle Vertäge dem Kunden angeboten werden können. Im
Folgenden finden sich die Akzeptanzkriterien:

\begin{outline}
  \1 Es soll eine Ausgabe in visualisierter Form über einen bestimmten Monat
  dargestellt werden
  \1 Ressourcen, welche benötigt werden, können direkt bereits ab Beginn
  zugewiesen werden
\end{outline}

Das dazugehörige Wireframe, kann dem Anhang~\ref{figure:timeframeuserstory}
entnommen werden.
\mr%

\section{Weboberfläche Userstory}
\textbf{Name der Story:} Der Administrator möchte die Visualisierung der
erhaltenen Daten zur weiteren Verwendung.

\textbf{Beschreibung:} Dem Administrator oder dem User soll es ermöglicht
werden, eine visualisierte Ausgabe in Form von Graphen zu erhalten. Diese
sollen Aufschluss über die Auslastung der einzelnen Systeme geben können. Im
Folgenden finden sich die Akzeptanzkriterien:

\begin{outline}
  \1 Visualisierung erfolgt für jeden einzelnen abgegriffenen Wert
  \1 Zeitspanne der Visualisierung kann über den Benutzer angepasst werden
  \1 Die Graphen, welche visualisiert wurden, können vom Benutzer individuell
  angepasst werden
\end{outline}

Das dazugehörige Wireframe, kann dem Anhang~\ref{figure:uiuserstory}
entnommen werden.
\mr%

\chapter{Realisierung}
In der Evaluierungsphase wurde für jeden Bereich die optimale Software gesucht
und gefunden. Hier wurde mit Absicht jeder Bereich getrennt betrachtet. Es
wurde nicht geprüft, ob die gewählte Software in einem Bereich mit den
Alternativen in den anderen beiden Bereichen kompatibel ist. Aufgrund der hohen
Auswahl an Möglichkeiten pro Bereich gibt es zu viele
Kombinationsmöglichkeiten. Diese können zeitlich nicht alle im Rahmen des
Projektes evaluiert werden. Es wurde darauf geachtet, dass jede Software mit
offenen und bekannten Schnittstellen arbeiten kann. Diese werden in der
Realisierungsphase miteinander kombiniert. Alle arbeiten wurden auf dem
Betriebssystem Archlinux durchgeführt, da hier die meiste Erfahrung im Team
vorhanden ist. Um eine große Kompatibilität sicherzustellen wurden allerdings
alle Tests nicht nur für Archlinux, sondern auch für die Distributionen Debian,
Ubuntu und CentOS durchgeführt.

\section{Prototyp 1}
Entschieden wurde sich für Postgres als Datenbank, collectd zur Datenerfassung
und Grafana zur Visualisierung. Hinzu kommt eine selbst entwickelte API für
komplexe Abfragen und Fragestellungen. Die erste Idee der Umsetzung ist
unter~\ref{figure:draft1} visualisiert. Die Architektur ist in drei Teile
gegliedert. Im ersten Teil befinden sich alle Hypervisors und virtuellen
Maschinen, welche Daten via collectd verschicken. Jede Instanz von collectd
soll ihre Daten in der zentralen Postgres Datenbank speichern. Für Tests wird
mit einer einzelnen Postgres Datenbank gearbeitet, diese kann bei Bedarf um
weitere Server erweitert werden, um einen hochverfügbaren Betrieb zu
realisieren. Komplexe Datenbankabfragen über die API können, gerade bei großen
Datenmengen, eine unerwartet hohe Last erzeugen. Es ist möglich, dass in dieser
Zeit eingehende Daten von collectd nicht direkt gespeichert werden können. Die
Daten würden dann verloren gehen. Sofern die Postgres Installation nicht
hochverfügbar ausgelegt ist, kann es auch zu einem Datenverlust bei regulären
Wartungsarbeiten oder Ausfällen an dem Datenbankserver kommen. Um den genannten
Problemen entgegen zu wirken, kam die Idee auf, einen Message Bus (auch Message
Queue genannt) zu installieren.
\tm%

\section{Message Bus}
Ein Message Bus ist eine Applikationen, welche auf einem oder mehreren Servern
betrieben wird und als \gls{Middleware} agiert. Über diverse
Netzwerkschnittstellen ist es möglich, Daten auf diesen Bus zu senden. Jedes
Datenpaket kann mit Tags versehen werden. Ein Tag kann ein beliebiger Text
sein, zum Beispiel der Name des Quellservers, Uhrzeit an dem das Paket erstellt
wurde oder eine Retention Period.

Auf der anderen Seite des Busses können Applikationen über die gleichen
Schnittstellen auf die Daten zugreifen und diese dem Bus entnehmen. Hierbei
kann auf die Tags gefiltert werden, um nur bestimmte Daten zu entnehmen.

Im Worst Case (dem schlimmsten möglichen Zustand) ist es möglich, dass die
Datenbank für mehrere Stunden oder Tage nicht erreichbar ist. Hierzu
unterstützen Message Busse eine Retention Period. Dieser Wert beschreibt wie
lange ein Datensatz gültig ist. Wenn er zu alt ist, wird er automatisch
gelöscht und vom Bus genommen. Der Bus kann somit nicht über die Zeit
volllaufen. Im Projektteam gab es bereits Erfahrung mit der
Message-Bus-Software RabbitMQ, weshalb diese für den ersten Prototypen gewählt
wurde.

Der Messagebus wird zwischen collectd und der Datenbank installiert. Collectd
ist nicht in der Lage, direkt SQL zu sprechen. SQL ist allerdings die einzige
Schnittstelle, welche von Postgres bereitgestellt wird. Die Daten müssen also
transformiert werden. Hierfür bietet sich das bereits zuvor evaluierte Logstash
an (~\ref{subsec:logstash}). Logstash besitzt ein Plugin, um vom Bus zu lesen
(vgl~\cite{logstash-rabbitmq}). Zusätzlich existiert ein Plugin, um über JDBC
mit Datenbanken zu kommunizieren (vgl.~\cite{logstash-jdbc}).
\tm%

\section{JVM}
Die \gls{JVM} (Java Virtual Machine) ist eine virtuelle Maschine zum Ausführen
von Java Bytecode. Sie stellt eine abgekapselte Umgebung (Sandbox) bereit in
der der Code ausgeführt wird. Dies bringt nicht nur eine erhöhte
Systemsicherheit, da die Applikation in der JVM nicht direkt mit dem Hostsystem
kommunizieren kann. Es hilft auch beim Multithreading. Die JVM ist in der Lage,
beliebig viele Prozesse zu starten und zu verwalten.

Logstash selbst ist in Ruby geschrieben. Die Ausführung selbst erfolgt aber
innerhalb einer JVM\. Normalerweise wird Ruby Code mit dem \gls{MRI}
Interpreter in Maschinencode übersetzt und direkt auf dem Betriebssystem
ausgeführt. Der Interpreter JRuby übersetzt den Ruby Code in Java Bytecode,
dieser wird dann von der JVM in einem Prozess ausgeführt und nicht direkt auf
dem Betriebssystem. Dies ermöglicht es, Multithreading bei der Entwicklung des
Ruby Codes zu ignorieren. Hierfür muss keine Funktionalität programmiert
werden, da dies automatisch von der JVM übernommen wird. Ein normales Ruby
Script wird nur auf einem Prozessorkern ausgeführt. Ein einzelner Kern kann
aber die erwartete Datenmenge nicht aus dem Message Bus abfragen und
verarbeiten. Hier kann die Architektur stark von der JVM profitieren. Sie
kümmert sich darum, dass alle zur Verfügung stehenden CPU-Kerne belastet werden
und startet entsprechend viele Prozesse mit Logstash. Somit erhält man eine
sehr einfache Möglichkeit der Datenverarbeitung mit mehreren
Prozessorkernen (vgl.~\cite{jruby}).
\tm%

\section{JDBC}
\gls{JDBC} (Java Database Connectivity) ist die Schnittstelle der JVM zu
Datenbanken. Sie stellt generische Funktionen bereit, um mit Datenbanken zu
kommunizieren. Für jede Datenbank mit der kommuniziert werden soll, wird ein
Treiber benötigt. Dieser beschreibt die spezifischen Eigenschaften eines
bestimmten \gls{DBMS} für die JDBC\@. Er kommuniziert auf der einen Seite mit
dem DBMS und auf der anderen Seite mit der JDBC\@. Für Logstash gibt es ein
Plugin, welches die JDBC-Schnittstelle innerhalb des JRuby Bytecodes nutzbar
macht. Somit ist Logstash in der Lage, in eine Datenbank zu schreiben. Hierfür
musste ausschließlich die Kommunikation mit der JDBC programmiert werden.
Postgres bietet einen Treiber für JDBC an (vgl.~\cite{postgres-jdbc}).
\tm%

\subsection{JDBC Treiber Typen}
Es gibt vier verschiedene Typen von JDBC-Treibern. Typ vier ist besonders
effizient, dieser wird auch von den Postgres Entwicklern bereitgestellt. Im
Folgenden sind die einzelnen Typen erklärt (vgl.~\cite{jdbc-types}).

\subsection{JDBC-ODBC Treiber}
Hierbei spricht die JDBC-Schnittstelle einen sehr generischen \gls{ODBC}
Treiber an. Dieser kommuniziert mit einer Bibliothek des DBMS\@. Der
ODBC-Treiber muss auf jedem System vorhanden sein, welches die JVM-Applikation
ausführt. Die Bibliothek kann, zusammen mit der Datenbank selbst, auf einem
anderen System im Netzwerk oder auch lokal betrieben werden.

Die Vorteile sind:

\begin{outline}
  \1 Sehr einfache Implementierung der JDBC, da ein Großteil der Arbeit von
  der ODBC übernommen wird.
  \1 Sehr einfacher Austausch der ODBC, da die Schnittstelle generisch ist.
  \1 Jede Datenbank, die SQL nutzt, bietet eine ODBC-Schnittstelle, somit
  ergibt sich eine lange Liste an alternativen DBMS, welche genutzt werden
  können.
\end{outline}

Die Nachteile sind:

\begin{outline}
  \1 Durch die große Anzahl an Komponenten, durch die die Daten fließen,
  ergeben sich viele Fehlerquellen.
  \1 Da alle Schnittstellen sehr generisch sind, kann dort keine Optimierung
  auf ein spezifisches DBMS erfolgen.
\end{outline}

\subsection{Nativer API-Treiber}
Die JDBC-Schnittstelle spricht hier über den Treiber direkt die Bibliothek
des DBMS an. Dafür muss der Treiber als auch die Bibiliothek auf dem gleichen
System vorhanden sein, wie die JVM-Applikation. Die Datenbank selbst kann auf
dem gleichen Server oder über das Netzwerk angesprochen werden.

Die Vorteile im Vergleich zum JDBC-ODBC-Treiber:
\begin{outline}
  \1 Eine höhere Performance, da Teile direkt in Java geschrieben sind.
  \1 Geringere Fehleranfälligkeit, da eine Komponente weniger vorhanden ist.
\end{outline}

Die Performance ist besser als beim JDBC-ODBC-Treiber, aber noch nicht optimal.
Außerdem ist es nicht immer möglich, Zusatzsoftware, wie die Bibliotheken, auf
dem gewünschten System zu installieren. Es gibt Firmen, die dies untersagen.
Dies trifft allerdings nicht in diesem Projekt zu, weshalb es nicht als
Negativpunkt gewertet wird.

\subsection{Netzwerkprotokolltreiber}
Datenbankoperationen werden an eine weitere, auch in Java geschriebene,
Netzwerkapplikation geschickt (auch \gls{Middleware} genannt). Diese spricht
dann die Datenbank an, welche oft auf einem dritten System installiert ist.
Hier muss keine Drittsoftware auf der eigentlichen Applikation installiert
werden.

Die Vorteile sind:

\begin{outline}
  \1 Die Middleware kann sich dediziert um SQL kümmern und von Fachleuten auf
  dem Gebiet entwickelt werden.
  \1 Datenbanklogik muss nicht in der eigentlichen Applikation implementiert
  werden. Die Entwickler der Applikation können sich auf die Features der
  Applikation konzentrieren.
  \1 In der Middlerware kann sehr einfach Auditing erfolgen.
\end{outline}

Die Nachteile sind:

\begin{outline}
  \1 Die eigentliche Applikation benötigt zwingend eine Netzwerkfunktionalität.
  \1 Die Datenbankoperationen erfolgen über zwei Netzwerkverbindungen, was
  zu einem Overhead führt.
  \1 Der Entwicklungsaufwand ist bedeutend komplexer als bei anderen Treibern,
  sofern man die Middleware selbst implementiert und nicht auf fertige Lösungen
  zurückgreifen kann.
\end{outline}

\subsection{Nativer JDBC-Treiber}
Hier liefern die Datenbankentwickler einen nativen Treiber. Dies bedeutet, dass
er komplett in Java geschrieben ist. Er wird beim Starten der Applikation
geladen und erlaubt es der JDBC-Schnittstelle, direkt mit der Datenbank zu
kommunizieren. Es werden keine weiteren Bibliotheken benötigt. Die Datenbank
kann auf dem gleichen System wie die Applikation laufen oder auch über das
Netzwerk angesprochen werden.

Die Vorteile sind:

\begin{outline}
  \1 Dieser Treiber bietet die größtmögliche Performance, da er Anfragen von
  der JDBC auf das spezifische DBMS optimieren kann.
  \1 Der Overhead wurde auf ein Minimum reduziert.
  \1 Es werden keine zusätzlichen Schnittstellen benötigt, was die
  Fehleranfälligkeit reduziert.
\end{outline}

Diese Art von Treiber bietet aus Sicht des Nutzers keine bekannten Nachteile.
Die Entwicklung benötigt für diese Art die meiste Arbeit, weshalb nicht jedes
DBMS solche Treiber zur Verfügung stellt. Für Postgres existieren solche
Treiber. Logstash ist somit in der Lage, besonders effizient auf die Datenbank
zuzugreifen.
\tm%

\section{JDBC in Kombination mit der JVM}
Die Kombination aus JVM und JDBC bietet einen großen Vorteil für das Projekt.
Logstash ist in der Lage, die Auslastung der Datenbank über JDBC zu ermitteln.
Bei einer zu hohen Last werden die Anzahl der Prozesse limitiert was in weniger
Anfragen zur Datenbank resultiert. Sobald die Datenbank wieder eine geringere
Auslastung hat, kann die JVM eigenständig mehr Prozesse starten und die
Datenmenge erhöhen. Somit verhindert man Performanceprobleme auf der Datenbank
bei komplexen Abfragen, welche über die API gestellt werden.

Anstatt die Datenbank als limitierenden Faktor zu nutzen (dem Ziel von
Logstash), kann dies auch mit der Quelle erfolgen. Der Message Bus ist
in diesem Setup die Quelle für Logstash. Die JVM kann anhand der vorhandenen
Datenmenge im Bus beliebig viele Prozesse starten, um die Datenmenge möglichst
schnell abzufragen. Dies wäre in einem Szenario sinnvoll, in dem der Speicher
im Bus besonders gering ist und die Datenbank eine garantierte Leistung
erbringen kann. Dies ist nur möglich, wenn Logstash der einzige Dienst ist,
welcher mit der Datenbank interagiert. Dieses Szenario ist innerhalb des
Projektes nicht möglich, weshalb es nicht weiter verfolgt wurde bei der
Implementierung.
\tm%


\section{Prototyp 2}
Die Implementierung des Message Busses stellte sich als sehr komplex heraus.
RabbitMQ ist zwar die Software in dem Bereich mit der größten Verbreitung, aber
auch sehr komplex zu betreiben und aufzusetzen. Es war dem Projektteam nicht
möglich, RabbitMQ an collectd anzubinden. Hierfür sind diverse Einstellungen
über die RabbitMQ \gls{CLI} notwendig. Diese konnte sich nicht gegenüber dem
RabbitMQ Server authentifizieren. Es wurde keine sinnvolle Fehlermeldung
ausgeben. Die Entwickler von RabbitMQ konnten ebenfalls keine Lösung für das
Problem liefern. Außerdem ist dem Projektteam aufgefallen, dass der erste
Prototyp unnötig komplex ist. Das Konzept eines Message Busses passt sehr gut
in das Projekt, allerdings muss dafür keine eigenständige Software genutzt
werden. Während der Suche nach Alternativen fiel auf, dass auch Postgres als
Bus genutzt werden kann.

Bei der Analyse von Grafana hat sich herausgestellt, das dort keine granulare
Rechteverwaltung möglich ist, wenn direkt mit einer Datenbank interagiert wird.
Da die selbst entwickelte API bereits Endpunkte für Analysen bereitstellen
muss, kann diese simpel erweitert werden, um auch Daten an Grafana zu liefern.
Somit liefert nun die API Daten an Grafana und Grafana kann nicht mehr direkt
mit der Datenbank sprechen. Die API ist in der Lage, alle Anfragen von Grafana
passend zu filtern und zu validieren. Eine Visualisierung des Prototypen
findet sich unter~\ref{figure:draft2}.
\tm%

\section{Automatisierung mit Puppet}
Die Firma Puppet Inc.\ entwickelt die Software puppet. Es ist eine quelloffene
Konfigurationsmanagement- und Automatisierungssoftware. Sie erlaubt es in
einer \glslink{Deklarative Programmiersprache}{deklarativen} \gls{DSL}, den
gewünschten Systemzustand zu beschreiben. Die Software realisiert dies. In
Modulen wird der puppet Code gebündelt. Jedes Modul enthält Code, welcher zur
Automatisierung einer bestimmten Software benötigt wird. Über die Plattform
\url{https://forge.puppet.com/} können diese Module ausgetauscht werden. Es
existieren bereits Module zur Verwaltung von Postgres, collectd, Grafana und
Logstash. Jedes Modul hat seinen eigenen \gls{Namespace} und bündelt ein oder
mehrere Klassen. Informationen über ein Modul werden in der
\texttt{metadata.json} Datei gespeichert. Diese befindet sich in jedem Modul
und enthält folgende Informationen:

\begin{outline}
  \1 Name des Modules
  \1 Autor
  \1 Lizenz
  \1 Link zum Quellcode
  \1 Eine Kurzbeschreibung
  \1 Liste der unterstützten Betriebssysteme
  \1 Eine Liste an anderen Modulen, welche für den Betrieb dieses Moduls
  benötigt werden
  \1 Die unterstützten Puppetversionen
\end{outline}
\tm%

\subsection{Strukturierung im puppet Code}
In der DSL werden immer Ressourcen beschrieben. Es gibt verschiedene Typen von
Ressourcen. Die wichtigsten sind:

\begin{outline}
  \1 Package - Verwaltet die Installation und Deinstallation von Paketen
  \1 Service - Managt Dienste auf dem System
  \1 File - Erstellt konfigurationsdateien und Verzeichnisse mit passenden
  Rechten
\end{outline}

Eine einzelne Ressource benötigt nur sehr wenig Code. Jede Deklaration einer
Ressource besteht aus drei Teilen:

\begin{center}
   \inputminted{puppet}{../listings/basic-resource.txt}
   \captionof{listing}{Puppet package resource für htop}
   \label{lst:puppet-resource}
\end{center}

Erst wird der Typ genannt, in diesem Fall \texttt{package}. Darauf folgt ein
Name für die Deklaration und zum Schluss eine Liste an Parametern. Der Name der
Ressource muss einzigartig sein. In diesem Fall ist es der Name des Paketes,
welches in der Version 2.0.2-1 installiert wird. Puppet bietet 48 verschiedene
Typen. Diese sind fest in puppet verankert und werden von der Firma Puppet
gepflegt. Außerdem kann puppet mit eigenen Typen erweitert werden. Für jeden
Typen gibt es einen Provider. Dieser implementiert den Typen auf einer
bestimmten Plattform. Das Grafana-Modul besitzt eigene Typen, welche Dashboards
und Datenquellen verwalten. Die Konfiguration von eben jenen erfolgt nicht über
normale Textdateien, sondern über eine API oder Weboberfläche. Über die eigenen
Typen kann Puppet auch hiermit interagieren und nicht nur die Installation,
sondern auch die komplette Konfiguration automatisieren
(vgl.~\cite{puppet-resource-types}). Da puppet deklarativ und nicht imperativ
arbeitet, ist die Anordnung der Ressourcen egal. Der Interpreter analysiert
alle Ressourcen, erkennt automatisch Abhängigkeiten und sortiert diese zu einem
gerichteten Graphen. Die Ressourcen werden deshalb nicht in der Reihenfolge
abgearbeitet, in der sie notiert sind.

Dieses modulare Konzept ermöglicht es, puppet auf einer großen Anzahl von
verschiedenen Betriebssystemen und Hardwaretypen zu betreiben. Durch die
Abstraktion kann Code sehr häufig wiederverwendet werden.

In einem Puppet-Profil werden mehrere Module und Ressourcen deklariert, um eine
bestimmte Aufgabe zu realisieren. Dieses Profil ist aus Sicht des Codes eine
normale Klasse. Mehrere Profile werden in einem Modul gebündelt. Ein Profil
wird einem oder mehreren Nodes zugewiesen. Im Rahmen des Projekts werden
diverse Profile erzeugt, um die einzelnen benötigten Komponenten gebündelt zu
automatisieren. Ein beispielhaftes Profil für Grafana findet sich
unter~\ref{lst:grafana}. Dies installiert Grafana und richtet außerdem ein
Dashboard und eine Datenquelle ein.
\tm%

\subsection{Ablauf eines Puppet Runs}
Puppet kann nach dem \gls{Client-Server}-Prinzip arbeiten. Der Agent läuft
dabei auf jedem System, welches verwaltet werden soll. Unterstützt werden neben
Linux-Distributionen auch Windows, BSD, sowie diverse Hersteller von
Netzwerkequipment. Periodisch ermittelt der Agent \glslink{Fact}{Facts} und
sendet diese über eine \gls{HTTPS}-Verbindung zum Puppetserver. Anhand dieser
Facts wird entschieden, welche Module zu einem Katalog zusammenkompiliert
werden. Der Katalog wird dem Agent zurückgeschickt. Dieser analysiert sein
vorhandenes System und gleicht es mit den Beschreibungen im Katalog ab. Sobald
es hier zu Differenzen kommt, wird das System dem Katalog angeglichen. Jede
durchgeführte Aktion kann reportet werden. Reports können lokal gespeichert
oder zu einer zentralen Datenbank gesendet werden. Die Firma Puppet bietet
außerdem die quelloffene PuppetDB an. Diese kann Reports speichern. Über eine
\gls{RESTful} API kann man alle Reports analysieren und Abfragen. Zusätzlich
können alle Facts dort gespeichert werden. Ein „Puppet Run“ beschreibt den
kompletten Ablauf vom Auslesen der Facts bis zum Erstellen des Reports.

Puppet kann auch ohne Puppetserver genutzt werden. Hier erfolgt der komplette
Ablauf dann lokal. Die Module müssen auf dem Server vorhanden sein auf dem
auch der Agent läuft. Er kompiliert sich dann seinen Katalog selbst. Diese
Vorgehensweise bietet sich gerade während der Entwicklung an. In großen
Produktivumgebungen mit vielen Modulen ist es oftmals sinnvoller, Puppetserver
zu nutzen und nach dem Client-Server Prinzip zu arbeiten.
\tm%

\section{Postgres}
Für die Implementierung der Postgres Datenbank wurde zunächst ein einzelner
Server für die Bearbeitung der Anforderung aufgebaut und mit vier Nutzern
ausgestattet. Ersterer ist der Administrator, oder auch root genannt, welcher
nur genutzt wird um Änderungen (Punkt~\ref{subsec:postgres_partition}) am
System vorzunehmen. Mit diesem werden ebenfalls die Zugriffsrechte der
einzelnen Nutzer geregelt. Der zweite Nutzer besitzt die Berechtigungen in
einem definierten Bereich innerhalb des Postgres-Systems die
Daten und dessen Struktur zu managen. Mit diesem Benutzer wurden z.B. die
Prozesse aus Punkt~\ref{subsec:postgres_etl} und
Punkt~\ref{subsec:postgres_datenschutz} durchgeführt. Dritter Nutzer wird
von dem in Punkt~\ref{sec:api_system} definierte API-System verwendet. Dieser
besitzt keinen direkten Lese- oder Schreibezugriffe auf die in dem definierten
Datentopf vorhandenen Tabellen. Er erhält lediglich ausführende Rechte auf die
im Punkt~\ref{subsec:postgres_datenschutz} erstellten Funktionen. Vierter
Nutzer wird für den im Punkt~\ref{subsec:postgres_etl} beschriebenen ETL-Prozess
im Wirkbetrieb genutzt. Er besitzt ausschließlich schreibende Rechte.

Neben den durchgeführten Aufgaben an dem Postgres-Server wurde bei
den Teammitgliedern jeweils die benötigte Entwicklerumgebung aufgesetzt
und eingerichtet. Mit dieser und zusammen mit einem berechtigten Nutzer,
können die Teammitglieder an den Inhalten der Datenbank arbeiten.
\nl%

\subsection{ETL}
\label{subsec:postgres_etl}
Durch das fehlschlagen der Anbindung eines Message Busses an die Datenerfassung
musste eine alternative Möglichkeit der Informationsintegration zwischen der
Datenhaltung und Datenerfassung entwickelt werden. Dazu wurde ein vollständig
eigener Extraktions-, Transformations- und Ladeprozess (ETL) in das System
implementiert. Die konkrete Umsetzung erfolgte durch das Hinzufügen von einem
eigenem Arbeitsbereich zwischen der Datenquelle und der Zieldatenbank.

Dabei findet der Extraktions-Teil zwischen dem erstellten Arbeitsbereich und
der Datenquelle statt. Es werden dabei die neuen Daten in ein allgemeines
Datenobjekt eingefügt und kurzzeitig zwischengespeichert. Dieses Datenobjekt
hat dabei keinerlei Anforderungen an die gelieferten Daten, und nimmt sie
zunächst einfach auf, sodass diese nicht verloren gehen. Sollten dabei
Daten nicht vollständig übermittelt werden oder während der Übertragung die
Datenquelle ausfallen, beziehungsweise andere Fehler entstehen, ist dies
zunächst für den eigentlichen Wirkbetrieb, also das Halten der vollständigen
Daten und dem Ausliefern der Daten an andere Systeme/API, nicht relevant.
Durch die Trennung der Systeme, beziehungsweise den einzelnen Aufgabenfeldern,
ist nur der Verlust von neuen Daten das größte Risiko. Ein Systemabsturz durch
dauerhafte Fehler beim Datenschreiben ist somit vermieden.

Sollten nun Daten in dem genannten allgemeinen Datenobjekt vorhanden sein,
beginnt der Transformations-Teil. Dabei greift ein weiterer Algorithmus auf
die zwischengespeicherten Daten zu und vergleicht die Inhalte von jedem neuen
Datensatz mit einem \gls{Repositorium}, beziehungsweise einem von dem
Projektteam definierten Metadatensatz. Dieser dient dazu, eine spezifische
Definition von allen in dem Datenhaltsungssystem vorhandenen Daten zu besitzen.
Er wird entsprechend als Dokumentationsbasis genutzt, sodass ein Nutzer genau
nachvollziehen kann, in welcher Struktur und an welchem Ort die Daten in dem
Datenhaltsungssystem abgespeichert sind. Gleichzeitig besitzt das Projektteam
und der spätere Administrator im Wirkbetrieb die Möglichkeit den Dateninput
über den ETL-Prozess zu kontrollieren. Darüber können zum Beispeil die Länge
von einzelnen Attributen, oder der Datentyp angepasst werden. Es kann jedoch
sogar tiefer ins Detail gegangen werden und der genaue Inhalt eines Attributes
bestimmt werden. Ist ein solches \gls{Repositorium} definiert, greift der
Transformations-Algorithmus darauf zu und püft die neuen Daten nach den
Definitionen. Sollte bei der Prüfung ein Konflikt entstehen, so kann der
Prozess diesen eigenständig lösen und die neuen Daten an die Definition
anpassen. Bei schwerwiegenden Fehlern, welche nicht von dem Algorithmus
gelöst werden können, wurde von dem Projektteam ein Fehlerreport definiert.
Dabei werden die auf Fehler gelaufenen Daten nicht entfernt und ein
beliebiger Verteiler darüber informiert. Anschließend kann ein Administrator
den Fehler beseitigen oder die Daten manuell in das Datenhaltsungssystem
überführen.

Ist der Transformations-Teil bei einem Datensatz vollständig durchgelaufen,
beginnt der Lade-Teil. Dabei übergibt der Transformations-Algorithmus den
vollständig korrekten Datensatz an den Lade-Algorithmus. Neben dem eigentlichen
Datensatz übermittelt dieser auch die Zieldatenbank und Zieltabelle, sodass
der Lade-Algorithmus nun nur noch die Daten in das entsprechende Objekt
einfügen muss. Dabei wird speziell auf die Auslastung des Zielobjektes
geachtet. Ist die Tabelle zum Beispeil durch einen anderen Prozess belegt,
wird dies von dem Lade-Algorithmus erkannt und das Einfügen der Daten
hinten angestellt. Für diesen Teil besitzt die Postgres Datenbank bereits
vorhandene Frameworks, welche von dem Projektteam genutzt werden. Somit muss
das Management der Prozessabarbeitung innerhalb des Datenhaltsungssystem nicht
selber entwickelt, sondern nur eingebunden werden.
\nl%

\subsection{Repositorium}
\label{subsec:postgres_repositorium}
Der in Punkt~\ref{subsec:postgres_etl} angesprochene Metadatensatz der in
diesem Projekt für die Metric-Daten jedes virtuellen Servers genutzt wird,
wurde zusammen mit allen Projektmitgliedern definiert und abgesprochen.
Dabei wurde sich auf eine spezielle Abspeicherung der Daten geeinigt. Es
handelt sich um die Verwendung von einer einzelnen Datentabelle, welche
nur zwei Attribute besitzt. Bei beiden Attributen handelt es sich um ein
JSON Objekt. Ersteres beinhaltet Metadaten zu dem einzelnen Datensatz.
Das ist zum Beispiel eine generierte ID, um den Datensatz spezifisch
selektieren zu können, oder die Informationen seit wann der Datensatz
in der Datenhaltung existiert, beziehungsweise wann und von welchem System
angefasst wurde. Zweites JSON Objekt/Attribut beinhaltet den eigentlichen
Datensatz, also die werthaltigen Informationen, welche auch später analysiert
werden. Die Inhalte des Attributes werden klar in dem Metadatensatz definiert.
Der Vorteil dieser Datenspeicherung ist zum einen die Effizienz des
Speicherplatzes sowie Geschwindigkeit und zum anderen das Zusammenhalten der
zugehörigen Informationen. Desweiteren können in diesem Model sehr einfach neue
Informationen hinzugefügt werden, indem diese in dem Metadatensatz definiert
werden. Dies ist ein Vorteil der Skalierbarkeit und es müssen anders wie bei
Datenbanken in der dritten Normalform die voneinander abhängigen Daten nicht
in jeweils eine eigene Tabelle ausgelagert werden.

Da sich bei diesem Model die Geschwindigkeit der Datenabfragen mit der Menge
der Daten in einer Tabelle verlangsamt, wird die automatische Partitionierung
angewandt, welche im nachfolgendem Punkt~\ref{subsec:postgres_partition}
beschrieben wird.
\nl%

\subsection{Partitionierung}
\label{subsec:postgres_partition}
Bei der Partitionierung in dem Postgres-System muss zwischen zwei Arten von
Partitionierung entschieden werden. Ersteres ist die manuelle Partitionierung
durch den Administrator. Dabei werden über einen Index Datenbereiche definiert,
welche logisch voneinander getrennt werden. Mit nachfolgendem Code-Prototyp
kann solch eine Trennung im groben vom Administrator definiert werden.

Unter dem im Anhang~\ref{lst:postgressample} beigefügten Prototypen wird eine
Haupttabelle in zwei Kinder-Tabellen aufgeteilt. Diese Kinder-Tabellen halten
dabei bestimmte definierte Datenbereiche. In diesem Prototypen enthält die
Kinder-Tabelle \texttt{measurement\_xMINUS12Hour} nur Datensätze bei denen der
Zeitstempel der beinhalteten Daten zwischen dem aktuellen Zeitpunkt und dem
aktuellen Zeitpunkt minus 12 Stunden liegt. Die zweite Kinder-Tabelle
beinhaltet wiederrum nicht die Daten von der ersten Kinder-Tabelle, sondern nur
alle Daten welche einen Zeitstempel innerhalb eines Tages besitzen. Diese Art
von Partitionierung kann auf jedem beliebigen Attribut angewendet werden,
sodass zum Beispiel auch noch CPU-Daten von RAM-Daten getrennt werden. Der
Anwender oder Analyst der Haupttabelle merkt dabei nicht, dass die verwendete
Tabelle partitioniert ist. Für die vollständige Implementierung wurden von dem
Projektteam jedoch auch noch weitere Funktionen programmiert und implementiert.
Darunter ist zum Beispiel das Verhalten der Datenbank bei dem Einfügen von
Daten mit einem älteren Zeitstempel.

Bei der manuellen Partitionierung kann ein Administrator jedoch auch durch die
Stärke der Partitionierungs-Intervalle an Optimierung verlieren, oder sogar das
System verschlechtern. Hinzu kommt ein starker administrativer Aufwand umso
mehr Partitionen eingefügt werden. Dabei kann auch die Übersichtlichkeit des
Codes verloren gehen. Entsprechend bietet Postgres auch die Methode der
automatisierten Partitionierung an. Dabei werden die Partitionen von dem
Postgres System selbständig angepasst und die Datenspanne definiert. Dies
bringt den Vorteil, dass eine schnelle Geschwindigkeit der Datenabfrage zu
jedem Zeitpunkt gewährleistet ist. Der administrative Aufwand wird mit dieser
Methode jedoch nicht verringert. Die manuelle Partionierung hat einen
kontinuierlich gleich bleibenden administrativen Aufwand. Bei der
automatisierten Partionierung muss vorallem zu Begin mehr Zeit investiert
werden. Erst bei der optimalen Einstellung der Konfigurationsmöglichkeiten
verringert sich der administrative Aufwand deutlich.

Das Projektteam hat sich entschieden die automatische Partionierung zu wählen.
Dies gewährleistet nicht nur eine dauerhaft schnelle Datenhaltung, sondern ist
auch Kundenfreundlicher, da dieser nicht dauerhaft einen Administrator für
die Partionierung einstellen muss. Die genaue Implementierung der Methode
kann aus dem Postgres-Code im Projektanhang entnommen werden.
\nl%

\subsection{Datenschutz}
\label{subsec:postgres_datenschutz}
Für den Zugriff auf die in dem Datenhaltsungssystem vorhandenen Daten wurde von
dem Projektteam eine weitere Sicherheitsebene eingefügt. Diese definiert, dass
ein Systembenutzer, also ein Account, welcher Daten von dem System abfragt,
keinen direkten Zugriff auf Tabellen im Datenhaltsungssystem erhält.
Stattdessen wird für jeden Anwendungsfall, also jede Datenbankabfrage,
beziehungsweise analytische Fragestellung, eine eigene Postgres-Funktion
bereitgestellt. Der Systembenutzer kann dabei ausschließlich die für ihn
freigeschalteten Funktionen ausführen. Bei dem Systembenutzer für die API sind
dies zum Beispiel nur Funktionen zur Abfrage von definierten Daten.

Dieses Verfahren hat den großen Vorteil, dass wenn einer der Systembenutzer
entfremdet wird, dieser keinen Schaden an dem Datenhaltsungssystem ausüben
kann. Desweiteren kann der Systemadministrator genau kontrollieren, wer auf
welche Daten Zugriff haben soll. Ein weiterer Vorteil ist, dass durch die
vordefinierten analytischen Fragestellungen bei jedem Benutzer die gleichen
Daten ausgegeben werden, und somit eine falsch Aussage, aufgrund von
verschiedenen Implementierungsweisen, vermieden wird.
\nl%

\section{API-System}
\label{sec:api_system}
Bei der Entwicklung des API-Systems wurde sich zunächst auf die Implementierung
der Postgres und Grafana Anbindung konzentriert. Voraussetzung für die
eigentliche Entwicklung ist eine funktionierende und auch möglich dem
Entwickler vertraute Entwicklungsumgebung. Bei der Swagger API wurde dazu
die Umgebung mit dem Namen „Spring Tool Suite“ verwendet. Diese basiert auf
dem Entwicklerstudio „Eclipse“ und war dem Projektteam bereits bekannt. Die
Spring Tool Suite ist dabei für die Entwicklung einer Swagger API optimiert
und liefert alle benötigten Grundlagen mit. Für die Einbindung weiterer
Funktionen wurden die entsprechenden Maven-Abhängigkeiten eingefügt. Diese
ermöglichen die einfache Integration von benötigten Bibliotheken oder
Code-Algorithmen. Das Spring Tool Suite liefert ebenfalls eine eigene
Testumgebung mit, in welcher der Entwickler für das Projekt angepasste Tests
definieren und durchführen kann.

Nach der Einrichtung des Entwicklerstudios wurde zunächst das API Model
definiert. Es handelt sich dabei um die Datenstruktur, welche das API System
von einem anderem System zugeschickt bekommt, beziehungsweise zuschickt. Da
sich das Projekt auf die Zusammenarbeit mit dem Grafana System konzentriert,
wurde mit der Modellierung dieser Datenstrukturen begonnen. Dabei arbeitet
Grafana nur mit vier URL's (Endpunkte) der API\@. Die erste ist dafür zuständig
zu prüfen, ob das API-System überhaupt erreichbar, beziehungsweise vorhanden
ist. Die zweite wird genutzt, um an dem API System anzufragen, welche Daten es
besitzt. Die dritte führt die eigentlichen Datenanfragen, basierend auf der
zweiten URL, aus und liefert die zu anzeigenden Daten zurück. Die letztere
ermöglicht es, einzelne Informationen in einem Graphen hervor zu heben. Alle
vier URL's besitzen eine eigene Anfrage-Methode und Datenstruktur an der API
und erwarten auch unterschiedliche Antworten von der API in Struktur und
Inhalt.

Nachdem diese URL's und Modelstrukturen definiert wurden, schaute das
Projektteam auf die genaue Zusammenarbeit zwischen Grafana und der API\@. Dies
hatte den Vorteil, dass die darauf folgende Modellierung der Datenabfrage bei
dem Datenhaltsungssystem besser nachvollziehbar war. Entsprechend wurden die
im Punkt~\ref{subsec:postgres_datenschutz} definierten Postgres-Funktionen nun
auch in dem API-System integriert.
\nl%

\section{Realisierung Grafana}
\label{sec:realisierung_grafana}
Nachdem nun die
Evaluierung~\ref{subsec:durchfuehrung_evaluierung_datenvisualisierung}
abgeschlossen ist, konnte mit der Installation und Konfiguration von Grafana
begonnen werden. Damit die eigentliche Konfiguration durchgeführt werden
konnte, musste zuvor das Installationspaket von Grafana heruntergeladen und
installiert werden. Auf einem ArchLinux-Betriebssystem wird dieses Paket mit
folgendem Befehl installiert:

\begin{minted}{text}
sudo pacman -S grafana
\end{minted}

Pacman ist ein Paketinstaller mit dem weitere Software aus einem Repository
heruntergeladen und installiert werden kann. Er durchsucht hier die gesamten
vordefinierten Repository URLs nach Grafana und installiert hier die aktuellste
vorliegende Version auf dem Archlinux-Betriebssystem.

Nachdem Grafana erfolgreich heruntergladen und installiert worden ist, muss
dieses mit folgendem Befehl gestartet und falls gewünscht im Autostart von
Archlinux hinterlegt werden. Mit folgendem Befehl wird der Grafana Dienst
gestartet:
\begin{minted}{text}
systemctl start grafana.service
\end{minted}

Um Grafana im Autostart zu hinterlegen, wird folgender Befehl benötigt:
\begin{minted}{text}
systemctl enable grafana.service
\end{minted}

Nun ist Grafana gestartet und kann über die Weboberfläche mit Port 3000
bereits aufgerufen und konfiguriert werden. Bei der Installation von Grafana
wird bereits ein vordefinierter Administrator User mit Passwort festgelegt. Die
Zugangsdaten hierzu lauten:

\begin{outline}
  \1 Benutzername: admin
  \1 Kennwort: admin
\end{outline}

Diese Logindaten können nach dem erfolgreichen Login angepasst und verändert
werden. Anschließend sieht man hier bereits das Home Dashboard von wo aus die
einzelnen Einstellungen durchgeführt werden können.

Das Home Dashboard ist wie folgt aufgebaut: Links gibt es ein Grafana Symbol,
worüber alle Einstellungen, wie Plugin-Verwaltung, Datenquellen (Datasources)
oder Benutzereinstellungen durchgeführt werden können. Rechts neben dem Grafana
Symbol wird immer die aktuelle Ebene angezeigt auf der der Administrator sich
befindet. Wenn der Administrator sich hier auf dem Home Dashboard befindet, so
wird hier „Home“ angezeigt und die Übersicht der einzelnen Dashboards. Sollte
der Administrator in den Einstellungen unter Datasources sein, so wird hier
Data Sources eingeblendet. Dies soll dem Administrator zusätzlich helfen, eine
bessere Übersichtlichkeit zu erhalten. Auf der rechten Seite des Home
Dashboards hat der Administrator oder User die Möglichkeit, den
Aktualisierungstimer der anzuzeigenen Daten zu verändern. Standardmäßig steht
dieser auf sechs Stunden, kann aber beliebig angepasst
werden~\ref{figure:grafana_dashboard}.


Zu Beginn sollte hier zuerst die Datenquelle von der Grafana die ermittelten
Daten erhält, konfiguriert werden. Hierzu navigiert der Administrator auf das
Grafana Symbol und wählt hier anschließend „Data Sources“ aus. Nun kann hier
unter „+ Add data source“ eine neue Datenquelle definiert werden. Grafana
bietet bereits eine Vielzahl an Auswahlmöglichkeiten von Datenhaltungssystemen
an. Diese können nach Belieben durch Plugins erweitert werden. Hierauf wird im
weiteren Verlauf der Implementierung eingegangen. Nachdem hier alle Werte
ausgefüllt worden sind, kann die Datenquelle über die Schaltfläche „Add“
hinzugefügt werden. Grafana prüft bereits beim Speichern der Datenquelle, ob
diese erreichbar ist.

Sobald die Datenquelle hinterlegt wurde, muss zunächst ein neues Dashboard
angelegt werden, damit die Graphen definiert werden können. Das neue Dashboard
kann unter dem Home Dashboard angelegt werden. Anschließend erhält der
Administrator bereits eine Vielzahl von vordefinierten Auswahlmöglichkeiten von
Graphen, Tabellen oder auch \glslink{singlestat}{Singlestats}. Diese können je
nach Verwendungszweck hinzugefügt werden. Eine Beschränkung, wieviele Graphen
maximal auf einem Dashboard angelegt werden können, gibt es nicht.

Nachdem der Graph ausgewählt worden ist, kann dieser angepasst und bearbeitet
werden. Hierzu wird der zu editierende Graph ausgewählt und anschließend auf
„edit“ geklickt. Unterhalb des Graphen öffnet sich dann ein Menü, wo der
grundsätzliche Aufbau, die Metrik, die Axen und diverse andere Einstellungen
für diesen Graphen definiert werden können. Hier wird zunächst die Metrik des
Graphen angepasst, damit der Graph Werte visualisieren kann. Unter dem
Abschnitt~\ref{subsec:graphen-definieren} wird hier genauer auf die Erstellung
von Graphen eingegangen. Im nachfolgenden ist ein Beispiel für eine solche
Abfrage:

\begin{figure}[H]
  \centering
  \includegraphics[width=1.0\textwidth]{../figures/graph.png}
  \caption{Beispielabfrage für eine Graphenvisualisierung}
\label{figure:graph}
\end{figure}

Anschließend können noch diverse Feineinstellungen für den Graphen vorgenommen
werden. Dieses Vorgehen wird nun für jeden Graphen erstellt, dessen Werte
visualisiert werden sollen. Nachdem alle Graphen erstellt worden sind, muss das
Dashboard abgespeichert werden, damit Einstellungen nicht verloren gehen
können. Sollte das Speichern vom Administrator vergessen werden, so warnt
Grafana bei dem Wechsel eines Dashboards, ob die Einstellungen gespeichert
werden sollen.

Die Grundeinstellungen sind nun abgeschlossen und Grafana kann verwendet
werden. Um weiteren Usern Zugriff zu dem Dashboard zu gewähren, können unter
dem Grafana Symbol > Admin > Global Users weitere User angelegt werden und
Zugriff auf die Weboberfläche von Grafana erhalten. Insbesondere hier kann
beschränkt werden, welche User welche Einstellungen und Graphen sehen können.
Ebenso ist es möglich, weitere Administratoren zu erstellen. Dies kann sowohl
für lokale User als auch für LDAP-User vergeben werden.  Sollte ein Unternehmen
die LDAP-Anbindung für die Authentifizierung an Grafana wünschen, so muss hier
in dem Verzeichnis von der Grafana-Installation die Datei grafana.ini angepasst
werden. Alle Einstellungen, die hier vorgenommen werden, beziehen sich auf die
Servereinstellungen, welche vom Kunden individuell angepasst werden kann.
Nachdem die Änderungen hier abgeschlossen sind, muss der Grafana-Dienst neu
gestartet werden. Dies wird mit folgendem Befehl durchgefüht:

\begin{minted}{text}
systemctl restart grafana.service
\end{minted}

Diese Konfiguration beschreibt die manuelle Installation und Konfiguration von
Grafana. Um hier jedoch nicht bei jedem einzelnen Kunden diese Anpassung
vornehmen zu müssen, kann die Installation, wie auch die Konfiguration
automatisiert werden.

Die Automatisierung erfolgt mit einem Puppet Profil. Das Puppet Profil hat
verschiedene Vorteile. So ist dieses Betriebsystem unabhängig, Versionen von
Grafana können vorgegeben werden und Konfiguration werden direkt während der
Installation in die Software implementiert. Um das Puppet Profil aufrufen zu
können, wird folgender Befehl benötigt:

\begin{minted}{text}
sudo puppet apply grafana.pp
\end{minted}

Innerhalb des Profils wird eine „grafana.pp“ Datei erstellt und anschließend
mit dem Puppet Profil aufgerufen. In dieser ist definiert mit welcher Version
Grafana installiert, welche Datenquelle verwendet oder welches Dashboard
bereits vordefiniert übergeben werden soll. Hier können zusätzlich noch
verschiedene andere Parameter, wie LDAP-Anbindung übergeben werden. Dies
erleichtert dem Administrator die Konfiguration, da dieses Profil nur einmal
auf dem Betriebssystem ausgeführt werden muss, um Grafana zu installieren und
zu konfigurieren. Eine zusätzliche Konfiguration nach der Installation von
Grafana wird nur dann benötigt, wenn spezielle Anforderungen vom Kunden
gewünscht sind. Dashboards werden in der Konfigurationsdatei des Puppet Moduls
für Grafana als \gls{JSON} definiert und können im Nachhinein angepasst werden.

Im Anhang~\ref{lst:grafana} befindet sich ein Beispiel für eine solche
Konfigurationsdatei.

Während der ersten Tests für die Automatisierung der Installation stellte sich
heraus, dass es hier ein Problem gibt, die Konfiguration an Grafana zu
übergeben. Das Puppet Profil prüft hier den Login auf die Weboberfläche nicht,
da es davon ausgeht, dass die Weboberfläche nach dem Neustart von Grafana
unmittelbar danach erreichbar ist. Das Problem liegt hier nicht in der
grafana.pp, sondern in dem eigentlichen Puppet Modul, welches die
Ressourcentypen \texttt{grafana\_dashboard} sowie \texttt{grafana\_datasource}
bereitstellt. Hierzu wurde ein Bugreport beim Entwickler des Puppet Moduls
erstellt (vgl.~\cite{grafana-issue}).
\mr%

\subsection{Graphen definieren}
\label{subsec:graphen-definieren}
Soabld die Grundkonfiguration von Grafana wie in~\ref{sec:realisierung_grafana}
beschrieben durchgeführt worden ist, können als nächstes Graphen visualisiert
werden.
\mr%

\subsubsection{General}
Zu Beginn wird hier unter dem Punkt \texttt{General} ein
aussagekräftiger Titel, sowie Beschreibung des Graphen hinterlegt. In dem
Beispiel, welches im Anhang~\ref{figure:grafana_general} enthalten ist, wird
der Graph für die Prozessor Auslastung definiert. Unter dem Punkt
\texttt{General} können alle Einstellungen für das Aussehen des Graphen
hinterlegt und angepasst werden. Ebenfalls können Verlinkungen zu anderen
Dashboards oder direkten URL Adressen hinterlegt werden. Diese werden
anschließend, wie im Anhang~\ref{figure:grafana_description} dargestellt. Dies
dient zur besseren Übersichtlichkeit, wenn ein Administrator oder die
Führungsebene diese Graphen analysieren möchten. Im zweiten Schritt werden nun
die einzelnen Graphen definiert.
\mr%

\subsubsection{Metrics}
Hierzu muss unter \texttt{Metrics} zunächst ausgewählt werden, von welcher
Datenquelle die erhaltenen Daten stammen und anschließend über \texttt{Add
Query} hinzugefügt werden. Grafana unterstützt unter anderem viele verschiedene
Datenquellen, wie zum Beipsiel OpenTSDB (siehe
Abschnitt~\ref{subsubsec:opentsdb}) oder Elasticsearch (siehe
Abschnitt~\ref{subsubsec:elasticsearch}). In dem Projekt wird für die Daten
Erhaltung eine HTTPS-API~\gls{API} Schnittstelle verwendet, welche im Laufe des
Projektes entwickelt und entworfen worden ist. Die HTTPS-API liefert hier ein
\gls{JSON}-Objekt zurück. Damit Grafana dieses als Datenquelle verwenden kann,
muss hier zuvor noch ein Plugin installiert werden. Es stehen viele
verschiedene Plugins auf der Grafana Webseite zur Verfügung. Zu jedem Plugin
gibt es hier eine kurze Erläuterung, sowie eine Schritt für Schritt Anleitung
zur Installation dieses Plugins. Die Plugins werden hier über einen
Kommandozeilenbefehl installiert. Anschließend kann unter Datenquelle die
HTTPS-API angesprochen werden. Sie wird nicht mittels einer SQL-Abfrage
abgerufen, sondern über eine \gls{HTTPS} Schnittstelle. Innerhalb der API
werden bereits vordefinierte Werte zurückgegeben. Wenn hier als Datenquelle
eine Datenbank angesprochen werden soll, so kann hier zunächst eine Art
\texttt{SELECT} \gls{SQL}-Abfrage hinterlegt werden. Innerhalb der SQL-Abfrage
wird definiert, von wo die erhaltenen Daten stammen, sowie welche Werte
abgerufen werden sollen. Ebenfalls kann diese SQL Abfrage mit anderen SQL
Abfragen innerhalb einer Gruppe zusammen definiert werden. Ein Beispiel hierzu
ist dem Anhang~\ref{figure:grafana_metrics} beigefügt.

Nun sind bereits einzelne Werte im Graphen visualisiert. Nachdem nun
die Grundeinstellung für die Graphen konfiguriert wurde, können anschließend
weitere Feineinstellungen durchgeführt werden.
\mr%

\subsubsection{Axes}
Im nachfolgenden Punkt \texttt{Axes} kann definiert werden, ob eine
Beschriftung für die X- oder Y-Achse benötigt wird. Zur Verdeutlichung ist dem
Anhang~\ref{figure:grafana_axes} ein Bild hierzu beigefügt. Zusätzlich kann
hier auch die entsprechende Einheit für diese Werte definiert werden, denn die
erhaltenen Werte aus der Datenbank oder der API sind meist nur Ganz- oder
Kommazahlen Werte, welche keine Einheiten übergeben bekommen.
\mr%

\subsubsection{Legend}
Unter dem Menüpunkt \texttt{Legend} wird die Legende für diesen Graphen
definiert. Eingestellt werden kann hier welchen Minimalen, Maximalen, Aktuellen
und Durschnitts Wert innerhalb der Analyse ausgegeben worden ist und ob diese
Werte als Tabelle ausgegeben werden sollen. Diese Ausgabe erfolgt direkt
unmittelbar unterhalb des Graphen und soll für eine bessere Übersichtlichkeit
für den Administrator dienen, da dieser die Minimalen- und Maximalenwerte
welche innerhalb der Analyse ausgegeben worden sind sehen kann. Ein Beispiel
hierzu ist dem Anhang~\ref{figure:grafana_legend} beigefügt.
\mr%

\subsubsection{Display}
Nachdem nun die Axenbeschriftung und Legende angepasst worden sind, gibt es
unter dem Punkt \texttt{Display} drei Einstellungsmöglichkeiten, \texttt{Draw
options}, \texttt{Series overrides} und \texttt{Thresholds}. Der Aufbau hiervon
kann dem Anhang~\ref{figure:grafana_display} entnommen werden. Unter dem Punkt
\texttt{Draw options} kann das Aussehen des Graphen angepasst werden.  So kann
bei jedem Schnittpunkt oder Veränderung des Graphen ein Punkt hinterlegt
werden. Ebenso kann definiert werden, ob der Graph zur besseren
Übersichtlichkeit mit Farbe gefüllt werden soll oder die Breite der Linien des
Graphen. Hier gibt es ebenfalls noch den Punkt, den Graphen als Balkendiagramm
zu visualisieren. Diese Einstellungen werden für alle Graphen innerhalb der
Visualisierung durchgeführt. Um dies für jeden einzelnen Graphen anpassen zu
können, gibt es den Punkt \texttt{Series Overrides}. In diesem können die
Einstellungen, welche ebenfalls in \texttt{Draw options} definiert werden
können, für jeden einzelnen Graphen festgelegt werden. Abschließend gibt es zu
diesem Menüpunkt noch den Punkt \texttt{Thresholds}.Dies sind Markierungen,
innerhalb des Graphens, wenn ein kritischer Wert erreicht worden ist. Diese
dienen dazu auf einem Blick zu erkennen, ob es hier Probleme gibt, oder alles
in Ordnung ist. Im Anhang~\ref{figure:grafana_thresholds} ist ein Bild
beigefügt, welches dies einmal verdeutlicht.
\mr%

\subsubsection{Alert}
Im vorletzten Punkt \texttt{Alert} können bestimmte Graphen überwacht werden
und können bei einem kritischen Wert eine Benachrichtgung versenden. Damit
diese Funktion verwendet werden kann, muss zum einen dies unter dem Grafana
Symbol oben Links Alerting konfiguriert werden und zum anderen muss dies die
Datenquelle ebenfalls unterstützen. Die Benachrichtigung hierzu kann
anschließend beispielsweise über Email oder die Messenger Telegram oder Threema
erfolgen. Weitere Kommunikationswege sind hier ebenfalls möglich. Innerhalb der
\texttt{Alert} Konfiguration, kann für jeden Graphen individuell die Kriterien
für die Benachrichtigung definiert werden. Ein Beispiel hierzu ist dem
Anhang~\ref{figure:grafana_alert} beigefügt. Die Bedingung, die Konfiguriert
werden kann, muss entwender einen Min-, Max- oder Durschchnittswert
beeinhalten, ab wann diese Bedingung erfüllt ist. Sobald diese Bedingung
erfüllt ist, wird entsprechend eine Benachrichtigung versendet. Diese wird
entweder mit einem Standardtext versendet oder kann über den Punkt
\texttt{Notifications} individuell angepasst werden. Hier kann ebenfalls
hinterlegt werden, über welchen Weg die Benachrichtung erfolgen soll.
\mr%

\subsubsection{Time range}
Im letzten Punkt \texttt{Time range} kann der Intervall der Visualisierung für
den Graphen definiert werden. Auf dem Grafana Dashboard oben Rechts, kann die
allgemeine Zeit definiert werden, wie lange alle Graphen auf dem Dashboard
Werte abrufen können. Diese Einstellung kann individuell für jeden einzelnen
Graphen unter \texttt{Time range} überschrieben werden.
\mr%

\section{Logstash und collectd}
Die Implementierung beider Komponenten gestaltet sich als sehr einfach. Für
beide existiert ein fertiges Puppetmodul, außerdem sind die
Konfigurationsoptionen überschaubar. Für Logstash wurde ein eigenes Puppet
Profil geschrieben, dieses befindet sich unter~\ref{lst:logstash-profile}. Es
installiert Logstash und anschließend das JDBC Plugin für die Kommunikation mit
der Datenbank. Hierfür wird noch der Postgres Treiber benötigt, dieser wird von
der Postgres Webseite heruntergeladen. Zum Schluss wird die Logstash
Konfiguration aus~\ref{lst:logstash} erzeugt und Logstash gestartet. Hierbei
stellte sich heraus, dass das Puppet Modul für Logstash nicht zu 100\% auf
Archlinux funktioniert. Dies wurde vom Projektteam behoben. Der Fix wurde den
Entwicklern in einem Pull Request zugeschickt (vgl.~\cite{logstash-bug}).

Die Konfiguration von collectd läuft sehr ähnlich ab. Es wurde wieder ein
Puppet Profil erzeugt, dieses befindet sich unter~\ref{lst:collectd-profile}.
Dies installiert collectd, richtet dann das Netzwerk-Plugin ein, um Daten zu
Logstash schicken zu können. Im Anschluss werden die verschiedenen Plugins zum
Sammeln der Metriken aktiviert. Hierbei wurde festgestellt, dass das Puppet
Modul für collectd keine sinnvollen Standardwerte für das Konfigurieren des
collectd Repositorys besitzt. Dies wurde behoben und der Fix wieder den
Entwicklern zur Verfügung gestellt (vgl.~\cite{collectd-bug}).
\tm%

\chapter{Tests und Qualitätssicherung}
Der genutzte sowie der selbst entwickelte Code ist äußerst komplex. Es
muss von Anfang an sichergestellt werden, dass dieser Code nicht nur
funktioniert, sondern auch den gängigen Best Practices der jeweiligen
Programmiersprache entspricht. Dadurch wird der Code strukturierter und
ist für Außenstehende leichter zu verstehen. Im folgenden Kapitel werden die
einzelnen Testverfahren vorgestellt und begründet.
\tm%

\section{Peer Review}
Alles an entwickelter Software befindet sich in einem \gls{Git}
\gls{Repository}. Die Repositories sind unter anderem auf der Webseite
\url{https://github.com/} gespeichert. Im Branch \texttt{master} befindet sich
ausschließlich getesteter Code. Es wird nie direkt in diesem Branch gearbeitet.
Jedes Teammitglied erstellt sich einen \texttt{Feature Branch}. Dies ist ein
kurzlebiger Branch, welcher auf \texttt{master} aufbaut. Im \texttt{Feature
Branch} wird ein einzelnes Feature von einem einzelnen Teammitglied
implementiert. Im Anschluss wird ein Pull Request eröffnet. Dies ist eine
Möglichkeit von GitHub, einen Branch mit einem anderen zusammenzuführen. Über
die Webseite werden die Unterschiede der beiden Branches gezeigt. Ein weiteres
Teammitglied liest diese Änderungen (Review). Man kann nun Korrekturen
anfordern oder den Code mergen. Dann wird der \texttt{Feature Branch} mit
\texttt{master} zusammengeführt. Ein Teammitglied darf nie seinen eigenen Code
mergen.
\tm%

\section{Continuous Integration}
Continuous Integration beschreibt eine Vorgehensweise in der
Softwareentwicklung. Hierbei wird für jede Änderung eine zuvor definierte
Testmatrix durchlaufen, um sicherzustellen, dass die Software weiterhin intakt
ist. Dies setzt vorraus, dass Tests vor dem eigentlichen Code entwickelt werden
(Test Driven Development) oder parallel zur Entwicklung des Codes. Ersteres
stellt sicher, dass ausschlieslich das implementiert wird, das gefordert ist.
In den Tests wird definiert, wie eine Software zu funktionieren hat, danach
wird so lange am Code entwickelt bis die Tests erfolgreich sind. Wenn man die
Tests parallel zum Code entwickelt, ist es möglich, dass der Code mehr als die
geforderte Funktionalität aufweist.

Im Rahmen des Projekts werden Tests immer parallel zum Code entwickelt. Test
Driven Development erfordert ein enormes Fachwissen über die genutzten
Frameworks für Tests. Dieses lässt sich aufgrund der Komplexität nicht
während des Projektes aneignen.
\tm%

\subsection{Travis}
Travis ist eine Plattform für Continuous Integration. Sie ist kostenlos nutzbar
und stellt verschiedene virtuelle Maschinen und Container bereit, in denen
Tests ausgeführt werden können. GitHub kann bei neuen Pull Requests eine
Benachrichtigung an Travis schicken. Daraufhin startet Travis eine neue VM und
klont das entsprechende GitHub Repository und den \texttt{Feature Branch}.
Danach wird eine definierte Testmatrix ausgeführt. Travis schickt eine
Benachrichtigung an GitHub, nachdem die Tests beendet wurden. Somit ist in der
Weboberfläche ersichtlich, ob die Tests erfolgreich waren oder nicht.

Dieses Verfahren ermöglicht es, eine reproduzierbare Testumgebung zu nutzen.
Diese wird immer auotmatisch gestartet. Es ist nicht mehr notwendig, jeden Test
lokal auszuführen. Travis stellt VMs bereit, in denen nur das nötigste
installiert ist. Dadurch ist sichergestellt, dass keine \glslink{Side
Effect}{Side Effects} auftreten. Bei lokalen Tests tritt dies häufig auf. Der
Entwickler hat diverse Bibliotheken installiert, welche von der entwickelten
Software benötigt werden. Innerhalb der entwickelten Software wird eine
Abhängigkeit nicht definiert, es funktioniert aber trotzdem, da die Bibliothek
zufällig vorhanden ist. Innerhalb von Travis fällt dies aber auf. Außerdem ist
es jedem Teammitglied möglich, die Travis Logs von überall aus zu begutachten.
Alternativ müsste jeder Entwickler die Tests selbst lokal ausführen.
\tm%

\section{Linter und Syntax Validator}
Linter- und Syntax-Tests werden immer genau dann benötigt, wenn der Quellcode
immer komplexer wird. Ein Linter-Test prüft von einem Quellcode die
Formatierung und macht diesen anschließend lesbarer, damit auch Laien diesen
Quellcode gut verstehen können. Der Sinn hinter einem Linter-Test ist es, das
best Practice als auch die Darstellung auf dem Quellcode nach einer
hinterlegten Liste von Formatierungen zu erzwingen. Diese Liste an
Formatierungen beeinhaltet jede Programmiersprache und deren Formatierungen,
da jede Programmiersprache eigene Formatierungen besitzt. Ein Syntax-Test macht
eine statische Code-Analyse und prüft, ob der gesamte Code syntaktisch in
Ordnung ist. Dies bedeutet, dass zum Beispiel geprüft wird, ob eine Klammer
(öffnen oder schließen) vergessen wurde oder ein Simikolon an der richtigen
Stelle ist.
\mr%

\section{Unit-Tests}
\label{sec:unit_tests}
Für die Tests und Qualitätsicherung wurden unter anderem Unit-Tests verwendet.
Unit-Tests sind Software-Tests und werden immer dann verwendet, wenn jeweils
die Kernfunktion von einem Quellcode geprüft werden soll. Rspec ist das
Framework und stellt eine \gls{DSL} zur Verfügung, in der Testszenarien
definiert werden können. Die Puppet Module, die in diesem Projekt verwendet
werden, wie zum Beispiel Grafana, besitzen schon einige Tests, welche erweitert
worden sind. Man kann hier mit einem Test zum Beispiel prüfen, ob das Profil
auf unterschiedlichen Betriebssystemen ausgeführt werden könnte oder ob
bestimmte Ressourcen im kompilierten Katalog enthalten sind. Zu Beginn müssen
diese vorab mit rspec in dem Katalog kompiliert und definiert werden, damit
diese verwendet werden können. Mit folgendem Befehl können die Tests
durchgeführt werden:
\begin{minted}{text}
bundle exec rake spec
\end{minted}

Die Tests für das Grafana-Profil sehen wie folgt aus:

\begin{center}
  \inputminted{ruby}{../listings/rspec-grafana.txt}
  \captionof{listing}{Tests für das Grafana-Profil}
\end{center}

Zuerst werden verschiedene Betriebssysteme simuliert, um das Grafana-Profil auf
diesen zu testen. Jedes Betriebssystem besitzt betriebssystemspezifische
\glslink{Fact}{Facts}. Diese werden von rspec simuliert und mit Hilfe einer
Liste der unterstützten Betriebssysteme auf dem entsprechenend Profil
getestet. Diese Liste an Betriebssystemen wird aus der \texttt{metadata.json}
ausgelesen und hierfür immer ein Katalog erstellt. Dieser zuvor erstellte
Katalog wird anschließend geprüft. Unter anderem kann das rspec prüfen, ob
benötigte Ressourcen wie \texttt{grafana\_dashboard} oder
\texttt{grafana\_datasources} in dem Profil enthalten sind. Diese Tests können
für jedes Profil einzeln definiert werden, da jedes Profil unterschiedliche
Ressourcen benötigt. Im Anhang befindet sich ein Auszug aus einem bestandenen
Test. Es wurde bewusst darauf geachtet, dass jedes Modul nur seine eigenen
Ressourcen auf Fehler prüft. Zusätzlich wurde jede einzelne Software
von den Entwicklern auf Funktionalität und Sicherheit geprüft.

Die Vorteile von solchen Unit-Tests mit rspec sind Folgende:

\begin{outline}
  \1 Verschiedene Tests, wie Funktionalität des Profiles, können gleichzeitig
  geprüft werden.
  \1 Direkte Ausgabe, mit detaillierter Fehlerbeschreibung.
  \1 Tests können für alle Profile gleich definiert werden.
  \1 Die Tests können gleichzeitig für unterschiedliche Profile durchgeführt
  werden.
  \1 Das Framework erlaubt es auf einfache Art und Weise, viele verschiedene
  Betriebssysteme mit einem Test zu testen.
\end{outline}
\mr%

\section{Akzeptanz Test}
Ein Akzeptanz Test prüft den zuvor erstellten Code in einer eigens dafür
gestarteten virtuellen Maschine oder abgesichertem Container mit einem
Betriebssystem auf die Funktionalität der einzelnen Module. Hierzu wurden schon
bereits vorhandene Test Szenarien von \gls{testkitchen} und \gls{serverspec}
verwendet und diese auf die im Projekt verwendeten Module angepasst. Der Code
wird innerhalb dieser Tests ausgeführt und geprüft, ob die Grundfunktion, sowie
die Funktionalität der einzelnen Module und ob das Puppet Modul idempotent und
fehlerfrei läuft. Diese Szenarien prüfen beispielsweise, ob die Erhaltung der
Daten, sowie speichern dieser Daten in der Datenbank und Visualisierung der
Graphen funktioniert. Die Ausgabe erfolgt hier über einen booleschen Wert,
welcher entweder \texttt{true} oder \texttt{false} sein kann. Anschließend
kann der Entwickler diesen Code erneut anpassen und testen. Nachfolgend sind
zwei Tests für einen Akzeptanztest für das Modul collectd angehangen. Der
vollständige Test ist dem Anhang~\ref{lst:collectd-test}beigefügt.

\begin{listing}
  \inputminted{ruby}{../listings/acceptance-test.txt}
  \caption{Zwei Akzeptanztests für collectd}
\end{listing}

Wie auch auch bei Unit-Tests~\ref{sec:unit_tests} werden Akzeptanztests für
verschiedene Betriebsysteme programmiert. Anders als bei
Unit-Tests~\ref{sec:unit_tests} werden diese nicht in einer
\texttt{Metadata.json} hinterlegt, sondern jedes Betriebssystem erhält eine
eigenen Datei, in der definiert ist, mit welcher virtuellen Maschine oder
Container dieser Code ausgeführt werden soll. Im Anhang~\ref{lst:def-centos}
ist eine solche Datei beigefügt.
\mr%

\chapter{Ausblick}
\label{chapter:ausblick}
Im Projekt wurde eine solide und modulare Architektur erstellt. Sie erfüllt
alle Anforderungen, bietet aber auch noch diverse Möglichkeiten für
weitere Optimierungen oder Erweiterungen.
\tm%

\section{Alerting}
Es ist nicht Bestandteil des Projektes, bei bestimmten Schwellwerten
Benachrichtigungen zu verschicken. Das komplette Setup ist darauf ausgelegt,
reaktiv und nicht aktiv zu arbeiten. Grafana bietet die Möglichkeit des
Alertings. Dabei können Schwellwerte für bestimmte Metriken definiert werden.
Sobald diese erreicht werden, wird eine Benachrichtigung verschickt. Dies kann
über verschiedenste Kommunikationswege erfolgen. Denkbar ist es, Kunden zu
informieren, wenn Ihr virtueller Server am oberen Ressourcenlimit läuft. Es
könnte auch der Vertrieb informiert werden, welcher dann dem Kunden eine
größere Instanz verkaufen kann.
\tm%

\section{Quality of Service}
Logstash arbeitet im \texttt{Batch-Mode}. Dies bedeutet, das es mehrere Werte
pro Prozess cached und diese dann gebündelt in einem SQL-Statement zur
Datenbank schickt. Der Standardwert liegt bei 125 Werten und ist
konfigurierbar. Es ist in bestimmten Konstellationen möglich, das Logstash
mehrere Minuten lang Werte zwischenspeichert bevor sie zur Datenbank geschickt
werden. Dies passiert, wenn besonders viele Logstash Prozesse gestartet worden
sind und wenig Daten von den Hypervisors kommen (zum Beispiel weil gerade viel
VMs deaktiviert sind oder Wartungsarbeiten an Teilen der Infrastruktur
durchgeführt wird). Wenn ein Kunde live seine Auslastung visualisieren möchte,
dauert dies eventuell zu lange. Zur Optimierung gibt es zwei Möglichkeiten.
Zum einen kann der Zwischenspeicher verringert werden. Dies erhöht allerdings
die Last auf der Datenbank, da dies zu wesentlich mehr SQL-Statements führt.
Alternativ kann mit Logstash \glslink{Quality of Service}{QoS} gemacht werden.
Hierbei wird auf die eingehenden Daten ein \gls{Grok} Filter gelegt. Dieser
erlaubt es Daten zu filtern und in Kategorien einzuordnen. Für jede Kategorie
kann die Größe des Zwischenspeichers definiert werden. Diese Filter kann man
auch dynamisch aktivieren. Sobald ein Kunde eine Live-Visualisierung im Grafana
startet, kann ein Grok Filter aktiviert werden. Dieser filtert Daten dieses
Kunden heraus kategorisiert diese in eine QoS Queue mit einem Zwischenspeicher
von 0. Somit wird jeder Datensatz von diesem Kunden direkt an die Datenbank
weitergeleitet.
\tm%

\section{Cache Invalidation}
Neben dem eingesetzten \texttt{Batch-Mode} gibt es noch eine weitere
Arbeitsmethode in Logstash. Hierbei wird mit \texttt{Cache Invalidation}
gearbeitet. Es wird ein Schwellwert gesetzt, der das maximale Alter des Caches
beschreibt. Sobald die Daten den Schwellwert erreicht haben (zum Beispiel 5
sekunden), werden sie zur Datenbank geschickt, auch wenn das Limit des
Zwischenspeichers noch nicht erreicht ist.
\tm%

\chapter{Fazit}
In vielen Monaten Arbeit wurde ein neues Softwareprojekt geschaffen, welches
eine sehr gute Alternative zu Telemetry aus dem OpenStack Projekt darstellt.
Die erarbeitete Lösung erfüllt alle definierten Anforderungen aus
Abschnitt~\ref{section:anforderungen}. Nicht nur können alle Userstorys über
Grafana visualisiert werden, auch die API stellt einen passenden Placement
Algorithmus zur Verfügung und bietet somit mehr Features als Telemetry. Das
Softwareprojekt ist äußerst flexibel gehalten und kann in beliebige Umgebungen
mit Linux Hypervisors integiert werden. Aufgrund des modularen Konzepts und
den offenen Schnittstellen gibt es auch mehrere Möglichkeiten für zukünftige
Erweiterungen oder Spezialisierungen für bestimmte Umgebungen, wie im
Abschnitt~\ref{chapter:ausblick} dargestellt.

Das Projektteam hat nicht nur viel über Software Engineering, sondern auch
über agiles Projektmanagement und kollaboratives Arbeiten gelernt. Währen der
Entwicklung ergaben sich diverse Probleme. Durch das konsequente arbeiten mit
Prototypen konnten Probleme im ersten Architektur-Prototyp schnell erkannt
werden. Gemeinsam wurde stets eine Lösung gefunden. Diverse kleinere Bugs
in der benutzen Software wurden vom Team behoben und mit zusätzlichen Unit
Tests versehen. Alle Änderungen wurden an die jeweiligen Entwickler geschickt.
\all%

%\appendix

\printglossaries%

\printbibliography[heading=bibnumbered]

\chapter{Anhang}

\begin{figure}[tbph]
  \centering
  \includegraphics[width=1.0\textwidth]{../figures/atop_1.png}
  \caption{atop mit geringer Systemlast}
\label{figure:atop1}
\end{figure}

\begin{figure}[tbp]
  \centering
  \includegraphics[width=1.0\textwidth]{../figures/atop_2.png}
  \caption{atop mit hoher CPU/Netzwerk Last}
\label{figure:atop2}
\end{figure}

\begin{figure}[tbp]
  \centering
  \includegraphics[width=1.0\textwidth]{../figures/diamond.png}
  \caption{Offene PRs und Issues im Diamond Projekt am 22.01.2017}
\label{figure:diamond}
\end{figure}

\begin{figure}[tbp]
  \centering
%%  \includesvg[svgpath=../figures/,path=../figures/]{messagebusv1}
  \includegraphics[width=1.0\textwidth]{../figures/messagebusv1_2.pdf}
  \caption{Architekturdraft Version 1}
\label{figure:draft1}
\end{figure}
\begin{figure}[tbp]
  \centering
%%  \includesvg[svgpath=../figures/,path=../figures/]{messagebusv1}
  \includegraphics[width=1.0\textwidth]{../figures/messagebusv2_3.pdf}
  \caption{Architekturdraft Version 2}
\label{figure:draft2}
\end{figure}
\FloatBarrier{}

\begin{figure}[tbp]
  \centering
  \includegraphics[]{../figures/grafana_dashboard_l.png}
  \caption{Grafana Dashboard Linke Seite}
\label{figure:grafana_dashboard_l}
\end{figure}

\begin{figure}[tbp]
  \centering
  \includegraphics[]{../figures/grafana_dashboard_r.png}
  \caption{Grafana Dashboard Rechte Seite}
\label{figure:grafana_dashboard_r}
\end{figure}

\begin{figure}[tbp]
  \centering
  \includegraphics[]{../figures/grafana_general.png}
  \caption{Grafana Graph Menüpunkt General}
\label{figure:grafana_general}
\end{figure}

\begin{figure}[tbp]
  \centering
  \includegraphics[]{../figures/grafana_description.png}
  \caption{Grafana Graph Beschreibung}
\label{figure:grafana_description}
\end{figure}

\begin{figure}[tbp]
  \centering
  \includegraphics[]{../figures/grafana_metrics.png}
  \caption{Grafana Graph Metriks}
\label{figure:grafana_metrics}
\end{figure}

\begin{figure}[tbp]
  \centering
  \includegraphics[]{../figures/grafana_axes.png}
  \caption{Grafana Graph Axen}
\label{figure:grafana_axes}
\end{figure}

\begin{figure}[tbp]
  \centering
  \includegraphics[]{../figures/grafana_legend.png}
  \caption{Grafana Graph Legende}
\label{figure:grafana_legend}
\end{figure}

\begin{figure}[tbp]
  \centering
  \includegraphics[]{../figures/grafana_display.png}
  \caption{Grafana Graph Display}
\label{figure:grafana_display}
\end{figure}

\begin{figure}[tbp]
  \centering
  \includegraphics[]{../figures/grafana_thresholds.png}
  \caption{Grafana Graph Schwellenwert Beispiel}
\label{figure:grafana_thresholds}
\end{figure}

\begin{figure}[tbp]
  \centering
  \includegraphics[]{../figures/grafana_alert.png}
  \caption{Grafana Graph Benachrichtigung}
\label{figure:grafana_alert}
\end{figure}

\begin{figure}[tbp]
  \centering
  \includegraphics[width=1.0\textwidth]{../figures/ssduserstory-crop.pdf}
  \caption{Wireframe für SSD Userstory}
\label{figure:ssduserstory}
\end{figure}

\begin{figure}[tbp]
  \centering
  \includegraphics[width=1.0\textwidth]{../figures/cpuuserstory-crop.pdf}
  \caption{Wireframe für CPU Userstory}
\label{figure:cpuuserstory}
\end{figure}

\begin{figure}[tbp]
  \centering
  \includegraphics[width=1.0\textwidth]{../figures/memoryuserstory-crop.pdf}
  \caption{Wireframe für Memory Userstory}
\label{figure:memoryuserstory}
\end{figure}

\begin{figure}[tbp]
  \centering
  \includegraphics[width=1.0\textwidth]{../figures/timeframeuserstory-crop.pdf}
  \caption{Wireframe für Zeitdefinierte Analyse Userstory}
\label{figure:timeframeuserstory}
\end{figure}

\begin{figure}[tbp]
  \centering
  \includegraphics[width=1.0\textwidth]{../figures/uiuserstory-crop.pdf}
  \caption{Wireframe für Weboberfläche Userstory}
\label{figure:uiuserstory}
\end{figure}

\FloatBarrier
\lstinputlisting[caption={atop ASCII Logausgabe},label=lst:atop]{../listings/atop.txt}
\lstinputlisting[caption={Logstash Konfigurationsdatei},label=lst:logstash]{../listings/logstash-conf.txt}

\FloatBarrier
\begin{center}
  \begin{tabular}{ll}
  \toprule
  Feldname    & Beschreibung                                                 \\
  \midrule
  host        & Name des Servers auf dem das Event entstand                  \\
  service     & Name des Dienstes der das Event ausgelöst hat                \\
  state       & Beliebiger Text unter 255Bytes. Zum Beispiel „Ok“, „Warning“ \\
  time        & Uhrzeit an dem das Event erstellt wurde                      \\
  description & Beliebiger Text                                              \\
  tags        & Array mit Strings anhand dessen gefiltert werden kann        \\
  metric      & Die eigentliche Information, zum Beispiel die CPU Temperatur \\
  ttl         & Anzahl an Sekunden die ein Event nach Erstellung gültig ist  \\
  \bottomrule
\end{tabular}
\captionof{table}{Definition einer Event-Struktur in Riemann}
\label{tbl:riemann}
\end{center}

\begin{center}
  \begin{tabular}{lll}
  \toprule
    Projekt  & URL                                                 \\
  \midrule
    Logstash & https://github.com/elastic/puppet-logstash/pull/334 \\
    Grafana  & https://github.com/voxpupuli/puppet-grafana/pull/32 \\
  \bottomrule
\end{tabular}
\captionof{table}{Beiträge zu Open Source Projekten}
\label{tbl:fossprs}
\end{center}

\begin{center}
  \begin{tabular}{ll}
  \toprule
    Projekt     & URL                                                   \\
  \midrule
    Puppet      & https://tickets.puppetlabs.com/browse/PA-668          \\
    Puppet      & https://tickets.puppetlabs.com/browse/PUP-7383        \\
    Mcollective & https://tickets.puppetlabs.com/browse/MCO-804         \\
    Grafana     & https://github.com/voxpupuli/puppet-grafana/issues/35 \\
  \bottomrule
\end{tabular}
\captionof{table}{Gemeldete Bugs in Open Source Projekten}
\label{tbl:fossissues}
\end{center}


\chapter{Erklärung}
Hiermit erklären wir, dass wir die Arbeit selbstständig verfasst und keine
anderen als die angegebenen Quellen und Hilfsmittel benutzt haben. Diese Arbeit
wurde keinem anderen Prüfungsausschuss in gleicher oder vergleichbarer Form
vorgelegt.

\vspace{10ex}
{\centering
\renewcommand{\arraystretch}{0.9}
\begin{tabular}{p{0.25\textwidth}p{0.05\textwidth}p{0.25\textwidth}p{0.05\textwidth}p{0.25\textwidth}}
  \dotfill                    & & \dotfill                      & & \dotfill \\
  \centering\footnotesize{Tim Meusel}& & \centering\footnotesize{Marcel Reuter}& & \centering\footnotesize{Nikolai Luis}%
\end{tabular}
}

%%% Local Variables:
%%% mode: latex
%%% TeX-master: "thesis-de"
%%% End:
