\section{Erklärung}
Hiermit erklären wir, dass wir die Arbeit selbstständig verfasst und und keine
anderen als die angegebenen Quellen und Hilfsmittel benutzt haben. Diese Arbeit
wurde keinem anderen Prüfungsausschuss in gleicher oder vergleichbarer Form
vorgelegt.

\vfill
{\centering
\renewcommand{\arraystretch}{0.9}
\begin{tabular}{p{0.25\textwidth}p{0.05\textwidth}p{0.25\textwidth}p{0.05\textwidth}p{0.25\textwidth}}
  \dotfill                    & & \dotfill                      & & \dotfill \\
  \centering\footnotesize{Tim Meusel}& & \centering\footnotesize{Marcel Reuter}& & \centering\footnotesize{Nikolai Luis}%
\end{tabular}
}

\newpage

\section{Einführung}
Das Heinrich-Hertz-Europakolleg Bonn verlang im fünften und sechsten Semester
der Weiterbildung zum staatlich geprüften Informatiker eine fachbezogene
Projektarbeit. Dieses Projekt wird in Gruppen von zwei bis vier Personen
durchgeführt und soll fachliche Inhalte sowie Inhalte aus dem Projektmanagement
kombinieren. Es handelt sich um eine praktische Arbeit. Jeder Abschnitt enthält
ein Kürzel des Authors, hierbei bedeutet:
\begin{outline}
  \1 \verb+[+NL\verb+]+ erstellt von Nikolai Luis
  \1 \verb+[+MR\verb+]+ erstellt von Marcel Reuter
  \1 \verb+[+TM\verb+]+ erstellt von Tim Meusel
\end{outline}

\section{Projektvorstellung}
\subsection{Projektteam}
\subsection{Auftraggeber}
\subsection{Aktuelle Situation}
\subsection{Anforderungen}

\cite{lee2013introduction}
way more blabla

\section{Projektmanagement}
agil, sprints

\printbibliography
%%% Local Variables:
%%% mode: latex
%%% TeX-master: "thesis-de"
%%% End:
