\section{Introduction}

Partner for this project is the company Puppet Inc. (hereafter Puppet), who
acts as the contracting authority. The company is represented by Steve Quin and
David Schmitt. Contractors are Tim Meusel, Nikolai Luis and Marcel Reuter. Tim
Meusel will also act as a team leader. The project is going to be realized as a
part of the final project for a study program at the
Heinrich-Hertz-Europakolleg. The lecturer Dirk Stegemann is going to supervise
the project as a representative of the school, he is also responsible for the
final evaluation. The course of studies is `Staatlich geprüfter Informatiker'
undertaken between 2014 and 2018.

Virtualization technologies are used at cloud providers to partition the
resources of a physical system into multiple pieces. Physical resources are
shared between all virtual instances on a host. The goal of the different
providers is it to host as many virtual machines as possible on each physical
server. Due to overcommit, from time to time a single virtual instance can
consumes too many resources, interfering with the intended functionality and
performance of other virtual machines on the same host. Therefore, the usage
patterns and resource requirements of each machine has to be cleanly documented
so it can be analysed. The gathered information will then be used for
optimisation and tailoring of the platform as a whole.

\section{Task}

The goal of this project is to create a working open source solution. This is
divided into three major subtasks:
\begin{enumerate}
    \item Collect all required information from each virtual instances on the
          host system, spool them and then periodically send them to the database system.
    \item Set up a suitable database software which is scalable and provides a
          certain level of redundancy. The goal is to save the information of at least
          10.000 virtual instances.
    \item Provide a web interface where customers can see visualizations of
          their utilized resources. Furthermore an API is required to expose the data to
          different kinds of users. A user story survey for the three types of users
          (customer, administrator, manager) will be conducted with different partners.
          This will help the project team figure out what kind of, possibly preprocessed
          and aggregated data needs to be available via the API.
\end{enumerate}

The data has to be stored in one or multiple databases forming a data store.
Complex queries will be executed on the datastore, therefore a data warehouse
approach should be considered and analysed. The data from the virtual instances
will be collected in a regular time frame from 1 second to 5 minutes. The
optimal timeframe has to be figured out during the project.

There are already finished open source projects providing the listed, isolated,
specific functionality. This applicability of those solutions to the given task
have to be evaluated.

\section{Differentiation of the System}

There are already several open source software projects that claim to provide a
solution for one of the subtasks outlined above. The fitness of the following
software products for the given tasks will be evaluated:

\subsection{Data Acquisition Systems}

\begin{outline}
    \1 collectd with virt plugin
    \1 coreutils
    \1 zabbix-agent
    \1 python-diamond
    \1 sysstat
    \1 atop
    \1 logtash
    \1 riemann
\end{outline}

\subsection{Database Systems}

\begin{outline}
    \1 elasticsearch
    \1 cassandra
    \1 postgres
    \1 OpenTSDB
    \1 KNIME
    \1 impala
    \1 hadoop
    \1 hive
\end{outline}

\subsection{Frontends}

\begin{outline}
    \1 grafana
    \1 graphite
    \1 zabbix-frontend
\end{outline}

For both the exploration and the production phase, composition of the generic
components listed above is preferred over a domain- and application specific
reimplementation with a possibly limited list of features and test coverage. The
two companies Host Europe GmbH and Travis CI GmbH provide the team with free
testing- and development- environments.

\section{Description of Interfaces}

There are several software interfaces in this project. The first interface
collects and exports information over the virtual machine on the physical hosts.
The host provides detailed usage statistics for each virtual instance. The
network interface of the host will be used to send the gathered data to the
database via TCP or UDP. In case of performance problems, an optional queuing or
caching service might be required in front of the database. In this case the
host will communicate with this service instead of the database. There are
several possible solutions for the communication channel between the physical
server and the data storage systems:

\begin{outline}
    \1 The host directly communicates with a single database system
    \1 The host sends data to a loadbalancer which is placed in front of one to multiple database servers
    \1 The host talks to a system that runs an ETL process. This process will import the data into a data storage
    \1 The host talks to a messagebus which will save the data in the data storage
\end{outline}
As the process is intentionally kept open and explorative in nature, it is
possible that during the project, based on the validation and results available
from testing with prototypes based on different data storage solutions, new
approaches and storage-interfaces will emerge. Technicians like data analysts or
persons from the C-level need to be able to request the data in exchangeable
format used by common analysis and data processing tools. These formats will be
defined in the user story survey.

Every interface takes over the role of a sender and/or receiver, as
documented in the list below:
\begin{outline}
    \1 The virtual instance on a physical node
        \2 Sender
    \1 The physical node
        \2 Sender and receiver
    \1 The data storage
        \2 Receiver
    \1 The API
        \2 Receiver of a datarequest, sender for data
    \1 The webinterface
        \2 Sends data requests to the data storage,  along with a generated view to the terminal device
    \1 The terminal device
        \2 Receiver
\end{outline}

\section{Problem Analysis}

There are two major problems that the project team needs to take care of. First
of all, it needs to cope with the amount of data produced by every single node.
The data will be transferred via a network that is shared with other services.
The network must not be overallocated and the impact on the other services on
the network must be minimized. Furthermore the resource footprint of the data
acquisition software on the nodes must be minimal in order to not harm the
performance of the running virtual machines.

\section{External Influences}

The software needs to be build in a way that supports long continuous uptime
periods. It must be able to run unsupervised without regular intervention from
system administrators.

Expected behaviour and behaviour during a disturbance The datacenter provider
has to deal with power outages. A disruption on the central services like data
storage should be absorbed by caching or standby nodes. An outage on a single
physical host will kill all virtual machines on it, there is no way to
workaround this. The operator of the host has to deal with this and not the
project team.

\section{Feasibility of the Project}

\begin{enumerate}
    \item Organisational and Economically View
          The required amount of time will be estimated with the help of the
          ticket system Jira and agile development methods. The milestones will be planned
          together with Puppet. Apart from the students involved and listed in the
          introduction, there are no manpower costs in this project.

    \item Technical View
          The required hardware to power this project including all development
          \& testing tasks will be provided Puppet or other partners. The final
          approval at the end of the project can be made on the provided platforms.
\end{enumerate}


\section{Scope of the Functional Specification Document}

This document is to be considered effective and valid after it received joint
approval from both Puppet and the project group. Further changes need to be
accepted by the project team, Puppet and the lecturer from the
Heinrich-Hertz-Europakolleg. Tim Meusel is the project groups representative for
any such project related questions that may arise.

\section{Project Organisation}

Besides the roles of the contracting authority (Puppet) and the supervising
lecturer, a distribution of roles and responsibilities among the project team
members in required. Even though all team members will naturally be familiar
with each of the three major components of the project on a basic level, each
part will receive an module-officer who is ultimately responsible for the
specific component:
\begin{outline}
    \1 Team leader, Tim Meusel, responsible for:
        \2 Development of the data gathering process
        \2 Transport of the data between the place of gathering and the data storage
    \1 Project member, Nikolai Luis, responsible for:
        \2 Development of the data storage part and API
    \1 Project member, Marcel Reuter, responsible for:
        \2 Development and automation of the web frontend
\end{outline}

\subsection{Phase Planning}

There won’t be detailed schedule for the planned stages at the beginning of the
project. The project will be managed with agile methods. This allows the project
team to work with short sprints that don’t need detailed planned phases from the
beginning until the end. Every sprint will be planned and managed by the team
leader. While agile, the project however still has the following stages of
conventional software-engineering models:
\begin{outline}
    \1 Analysis
    \1 Design/Prototyping
    \1 Implementation
    \1 Testing/Bugfixing
    \1 Documentation
    \1 Acceptance stage
\end{outline}

Milestones will be defined by the project team in agreement with Puppet or
with the lecturer.

\section{Procedure Control}

Several different software solutions will be used to track the progress of the
project. They will allow every involved person to gather information about the
current status. ``Concluence'' will be used for documentation, ``Jira'' for tracking
issues and ``git'' to manage sourcecode and related artifacts such as build
scripts, etc.. The project team will create weekly reports on the progress
achieved in the past weeks.

\section{Risk Register}

The project team will work together with partner companies. Among other things they will
provide the team with real world data from their systems that can be used in tests.
It is possible that the partners will be unable to provide the requested data in the
desired timeframe or a lack of interest in the project and changed priorities.
In this case the project team has to generate and mock data by themself. Mocked data is never
as good as real data, it is possible that this distorts the tests and leads to
wrong conclusions. A second risk is the human factor of the project, e.g.
illness and the subsequent mid to long term drop out of one of the project
members.

%%% Local Variables:
%%% mode: latex
%%% TeX-master: "fsd-en"
%%% End:
