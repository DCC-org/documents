\begin{center}
\begin{tabularx}{\textwidth}{l lX}
  \toprule
  Feldname    & Beschreibung                                                 \\
  \midrule
  host        & Name des Servers auf dem das Event entstand                  \\
  service     & Name des Dienstes der das Event ausgelöst hat                \\
  state       & Beliebiger Text unter 255Bytes. Zum Beispiel „Ok“, „Warning“ \\
  time        & Uhrzeit an dem das Event erstellt wurde                      \\
  description & Beliebiger Text                                              \\
  tags        & Array mit Strings anhand dessen gefiltert werden kann        \\
  metric      & Die eigentliche Information, zum Beispiel die CPU Temperatur \\
  ttl         & Anzahl an Sekunden die ein Event nach Erstellung gültig ist  \\
  \bottomrule
\end{tabularx}
\captionof{table}{Definition einer Event-Struktur in Riemann}
\label{tbl:riemann}
\end{center}

\begin{center}
  \begin{tabularx}{\textwidth}{l lX}
  \toprule
    Projekt  & URL                                                 \\
  \midrule
    Logstash & https://github.com/elastic/puppet-logstash/pull/334 \\
    Grafana  & https://github.com/voxpupuli/puppet-grafana/pull/32 \\
  \bottomrule
\end{tabularx}
\captionof{table}{Beiträge zu Open Source Projekten}
\label{tbl:fossprs}
\end{center}

\begin{center}
  \begin{tabularx}{\textwidth}{l lX}
  \toprule
    Projekt     & URL                                                   \\
  \midrule
    Puppet      & https://tickets.puppetlabs.com/browse/PA-668          \\
    Puppet      & https://tickets.puppetlabs.com/browse/PUP-7383        \\
    Mcollective & https://tickets.puppetlabs.com/browse/MCO-804         \\
    Grafana     & https://github.com/voxpupuli/puppet-grafana/issues/35 \\
  \bottomrule
\end{tabularx}
\captionof{table}{Gemeldete Bugs in Open Source Projekten}
\label{tbl:fossissues}
\end{center}
