\newglossaryentry{API}
{
  name=API,
  description={Application Programming Interface, ermöglicht eine
               standardisierte Anbindung von externen Softwaresystemen über
               öffentliche Methoden einer Bibliothek~\cite{HowDoAPIsEvolve}}
}
\newglossaryentry{Cloud}
{
  name=Cloud,
  description={Eine Plattform, die nach dem Client-Server-Modell arbeitet. Eine
               oder mehrere bestimmte Dienstleistungen werden hier zur
               Verfügung gestellt (z.b. Mailhosting). Die Verteilung von
               Ressourcen erfolgt dynamisch}
}
\newglossaryentry{Public Cloud}
{
  name=Public Cloud,
  description={Siehe auch \gls{Cloud}, auf dieser Plattform teilen sich mehrere
               Benutzer die gleiche physische Infrastruktur}
}
\newglossaryentry{Private Cloud}
{
  name=Private Cloud,
  description={Siehe auch \gls{Cloud}, Hier wird jedem Nutzer eine dedizierte
               Infrastruktur bereitgestellt}
}
\newglossaryentry{JSON}
{
  name=JSON,
  description={JavaScript Object Notation, ein Datenformat mit dem Ziel, von
               Menschen und Maschinen verarbeitet werden zu können.
               Standardisiert in~\cite{RFC7159}}
}
\newglossaryentry{Jepsen}
{
  name=Jepsen,
  description={Jepsen ist eine in Clojure geschriebene Bibliothek mit der
               verteilte Datenbanken auf Persistenz und Zuverlässigkeit
               getestet werden können~\cite{Jepsen_Introduction}}
}
\newglossaryentry{Gnocchi}
{
  name=Gnocchi,
  description={Datenbank aus dem OpenStack-Projekt zum Speichern von Metriken}
}
\newglossaryentry{Unidirektional}
{
  name=Unidirektional,
  description={Eine Punkt-zu-Punkt-Verbindung, in der Daten nur in eine
               Richtung fließen}
}
\newglossaryentry{Bidirektional}
{
  name=Bidirektional,
  description={Eine Punkt-zu-Punkt-Verbindung, in der Daten in beide Richtungen
               fließen}
}
\newglossaryentry{DBMS}
{
  name=DBMS,
  description={Datenbankmanagementsystem, dieses System ist die Schnittstelle
               zwischen gespeicherten Daten und den Anwendern. Es stellt
               Operationen wie Datenzugriff und Datenmanipulation bereit.
               Außerdem verwaltet es das Datenbankschema. Das DBMS ermöglicht
               parallele Zugriffe, bietet Datenintegrität sowie administrativen
               Zugriff}
}
\newglossaryentry{SQL}
{
  name=SQL,
  description={Structured Query Language, SQL ist ein Medium, um mit einem
               \gls{DBMS} zu kommunizieren~\cite{sumathi2007fundamentals}}
}
\newglossaryentry{Streaming Replication}
{
  name=Streaming Replication,
  description={Hierbei schreibt ein Server seine Änderungen an einer Datenbank
               in eine Logdatei. Die Änderungen an der Datei werden
               kontinuierlich an alle passiven Server übertragen, diese wenden
               die Änderungen ebenfalls an}
}
\newglossaryentry{Active-Passive-Prinzip}
{
  name=Active-Passive Prinip,
  description={Eine Möglichkeit des Serverbetriebs. Ein aktiver Server erhält
               alle Anfragen von Clients, ein oder mehrere passive Server sind
               verfügbar. Der aktive Server kann deaktiviert werden und ein
               passiver wird dafür aktiv geschaltet}
}
\newglossaryentry{Pgpool-II}
{
  name=Pgpool-II,
  description={Eine Erweiterung für Postgres. Sie ermöglicht das
               Wiederverwenden von Datenbankverbindungen, das hochverfügbare
               Replizieren der Daten sowie das Verteilen von
               Leseanfragen~\cite{pgpool}}
}
\newglossaryentry{Postgres-XC}
{
  name=Postgres-XC,
  description={Eine Erweiterung für Postgres. Sie ermöglicht den vertikal
               skalierenden Cluster. Hierbei skaliert die Leistung sowohl beim
               Lese- als auch beim Schreibzugriff. Je mehr Nodes man in den
               Cluster stellt, desto höher wird die
               Schreibleistung~\cite{postgres-xc}}
}
\newglossaryentry{Skalierung}
{
  name=Skalierung,
  description={Beschreibt das Erhöhen der Ressourcen in einem komplexen
               Softwaresetup. Bei horizontaler Skalierung wird die Leistung
               eines einzelnen Servers durch ein Hardwareupgrade erhöht,
               bei vertikaler Skalierung wird das Softwaresetup auf mehrere
               Server verteilt. In großen Setups wird die vertikale
               Vorgehensweise bevorzugt, da sie mehr Leistung bieten kann}
}
\newglossaryentry{Hochverfügbarkeit}
{
  name=Hochverfügbarkeit,
  description={Ein IT-System gilt als verfügbar, wenn es seine definierte
               Aufgabe abarbeiten kann. Als hochverfügbar gilt es, wenn es
               diese Aufgabe trotz Ausfall einer oder mehrerer Komponenten noch
               erfüllen kann}
}
\newglossaryentry{Git}
{
  name=Git,
  description={Freie Software zur Versionsverwaltung von Dateien}
}
\newglossaryentry{Repository}
{
  name=Repository,
  description={Verzeichnis zum Speichern von Objekten. Im Git Kontext: Alle
               logisch zusammengehörenden Dateien eines Projekts}
}
\newglossaryentry{RESTful}
{
  name=RESTful,
  description={REST steht für Representational State Transfer. Es ist ein
               Architekturprinzip für Schnittstellen auf Basis von HTTP.
               RESTful bedeutet, dass eine Applikation das REST-Prinzip
               implementiert~\cite{fielding2000architectural}}
}
\newglossaryentry{Lucene}
{
  name=Lucene,
  description={Lucene ist eine Bibliothek der Apache Software Foundation zur
               Volltextsuche. Sie ist in Java geschrieben und erzeugt Indizes,
               welche im Anschluss durchsucht werden können. Lucene
               implementiert mehrere Suchalgorithmen. Die Indexgröße beträgt
               20-30\% der Größe der Nutzdaten~\cite{lucene_features}}
}
\newglossaryentry{Garbage Collector}
{
  name=Garbage Collector,
  description={Oft als GC abgekürzt, übersetzt: Müllabfuhr/Müllman. Es
               beschreibt die automatische Speicherwaltung eines Programms.
               Ein Programm allokiert Speicher dynamisch. Der GC prüft
               periodisch den belegten Speicher und gibt nicht mehr benötigte
               Teile wieder frei}
}
\newglossaryentry{HTTP}
{
  name=HTTP,
  description={HyperText Transfer Protocol ist ein
              Protokoll, welches für die Übertragung von Inhalten wie z.B.
              Webseiteninhalte verwendet wird}
}
\newglossaryentry{HTTPS}
{
  name=HTTPS,
  description={Hypertext Transfer Protocol Secure. Basiert auf
              \gls{HTTP} mit zusätzlicher Transportverschlüsselung}
}
\newglossaryentry{LDAP}
{
  name=LDAP,
  description={Lightweight Directory Access Protocol, ist
            ein Netzwerkprotokoll, welches zur Abfrage und Änderung von
            Informationen aus einem Unternehmensverzeichnis verwendet wird.
            LDAP besitzt keine Transportverschlüsselung, diese wurde bei LDAPS
            nachgerüstet}
}
\newglossaryentry{Active Directory}
{
  name=Active Directory,
  description={Ist ein von Microsoft entwickelter Server-Dienst, welcher auf
              Windows Server Maschinen installiert werden kann, um in einem
              Unternehmensverzeichnis über das Protokoll \gls{LDAP} eine
              Abfrage oder Änderung durchführen zu können}
}
\newglossaryentry{Treiber}
{
  name=Treiber,
  description={Ein Treiber definiert die Schnittstelle zwischen einer
              Software und einer Hardware. Damit eine gewisse Software auf
              eine Hardware zugreifen kann, wird ein Treiber verwendet,
              damit hier die volle Unterstützung zwischen Hard- und Software
              gegeben ist}
}
\newglossaryentry{Datenbanktreiber}
{
  name=Datenbanktreiber,
  description={Wie auch bei \glslink{Treiber}{Treibern}, wird ein
              Datenbanktreiber benötigt, um eine Verbindung mit einer
              Datenbank aufbauen zu können. Zusätzlich wird hier
              das Schema der Inhalte dieser Datenbank überprüft und
              für die Verwendung aufbereitet}
}
\newglossaryentry{Statement}
{
  name=Statement,
  description={Ein Statement ist eine Abfolge von Befehlen für eine \gls{SQL}
              Abfrage der Datenbank. Mit dieser können Werte abgerufen,
              verändert oder gelöscht werden}
}
\newglossaryentry{Dashboard}
{
  name=Dashboard,
  description={Ein Dashboard ist eine Oberfläche auf einer Webseite, um Graphen
              oder Metriken einem User zur Verfügung stellen zu können. Dem
              Anwender werden diese Graphen oder Metriken visualisiert
              dargestellt}
}
\newglossaryentry{Templates}
{
  name=Templates,
  description={Vorlage für Dokumente, die mit Inhalt
               gefüllt und auch nach Belieben angepasst werden können. Dies
               kann zum Beispiel die Planung eines Graphen sein oder einer
               Weboberfläche}
}
\newglossaryentry{IRC-Channel}
{
  name=IRC Channel,
  description={Internet Relay Chat ist genau wie \gls{HTTPS} oder \gls{LDAP}
              ein Netzwerkprotokoll. Es ermöglicht die Kommunikation über einen
              Chat. Hier hat der Anwender die Möglichkeit, sowohl privat als
              auch öffentlich mit anderen Mitgliedern zu schreiben. Um den Chat
              verwenden zu können, wird eine Client-Software, wie zum Beispiel
              weechat, benötigt}
}
\newglossaryentry{Agents}
{
  name=Agents,
  description={Ist meist ein Programm, welches bestimmte Abläufe selbständig
              ohne weitere Eingriffe des Benutzers überwachen kann. Sobald
              hier ein bestimmter Wert errreicht wurde, kann auf
              verschiedene Aktionen zurückgegriffen werden}
}
\newglossaryentry{CSV}
{
  name=CSV,
  description={Comma Seperated Values, Kommaseparierte Daten. Ursprünglich
               wurde nur das Komma als Delimiter genutzt, mittlerweile sind
               verschiedene Trennzeichen zulässig. Eine Kopfzeile, welche die
               Daten beschreibt, ist optional~\cite{RFC4180}}
}
\newglossaryentry{Carbon}
{
  name=Carbon,
  description={Carbon ist eine Software aus dem Graphite-Projekt. Sie empfängt
               Metriken aus dem Netzwerk und schreibt diese in eine Datenbank}
}
\newglossaryentry{Backlog}
{
  name=Backlog,
  description={Begriff aus der Softwareentwicklung und aus dem
               Projektmanagement. Bezeichnet die Menge an gefundenen Fehler,
               die noch nicht behoben worden sind, sowie die Menge der
               unbearbeiteten Änderungswünsche (Featurerequests)}
}
\newglossaryentry{XML}
{
  name=XML,
  description={Extensible Markup Language, eine erweiterbare
               Auszeichnungssprache für strukturierte
               Daten~\cite{xml_definition}}
}
\newglossaryentry{Kibana}
{
  name=Kibana,
  description={Webinterface zum Visualisieren von Logs aus einer
               Elasticsearch-Datenbank}
}
\newglossaryentry{EVA}
{
  name=EVA,
  description={Eingabe, Verarbeitung, Ausgabe. Ein Prinzip aus der
               Datenverarbeitung}
}
\newglossaryentry{Peer-to-Peer}
{
  name=Peer-to-Peer,
  description={Begriff aus der Computernetzwerkstruktur. Es beschreibt die
               Verbindungsart zwischen zwei oder mehreren Computern in einem
               Netzwerk. Dabei kommunizieren zwei oder mehr Teilnehmer über
               eine direkte Verbindung miteinander. In einem Peer-to-Peer
               Netzwerk sind alle Teilnehmer gleichgestellt, die Alternative
               ist das Client-Server-Modell}
}
\newglossaryentry{Client-Server}
{
  name=Client-Server,
  description={Begriff aus der Computer Netzwerkstruktur. Es beschreibt die
               Verbindungsart zwischen zwei oder mehreren Computern in einem
               Netzwerk. Dabei kommuniziert ein oder mehrere Teilnehmer mit
               einem zentral definierten Server. In einem Client-Server-Modell
               sind die Teilnehmer immer auf den Server angewiesen, da er
               benötigte Dienste im Netzwerk anbietet}
}
\newglossaryentry{Partitionen}
{
  name=Partitionen,
  description={Eine Partition in einem Datenbankmanagementsystem definiert
               einen Umgang mit großen Datenmengen, indem eine einzelne Tabelle
               in Tochtertabellen aufgeteilt wird. Dabei werden die Datenmengen
               mit Hife von Vererbungen auf mehrere abhängige Tabellen
               geteilt}
}
\newglossaryentry{Ringtopologie}
{
  name=Ringtopologie,
  description={In einer Ringtopologie ist jede Station mit ihrern direkten
               Nachbarstation verbunden. Entsprechend empfängt eine Station im
               Ring die Daten und reicht sie an den Nachbarn weiter. Der
               Nachrichtenumlauf ist dabei vorgegeben}
}
\newglossaryentry{iowait}
{
  name=iowait,
  description={Ein iowait beschreibt, ob ein Prozessor mit der Verarbeitung von
              Aufgaben warten muss, weil der Datenspeicher (HDDs, SSDs) beim
              Lesen oder Schreiben nicht hinterherkommt}
}
\newglossaryentry{Interrupt}
{
  name=Interrupt,
  description={Ein Interrupt ist die vorübergehende Unterbrechung eines
               bestimmten Prozesses, damit andere Prozesse, welche wichtiger
               sind, Vorrang bekommen. Dieser Interrupt wird durch die Hardware
               ausgelöst}
}
\newglossaryentry{Soft-IRQ}
{
  name=Soft-IRQ,
  description={Ein Soft-IRQ ist ähnlich wie ein \gls{Interrupt} für die
              Unterbrechung eines bestimmten Prozess zuständig, jedoch wird
              dieser durch einen Software-Interrupt ausgelöst}
}
\newglossaryentry{GitHub}
{
  name=GitHub,
  description={GitHub ist ein Filehoster, welcher für die Veröffentlichung und
               Sicherung von sogenannten
               Software-\glslink{Repository}{Repositorys} verwendet werden
               kann. GitHub basiert auf \gls{Git} und Entwickler können hier
               entweder alleine oder im Team an einem Softwareprojekt
               zusammenarbeiten. Jeder Entwickler kann sein eigenes
               \gls{Repository} erstellen und dieses verändern, bearbeiten oder
               weiterentwickeln}
}
\newglossaryentry{Shared-Nothing-Architektur}
{
  name=Shared-Nothing-Architektur,
  description={Eine mögliche Architektur für verteilte Systeme. Hier besitzt
               jeder Teilnehmer im Cluster alle notwendigen Daten, um den
               Ausfall eines anderen Teilnehmers zu kompensieren. Das
               System ist somit weiterhin komplett erreichbar, allerdings mit
               eingeschränkter Leistung}
}
\newglossaryentry{Maintainer}
{
  name=Maintainer,
  description={zu Deutsch: Erhalter, Verantwortlicher. Im Kontext der
               Softwareentwicklung eine Rolle, die alle Rechte in einem
               \gls{Repository} besitzt. Sie entscheidet, welche Änderungen von
               externen Personen in das Projekt aufgenommen werden. Oftmals
               leiten Maintainer auch ein Softwareprojekt. Projekte können
               mehrere Maintainer haben}
}
\newglossaryentry{Transaktion}
{
  name=Transaktion,
  description={Begriff aus der Datenbanktechnik, eine Transaktion bündelt ein
               oder mehrere Datenbankabfragen}
}
\newglossaryentry{Commit}
{
  name=Commit,
  description={Aus der Softwarentwicklung: Ein Commit bündelt eine oder mehrere
               Änderungen in einem \gls{Repository}. Aus der Datenbanktechnik:
               Ein Commit beendet eine \gls{Transaktion}}
}
\newglossaryentry{Semantic Versioning}
{
  name={Semantic Versioning},
  description={Ein anerkanntes Verfahren zum Definieren von Versionsnummern. Es
               nutzt das Schema Major.Minor.Patch. Nur Releases ab 1.0.0 sind
               für den produktiven Betrieb ausgelegt~\cite{semver}}
}
\newglossaryentry{singlestat}
{
  name={singlestat},
  description={Ein singlestat ist die Zusammenfassung einer Statistik aus
               mehreren Werten. Sie zeigt hier wahlweise den Minimal- oder
               Maximalwert. Diese können in einem Panel visuell dargestellt
               werden}
}
\newglossaryentry{JDBC}
{
  name=JDBC,
  description={Java Database Connectivity, generische Schnittstelle der JVM zur
               Interaktion mit einem \gls{DBMS}}
}
\newglossaryentry{JVM}
{
  name=JVM,
  description={Java Virtual Machine, eine Laufzeitumgebung für Java Bytecode}
}
\newglossaryentry{ODBC}
{
  name=ODBC,
  description={Open Database Connectivity, Standardisierte Schnittstelle für
               ein \gls{DBMS} via \gls{SQL}}
}
\newglossaryentry{Fact}
{
  name=Fact,
  description={Ein Fact ist ein Key-Value-Wert, welcher mit dem Programm Facter
               aus dem laufenden System ermittelt wurde. Die Facter Notation
               erfolgt in \gls{JSON}. Ein Beispiel findet sich
               unter~\ref{lst:facter}}
}
\newglossaryentry{DSL}
{
  name=DSL,
  description={Domain Specific Language, eine Programmiersprache für einen sehr
               spezifischen Einsatzzweck}
}
\newglossaryentry{MRI}
{
  name=MRI,
  description={Referenzimplementierung eines Ruby Interpreters in der
               Programmiersprache C. Die Abkürzung steht für „Matz's Ruby
               Interpreter“, Yukihiro „Matz“ Matsumoto ist der Erfinder von
               Ruby}
}
\newglossaryentry{Middleware}
{
  name=Middleware,
  description={Eine Softwarekomponente, welche Daten zwischen zwei oder
               mehreren verschiedenen Schnittstellen oder Diensten
               transportiert. Oftmals erfolgt auch eine Transformation
               der Daten zwischen zwei verschiedenen Protokollen}
}
\newglossaryentry{Deklarative Programmiersprache}
{
  name={Deklarative Programmiersprache},
  description={Hierbei wird in der Programmiersprache das zu lösende Problem
               beschrieben. Der Interpreter oder Übersetzer der Sprache
               analysiert dies und erarbeitet eigenständig einen Lösungsweg.
               Die bekanntesten Sprachen sind hier \gls{SQL} und puppet. Ein
               alternativer Programmierstil ist
               \glslink{Imperative Programmiersprache}{imperativ}}
}
\newglossaryentry{Imperative Programmiersprache}
{
  name={Imperative Programmiersprache},
  description={Die Sprache besteht aus einer Reihe von spezifischen
               Anweisungen, welche der Reihe nach abgearbeitet werden. Die
               meisten Programmiersprachen arbeiten nach diesem Paradigma}
}
\newglossaryentry{Namespace}
{
  name=Namespace,
  description={Zu deutsch: Namensraum. Variablen sind nur innerhalb eines
               Namespaces gültig. Der gleiche Variablenname kann in einem
               anderen Namespace wiederverwendet werden}
}
\newglossaryentry{Side Effect}
{
  name={Side Effect},
  description={In Deutsch: Seiteneffekt. Der Ablauf eines Programmes ändert
               sich durch unerwarteten externen Einfluss. Dies tritt sehr
               häufig bei der Validierung von Nutzerdaten auf. Es wird ein
               bestimmer Datentyp erwartet aber ein anderer übergeben}
}
\newglossaryentry{CLI}
{
  name=CLI,
  description={Command Line Interface, eine Schnittstelle zu einem Service,
               welche über eine Kommandozeile genutzt wird}
}
\newglossaryentry{Repositorium}
{
  name=Repositorium,
  description={Eine Bibliothek über Schemadaten von Datenbanktabellen und
               den notwendigen Informationen zu den Datenbereinigungs- und
               Transformationsregeln in einem Datenhaltungssystem}
}
\newglossaryentry{cURL}
{
  name=cURL,
  description={Augeschrieben "Client for URL", ist ein Kommandozeilen Programm,
               für die Übertragung von Daten. Hierzu ist keine
               Benutzerinteraktion erforderlich}
}
\newglossaryentry{testkitchen}
{
  name=testkitchen,
  description={Testkitchen ist ein Tool, womit der zu testende Code auf
              mehreren unterschiedlichen Plattformen in einem Container
              getestet werden kann. Hier können Tests, welche mit RSpec
              oder Serverspec geschrieben worden sind ausgeführt und
              getestet werden}
}
\newglossaryentry{serverspec}
{
  name=serverspec,
  description={Hiermit werden RSpec Tests geschrieben, um einen Server
              auf eine Misskonfiguration zu prüfen. Hierzu werden keine
              zusätzlichen Agents auf den einzelnen Server benötigt. Tests
              können über SSH oder WinRM abgesetzt werden}
}
\newglossaryentry{Quality of Service}
{
  name={Quality of Service},
  description={Begriff aus der Datenverarbeitung und der Netzwerktechnik. Daten
               werden in verschiedene Klassen kategorisiert und mit
               unterschiedlicher Priorität behandelt. Besonders bekannt ist
               dies bei der Behandlung von Telefoniedaten im Internet. Diese
               werden immer mit höchster Priorität transportiert}
}
\newglossaryentry{Grok}
{
  name=Grok,
  description={Ein Grok ist ein Filter zum extrahieren und strukturieren von
               Daten. Dieser besteht aus einem oder mehreren regulären
               Ausdrücken. Logstash nutzt diese zum filtern}
}
\newglossaryentry{Hypervisor}
{
  name=Hypervisor,
  description={Eine Software welche Hardware partitioniert und virtuellen
               Maschinen zur Verfügung stellt}
}
