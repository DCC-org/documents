\newglossaryentry{API}
{
  name=API,
  description={Application Programming Interface, ermöglicht eine
               standardisierte Anbindung von externen Softwaresystemen über
               öffentliche Methoden einer Bibliothek~\cite{HowDoAPIsEvolve}}
}
\newglossaryentry{Cloud}
{
  name=Cloud,
  description={Eine Platform die nach dem Client-Server Modell arbeitet. Eine
               oder mehrere bestimmte Dienstleistungen werden hier zur
               Verfügung gestellt (z.b. Mailhosting). Die Verteilung von
               Ressourcen erfolgt dynamisch}
}
\newglossaryentry{Public Cloud}
{
  name=Public Cloud,
  description={Siehe auch \gls{Cloud}, auf dieser Platform teilen sich mehrere
               Benutzer die gleiche physische Infrastruktur}
}
\newglossaryentry{Private Cloud}
{
  name=Private Cloud,
  description={Siehe auch \gls{Cloud}, Hier wird jedem Nutzer eine dedizierte
               Infrastruktur bereitgestellt}
}
\newglossaryentry{JSON}
{
  name=JSON,
  description={JavaScript Object Notation, ein Datenformat mit dem Ziel von
               Menschen und Maschinen verarbeitet werden zu können.
               Standardisiert in~\cite{RFC7159}}
}
\newglossaryentry{Jepsen}
{
  name=Jepsen,
  description={Jepsen ist eine in Clojure geschriebene Bibliothek mit der
               verteilte Datenbanken auf Persistenz und Zuverlässigkeit
               getestet werden können~\cite{Jepsen_Introduction}}
}
\newglossaryentry{Gnocchi}
{
  name=Gnocchi,
  description={Datenbank aus dem OpenStack Projekt zum speichern von Metriken}
}
\newglossaryentry{Unidirektional}
{
  name=Unidirektional,
  description={Eine Punkt zu Punkt Verbindung in der Daten nur in eine Richtung
               fließen}
}
\newglossaryentry{Bidirektional}
{
  name=Bidirektional,
  description={Eine Punkt zu Punkt Verbindung in der Daten in beide Richtung
               fließen}
}
\newglossaryentry{DBMS}
{
  name=DBMS,
  description={Datenbankmanagementsystem, dieses System ist die Schnittstelle
               zwischen gespeicherten Daten und den Anwendern, es stellt
               Operationen wie Datenzugriff und Datenmanipulation bereit.
               Außerdem verwaltet es das Datenbankschema. Das DBMS ermöglicht
               parallele Zugriffe, bietet Datenintegrität sowie administrativen
               Zugriff}
}
\newglossaryentry{SQL}
{
  name=SQL,
  description={Structured Query Language, SQL ist ein Medium um mit einem
               \gls{DBMS} zu kommunizieren~\cite{sumathi2007fundamentals}}
}
\newglossaryentry{Streaming Replication}
{
  name=Streaming Replication,
  description={Hierbei schreibt ein Server seine Änderungen an einer Datenbank
               in eine Logdatei. Die Änderungen an der Datei werden
               kontinuierlich an alle passiven Server übertragen, diese wenden
               die Änderungen ebenfalls an}
}
\newglossaryentry{Active-Passive Prinzip}
{
  name=Active-Passive Prinip,
  description={Eine Möglichkeit des Serverbetriebs. Ein aktiver Server erhält
               alle Anfragen von Clients, ein oder mehrere passive Server sind
               verfügbar. Der aktive Server kann deaktiviert werden und ein
               passiver wird dafür aktiv geschaltet}
}
\newglossaryentry{Pgpool-II}
{
  name=Pgpool-II,
  description={Eine Erweiterung für Postgres. Sie ermöglicht das
               wiederverwenden von Datenbankverbindungen, das hochverfügbare
               Replizieren der Daten sowie das verteilen von
               Leseanfragen~\cite{pgpool}}
}
\newglossaryentry{Postgres-XC}
{
  name=Postgres-XC,
  description={Eine Erweiterung für Postgres. Sie ermöglicht den vertikal
               skalierendes Cluster. Hierbei skaliert nicht nur die Leistung
               für den lesezugriff, sondern auch für den Schreibzugriff. Je
               mehr Nodes man in das Cluster stellt desto höher wird die
               Schreibleistung~\cite{postgres-xc}}
}
\newglossaryentry{Skalierung}
{
  name=Skalierung,
  description={Beschreibt das erhöhen der Ressourcen in einem komplexen
               Softwaresetup. Bei horizontaler Skalierung wird die Leistung
               eines einzelnen Servers erhöht durch ein Hardwareupgrade,
               bei vertikaler Skalierung wird das Softwaresetup auf mehrere
               Server verteilt. In großen Setups wird die vertikale
               Vorgehensweise bevorzugt da sie mehr Leistung bieten kann}
}
\newglossaryentry{Hochverfügbarkeit}
{
  name=Hochverfügbarkeit,
  description={Ein IT-System gilt als verfügbar, wenn es seine definierte
               Aufgabe abarbeiten kann. Als hochverfügbar gilt es, wenn es
               diese Aufgabe trotz Ausfall einer oder mehrerer Komponten noch
               erfüllen kann}
}
\newglossaryentry{Git}
{
  name=Git,
  description={Open Source Tool zur Versionsverwaltung von Dateien}
}
\newglossaryentry{Repository}
{
  name=Repository,
  description={Verzeichnis zum speichern von Objekten. Im Git Kontext: Alle
               logisch zusammengehörenden Dateien eines Projekts}
}
\newglossaryentry{RESTful}
{
  name=RESTful,
  description={REST steht für Representational State Transfer. Es ist ein
               Architekturprinzip für Schnittstellen auf Basis von HTTP.
               RESTful bedeutet, dass eine Applikation das REST Prinzip
               implementiert~\cite{fielding2000architectural}}
}
\newglossaryentry{Lucene}
{
  name=Lucene,
  description={Lucene ist eine Bibliothek der Apache Software Foundation zur
               Volltextsuche. Sie ist in Java geschrieben und erzeugt Indexe
               welche im Anschluss durchsucht werden können. Lucene
               implementiert mehrere Suchalgorithmen. Die Indexgröße beträgt
               20-30\% der Größe der Nutzdaten~\cite{lucene_features}}
}
\newglossaryentry{Garbage Collector}
{
  name=Garbage Collector,
  description={Oft als GC abgekürzt, übersetzt: Müllabfuhr/Müllman. Es
               beschreibt die automatische Speicherwaltung einer Programms.
               Ein Programm allokiert Speicher dynamisch. Der GC prüft
               periodisch den belegten Speicher und gibt nicht mehr benötigte
               Teile wieder frei}
}
\newglossaryentry{HTTP}
{
  name=HTTP,
  description={HTTP, ausgeschrieben HyperText Transfer Protocol ist ein
              Protokoll, welches für die Übertragung von Inhalten wie z.B.
              Webseiteninhalte verwendet wird}
}
\newglossaryentry{HTTPS}
{
  name=HTTPS,
  description={HTTPS,Hypertext Transfer Protocol Secure. Basiert auf
              \gls{HTTP} mit zusätzlicher Transportverschlüsselung}
}
\newglossaryentry{LDAP}
{
  name=LDAP,
  description={Lightweight Directory Access Protocol, ist wie auch \gls{HTTPS}
            ein Netzwerkprotokoll, welches zur Abfrage und Änderung von
            Informationen aus einem Unternehmensverzeichnis verwendet wird.
            LDAP besitzt keine Transportverschlüsselung, diese wurde bei LDAPS
            nachgerüstet}
}
\newglossaryentry{Active Directory}
{
  name=Active Directory,
  description={Ist ein von Microsoft entwickelter Server Dienst, welcher auf
              Windows Server Maschinen installiert werden kann um in einem
              Unternehmensverzeichnis über das Protokoll \gls{LDAP} eine
              Abfrage oder Änderung durchführen zu können}
}
\newglossaryentry{Treiber}
{
  name=Treiber,
  description={Ein Treiber definiert die Schnittstelle zwischen einer
              Software und einer Hardware. Damit eine gewisse Software auf
              eine Hardware zugreifen kann, wird ein Treiber verwendet,
              damit hier die volle Unterstützung zwischen Hard- und Software
              gegeben ist}
}
\newglossaryentry{Datenbanktreiber}
{
  name=Datenbanktreiber,
  description={Wie auch bei \glslink{Treiber}{Treibern}, wird ein
              Datenbanktreiber benötigt, um eine Verbindung mit einer
              Datenbank aufbauen zu können. Zusätzlich wird hier
              das Schema der Inhalte dieser Datenbank überprüft und
              für die Verwendung aufbereitet}
}
\newglossaryentry{Statement}
{
  name=Statement,
  description={Ein Statement ist eine Abfolge von Befehlen, für eine \gls{SQL}
              Abfrage der Datenbank. Mit dieser können Werte abgerufen,
              verändert oder gelöscht werden}
}
\newglossaryentry{Dashboard}
{
  name=Dashboard,
  description={Ein Dashboard ist eine Oberfläche auf einer Webseite um Graphen
              oder Metriken einem User zur Verfügung stellen zu können. Dem
              Anwender werden diese Graphen oder Metriken visualisiert
              dargestellt}
}
\newglossaryentry{Templates}
{
  name=Templates,
  description={Auf Deutsch Vorlage, dient in für Dokumente, die mit Inhalt
               gefüllt und nach belieben auch angepasst werden können. Dies
               kann zum Beispiel die Planung eines Graphen sein oder einer
               Weboberfläche}
}
\newglossaryentry{IRC Channel}
{
  name=IRC Channel,
  description={Internet Relay Chat ist genau wie \gls{HTTPS} oder \gls{LDAP}
              ein Protokoll, welches für die Kommunikation über einen Chat
              ermöglicht. Hier hat der Anwender die Möglichkeit, sowohl
              Privat als auch öffentlich mit anderen Mitgliedern zu schreiben.
              Um den Chat verwenden zu können wird eine Software wie z.B.
              mIRC benötigt}
}
\newglossaryentry{Agents}
{
  name=Agents,
  description={Ist meist ein Programm, welches bestimmte Abläufe selbständig
              ohne weitere Eingriffe des Benutzers überwachen kann. Sobald
              hier ein bestimmter Wert errreicht wurde kann auf
              verschiedene Aktionen zurückgegriffen werden}
}
\newglossaryentry{CSV}
{
  name=CSV,
  description={CSV steht für Comma Seperated Values, Kommaseparierte Daten.
              Ursprünglich wurde nur der Komma als Delimiter genutzt,
              mittlerweile sind verschiedene Trennzeichenzulässig. Eine
              Kopfzeile, welche die Daten beschreibt, ist
              optional~\cite{RFC4180}}
}
\newglossaryentry{Carbon}
{
  name=Carbon,
  description={Carbon ist eine Software aus dem Graphite Projekt. Sie empfängt
               Metriken aus dem Netzwerk und schreibt diese in eine Datenbank}
}
\newglossaryentry{Backlog}
{
  name=Backlog,
  description={Begriff aus der Softwareentwicklung und aus de
               Projektmanagement. Bezeichnet die Menge an gefundenen Fehler die
               noch nicht behoben worden sind, sowie die die Menge der
               unbearbeiteten Änderungswünschen (Featurerequests)}
}
\newglossaryentry{XML}
{
  name=XML,
  description={Extensible Markup Language, eine erweiterbare
               Auszeichnungssprache für strukturierter
               Daten~\cite{xml_definition}}
}
\newglossaryentry{Kibana}
{
  name=Kibana,
  description={Webinterface zum visualisieren von Logs aus einer Elasticsearch
               Datenbank}
}
\newglossaryentry{EVA}
{
  name=EVA,
  description={Eingabe, Verarbeitung, Ausgabe. Ein Prinzip aus der
               Datenverarbeitung}
}
\newglossaryentry{Peer-to-Peer}
{
  name=Peer-to-Peer,
  description={Begriff aus der Computer Netzwerkstruktur. Es beschreibt die
               Verbindungsart zwischen zwei oder mehreren Computern in einem
               Netzwerk. Dabei kommunizieren zwei oder mehr Teilnehmer über
               eine direkte Verbindung miteinander. In einem Peer to Peer
               Netzwerk sind alle Teilnehmer gleichgestellt, die Alternative
               ist das Client-Server Modell}
}
\newglossaryentry{Client-Server}
{
  name=Client-Server,
  description={Begriff aus der Computer Netzwerkstruktur. Es beschreibt die
               Verbindungsart zwischen zwei oder mehreren Computern in einem
               Netzwerk. Dabei kommuniziert ein oder mehrere Teilnehmer mit
               einem zentral definierten Server. In einem Client-Server Modell
               sind die Teilnehmer immer auf den Server angewiesen, da er
               benötigte Dienste im Netzwerk anbietet}
}
\newglossaryentry{Partitionen}
{
  name=Partitionen,
  description={Eine Partitione in einem Datenbankmanagementsystem definiert
               einen Umgang mit großen Datenmengen indem eine einzelne Tabelle
               in Tochtertabellen aufgeteilt wird. Dabei werden die Datenmengen
               mit Hife von Vererbungen auf mehrere abhängige Tabellen
               geteilt}
}
\newglossaryentry{Ringtopologie}
{
  name=Ringtopologie,
  description={In einer Ringtopologie ist jede Station mit ihrern direkten
               Nachbarstation verbunden. Entsprechend empfängt eine Station im
               Ring die Daten und reicht sie an den Nachbarn weiter. Der
               Nachrichtenumlauf ist dabei vorgegeben}
}
\newglossaryentry{iowait}
{
  name=iowait,
  description={Ein iowait beschreibt, ob ein Prozessor mit der Verarbeitung von
              Aufgaben warten muss, weil der Datenspeicher (HDDs, SSDs) mit
              lesen oder schreiben nicht hinterherkommen}
}
\newglossaryentry{Interrupt}
{
  name=Interrupt,
  description={Ein Interrupt ist die vorrübergehende Unterbrechung eines
               bestimmten Prozesses, damit andere Prozesse, welche wichtiger
               sind Vorrang bekommen. Dieser Interrupt wird durch die Hardware
               ausgelöst}
}
\newglossaryentry{Soft-IRQ}
{
  name=Soft-IRQ,
  description={Ein Soft-IRQ ist ähnlich wie ein \gls{Interrupt} für die
              Unterbrechung eines bestimmten Prozess zuständig, jedoch wird
              dieser durch einen Software Interrupt ausgelöst}
}
\newglossaryentry{GitHub}
{
  name=GitHub,
  description={GitHub ist ein Filehoster, welcher für die Veröffentlichung und
               Sicherung von sogenannten
               Software-\glslink{Repository}{Repositorys} verwendet werden kann.
               GitHub basiert auf \gls{Git} und Entwickler können hier entweder
               alleine oder im Team an einem Softwareprojekt zusammen arbeiten.
               Jeder Entwickler kann sein eigenes \gls{Repository} erstellen
               und dieses verändern, bearbeiten oder weiterentwickeln}
}
\newglossaryentry{Shared-Nothing-Architektur}
{
  name=Shared-Nothing-Architektur,
  description={Eine mögliche Architektur für verteilte Systeme. Hier besitzt
               jeder Teilnehmer im Cluster alle notwendigen Daten um beim
               Ausfall eines anderen Teilnehmers dies zu kompensieren. Das
               System ist somit weiterhin komplett erreichbar, allerdings mit
               eingeschränkter Leistung}
}
