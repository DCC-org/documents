\newglossaryentry{API}
{
  name=API,
  description={Application Programming Interface, ermöglicht eine
               standardisierte Anbindung von externen Softwaresystemen über
               öffentliche Methoden einer Bibliothek.\cite{HowDoAPIsEvolve}}
}
\newglossaryentry{Cloud}
{
  name=Cloud,
  description={Eine Platform die nach dem Client-Server Modell arbeitet. Eine
               oder mehrere bestimmte Dienstleistungen werden hier zur
               Verfügung gestellt (z.b. Mailhosting). Die Verteilung von
               Ressourcen erfolgt dynamisch.}
}
\newglossaryentry{Public Cloud}
{
  name=Public Cloud,
  description={Siehe auch \gls{Cloud}, auf dieser Platform teilen sich mehrere
               Benutzer die gleiche physische Infrastruktur.}
}
\newglossaryentry{Private Cloud}
{
  name=Private Cloud,
  description={Siehe auch \gls{Cloud}, Hier wird jedem Nutzer eine dedizierte
               Infrastruktur bereitgestellt.}
}
\newglossaryentry{JSON}
{
  name=JSON,
  description={JavaScript Object Notation, ein Datenformat mit dem Ziel von
               Menschen und Maschinen verarbeitet werden zu können.
               Standartisiert in \cite{RFC7159}.}
}
\newglossaryentry{Jepsen}
{
  name=Jepsen,
  description={Jepsen ist eine in Clojure geschriebene Bibliothek mit der
               verteilte Datenbanken auf Persistenz und Zuverlässigkeit
               getestet werden können \cite{Jepsen_Introduction}.}
}
\newglossaryentry{Gnocchi}
{
  name=Gnocchi,
  description={Datenbank aus dem OpenStack Projekt zum speichern von Metriken}
}
\newglossaryentry{Unidirektional}
{
  name=Unidirektional,
  description={Eine Punkt zu Punkt Verbindung in der Daten nur in eine Richtung
               fließen.}
}
\newglossaryentry{Bidirektional}
{
  name=Bidirektional,
  description={Eine Punkt zu Punkt Verbindung in der Daten in beide Richtung
               fließen.}
}
\newglossaryentry{SQL}
{
  name=SQL,
  description={Structured Query Language, SQL ist ein Medium um mit einem
               \gls{DBMS} zu kommunizieren \cite{sumathi2007fundamentals}.}
}
\newglossaryentry{DBMS}
{
  name=DBMS,
  description={Datenbankmanagementsystem, dieses System ist die Schnittstelle
               zwischen gespeicherten Daten und den Anwendern, es stellt
               Operationen wie Datenzugriff und Datenmanipulation bereit.
               Außerdem verwaltet es das Datenbankschema. Das DBMS ermöglicht
               parallele Zugriffe, bietet Datenintegrität sowie administrativen
               Zugriff.}
}
\newglossaryentry{Streaming Replication}
{
  name=Streaming Replication,
  description={Hierbei schreibt ein Server seine Änderungen an einer Datenbank
               in eine Logdatei. Die Änderungen an der Datei werden
               kontinuierlich an alle passiven Server übertragen, diese wenden
               die Änderungen ebenfalls an.}
}
\newglossaryentry{Active-Passive Prinzip}
{
  name=Active-Passive Prinip,
  description={Eine Möglichkeit des Serverbetriebs. Ein aktiver Server erhält
               alle Anfragen von Clients, ein oder mehrere passive Server sind
               verfügbar. Der aktive Server kann deaktiviert werden und ein
               passiver wird dafür aktiv geschaltet.}
}
\newglossaryentry{Pgpool-II}
{
  name=Pgpool-II,
  description={Eine Erweiterung für Postgres. Sie ermöglicht das wiederverwenden
               von Datenbankverbindungen, das hochverfügbare Replizieren der
               Daten sowie das verteilen von Leseanfragen \cite{pgpool}.}
}
\newglossaryentry{Postgres-XC}
{
  name=Postgres-XC,
  description={Eine Erweiterung für Postgres. Sie ermöglicht den vertikal
               skalierendes Cluster. Hierbei skaliert nicht nur die Leistung
               für den lesezugriff, sondern auch für den Schreibzugriff. Je
               mehr Nodes man in das Cluster stellt desto höher wird die
               Schreibleistung \cite{postgres-xc}.}
}
